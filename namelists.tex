\documentclass{article}
\usepackage[a4paper,margin=0.7in,landscape]{geometry}
\usepackage{booktabs}
\usepackage{longtable}
\usepackage{enumitem}
\setlist[itemize]{noitemsep, topsep=0pt}
\setlength{\parindent}{0pt}
\makeatletter
\newcommand\hlinenobreak{%
  \multispan\LT@cols
  \unskip\leaders\hrule\@height\arrayrulewidth\hfill\\*}\makeatother
\begin{document}
% Amplitudes_2pi     code/init/lowPhoton/twoPi_production/gamma2pi_Amplitudes.f90

\begin{longtable}{llll}
\toprule
\textbf{\large{Amplitudes\_2pi}} & \multicolumn{3}{l}{\footnotesize{code/init/lowPhoton/twoPi\_production/gamma2pi\_Amplitudes.f90}}\\*
\midrule
\endfirsthead
\midrule
\endhead
inMedium\_delta\_width & \begin{minipage}[t]{2cm}logical\end{minipage} & \begin{minipage}[t]{2cm}.false.\end{minipage} & \begin{minipage}[t]{12cm}turn on/off the in-medium-width of the delta\end{minipage}\\*
\midrule
inMedium\_delta\_potential & \begin{minipage}[t]{2cm}logical\end{minipage} & \begin{minipage}[t]{2cm}.false.\end{minipage} & \begin{minipage}[t]{12cm}turn on/off the in-medium-potential of the delta\end{minipage}\\*
\midrule
inMedium\_nucleon\_potential & \begin{minipage}[t]{2cm}logical\end{minipage} & \begin{minipage}[t]{2cm}.false.\end{minipage} & \begin{minipage}[t]{12cm}turn on/off the in-medium-potential of the nucleon\end{minipage}\\*
\midrule
inMedium\_pion\_potential & \begin{minipage}[t]{2cm}logical\end{minipage} & \begin{minipage}[t]{2cm}.false.\end{minipage} & \begin{minipage}[t]{12cm}turn on/off the in-medium-potential of the pion\end{minipage}\\*
\midrule
buuPotential & \begin{minipage}[t]{2cm}logical\end{minipage} & \begin{minipage}[t]{2cm}.true.\end{minipage} & \begin{minipage}[t]{12cm}use buu potentials, else constants\end{minipage}\\*
\bottomrule
\end{longtable}
{ }



% AnalyzeSpectra     code/analysis/analyzeSpectra.f90

\begin{longtable}{llll}
\toprule
\textbf{\large{AnalyzeSpectra}} & \multicolumn{3}{l}{\footnotesize{code/analysis/analyzeSpectra.f90}}\\*
\midrule
\endfirsthead
\midrule
\endhead
realID & \begin{minipage}[t]{2cm}logical, dimension(1:122)\end{minipage} & \begin{minipage}[t]{2cm}.false.\end{minipage} & \begin{minipage}[t]{12cm}Switch on/off the output for specific particle IDs from the real particles vector\end{minipage}\\*
\midrule
pertID & \begin{minipage}[t]{2cm}logical, dimension(1:122)\end{minipage} & \begin{minipage}[t]{2cm}.false.\end{minipage} & \begin{minipage}[t]{12cm}Switch on/off the output for specific particle IDs from the pert particles vector\end{minipage}\\*
\bottomrule
\end{longtable}
{ }



% angular_distribution     code/collisions/phaseSpace/winkelVerteilung.f90

\begin{longtable}{llll}
\toprule
\textbf{\large{angular\_distribution}} & \multicolumn{3}{l}{\footnotesize{code/collisions/phaseSpace/winkelVerteilung.f90}}\\*
\midrule
\endfirsthead
\midrule
\endhead
deltaPWave & \begin{minipage}[t]{2cm}logical\end{minipage} & \begin{minipage}[t]{2cm}.true.\end{minipage} & \begin{minipage}[t]{12cm}Switch for P-Wave decay of delta in pion nucleon Only relevant for deltas which are produced in pion-nucleon collisions. $\rightarrow$ see also master\_2body Values:\begin{itemize}\leftmargin0em\itemindent0pt\item .false.= isotropic in CM-Frame\item .true. = 1+3*cos(theta)**2 in CM Frame (theta is angle of producing   pion to outgoing pion)\end{itemize}\end{minipage}\\*
\midrule
pionNucleon\_backward & \begin{minipage}[t]{2cm}logical\end{minipage} & \begin{minipage}[t]{2cm}.true.\end{minipage} & \begin{minipage}[t]{12cm}Switch for backward peaked pion nucleon cross section:\begin{itemize}\leftmargin0em\itemindent0pt\item .true.= use backward peaked distribution\item .false.= isotropic\end{itemize}\end{minipage}\\*
\midrule
pionNucleon\_backward\_exponent & \begin{minipage}[t]{2cm}real\end{minipage} & \begin{minipage}[t]{2cm}26.5\end{minipage} & \begin{minipage}[t]{12cm}Exponent for backward peaked pion nucleon cross section. Distribution=(coeff-cos(theta))**exponent*(pole-sqrt(s)/pole) Only used if pionNucleon\_backward=.true. .\end{minipage}\\*
\midrule
pionNucleon\_backward\_coeff & \begin{minipage}[t]{2cm}real\end{minipage} & \begin{minipage}[t]{2cm}1.9\end{minipage} & \begin{minipage}[t]{12cm}Exponent for backward peaked pion nucleon cross section. Distribution=(coeff-cos(theta))**exponent*(pole-sqrt(s)/pole) Only used if pionNucleon\_backward=.true. .\end{minipage}\\*
\midrule
rho\_pipi\_nonIsotropic & \begin{minipage}[t]{2cm}logical\end{minipage} & \begin{minipage}[t]{2cm}.true.\end{minipage} & \begin{minipage}[t]{12cm}Switch for non-isotropic rho $\rightarrow$ pi pi decay:\begin{itemize}\leftmargin0em\itemindent0pt\item .false.= isotropic in CM-Frame\item .true. = non-isotropic\end{itemize}\end{minipage}\\*
\midrule
NNisotropic & \begin{minipage}[t]{2cm}logical\end{minipage} & \begin{minipage}[t]{2cm}.false.\end{minipage} & \begin{minipage}[t]{12cm}if .true.: set isotropic nucleon-nucleon elastic cross section\end{minipage}\\*
\midrule
iParam\_gammaNVN & \begin{minipage}[t]{2cm}integer\end{minipage} & \begin{minipage}[t]{2cm}3\end{minipage} & \begin{minipage}[t]{12cm}for gamma N $\rightarrow$ V N events, this parameter is given to the routine vecmesa and selects there, how dsigma/dt is calculated. Only if iParam\_gammaNVN $>$= 0 the default value of that routine is overwritten.\\ Possible values:\begin{itemize}\leftmargin0em\itemindent0pt\item 0: 'old' parametrisation for gammaN$\rightarrow$VN (cf. Effenberger PhD):   dsigma/dt \~{} exp(Bt).   Slope paremeter B according ABBHHM collab, PR 175, 1669 (1968).\item 1: Pythia parametrisation:   Slope parameter B=2*b\_p+2*b\_V+4*s**eps-4.2\item 2: 'Donnachie, Landshoff'   Select t according dsig/dt as given by VecMesWinkel/dsigdt, not   by a given slope parameter\item 3: as 1, but for rho and W$<$\~{}6GeV slope parameter adjusted   according CLAS experimental data [Morrow et al, EPJ A39, 5 (2009)]\item 4: Muehlich PhD, Appendix E\item 5: Rho0 Toy Init\item 6: Rho0 Toy Init: Fit to PYTHIA-VMD\item 7: Flat (not exp.)\end{itemize} cf. VecMesWinkel/vecmesa for a detailed description.\end{minipage}\\*
\midrule
NN\_NR\_noniso & \begin{minipage}[t]{2cm}logical\end{minipage} & \begin{minipage}[t]{2cm}.false.\end{minipage} & \begin{minipage}[t]{12cm}If .true., use non-isotropic angular distr. for NN $\rightarrow$ NR, according to dsigma/dt = b/t**a.\end{minipage}\\*
\bottomrule
\end{longtable}
{ }



% annihilation     code/collisions/twoBodyReactions/annihilation/Annihilation.f90

\begin{longtable}{llll}
\toprule
\textbf{\large{annihilation}} & \multicolumn{3}{l}{\footnotesize{code/collisions/twoBodyReactions/annihilation/Annihilation.f90}}\\*
\midrule
\endfirsthead
\midrule
\endhead
model & \begin{minipage}[t]{2cm}integer\end{minipage} & \begin{minipage}[t]{2cm}2\end{minipage} & \begin{minipage}[t]{12cm}Switch between the models of annihilation:\begin{itemize}\leftmargin0em\itemindent0pt\item 1 -- string based model,\item 2 -- statistical model\end{itemize}\end{minipage}\\*
\midrule
position\_flag & \begin{minipage}[t]{2cm}integer\end{minipage} & \begin{minipage}[t]{2cm}1\end{minipage} & \begin{minipage}[t]{12cm}Switch between the choices of position of outgoing mesons:\begin{itemize}\leftmargin0em\itemindent0pt\item 1 -- at the c.m. of the baryon and antibaryon,\item 2 -- at the antibaryon position\end{itemize}\end{minipage}\\*
\bottomrule
\end{longtable}
{ }



% barAntiBar_input     code/collisions/twoBodyReactions/baryonBaryon/barAntiBar.f90

\begin{longtable}{llll}
\toprule
\textbf{\large{barAntiBar\_input}} & \multicolumn{3}{l}{\footnotesize{code/collisions/twoBodyReactions/baryonBaryon/barAntiBar.f90}}\\*
\midrule
\endfirsthead
\midrule
\endhead
fact\_LambdaBar & \begin{minipage}[t]{2cm}real\end{minipage} & \begin{minipage}[t]{2cm}1.\end{minipage} & \begin{minipage}[t]{12cm}Enhancement factor of pbar p $\rightarrow$ Lambda LambdaBar cross section (for larger statistics)\end{minipage}\\*
\midrule
fact\_JPsi & \begin{minipage}[t]{2cm}real\end{minipage} & \begin{minipage}[t]{2cm}1.\end{minipage} & \begin{minipage}[t]{12cm}Enhancement factor of pbar p $\rightarrow$ J/Psi cross section (for larger statistics)\end{minipage}\\*
\midrule
fact\_JPsi\_width & \begin{minipage}[t]{2cm}real\end{minipage} & \begin{minipage}[t]{2cm}1.\end{minipage} & \begin{minipage}[t]{12cm}Enhancement factor of the J/Psi total width (for larger statistics)\end{minipage}\\*
\midrule
useAnni & \begin{minipage}[t]{2cm}logical\end{minipage} & \begin{minipage}[t]{2cm}.true.\end{minipage} & \begin{minipage}[t]{12cm}Flag whether to perform Baryon-Antibarion annihilation or not at all\end{minipage}\\*
\bottomrule
\end{longtable}
{ }



% barBar_barBar     code/collisions/twoBodyReactions/baryonBaryon/barBar_barBar.f90

\begin{longtable}{llll}
\toprule
\textbf{\large{barBar\_barBar}} & \multicolumn{3}{l}{\footnotesize{code/collisions/twoBodyReactions/baryonBaryon/barBar\_barBar.f90}}\\*
\midrule
\endfirsthead
\midrule
\endhead
mat\_NR & \begin{minipage}[t]{2cm}real, dimension(Delta:F37\_1950)\end{minipage} & \begin{minipage}[t]{2cm}\dots\end{minipage} & \begin{minipage}[t]{12cm}Squared matrix elements M**2/16pi for N N $\rightarrow$ N R. See http://arxiv.org/abs/1203.3557.\end{minipage}\\*
\midrule
mat\_DR & \begin{minipage}[t]{2cm}real, dimension(Delta:F37\_1950)\end{minipage} & \begin{minipage}[t]{2cm}\dots\end{minipage} & \begin{minipage}[t]{12cm}Squared matrix elements M**2/16pi for N N $\rightarrow$ Delta R. See http://arxiv.org/abs/1203.3557.\end{minipage}\\*
\midrule
icugnon & \begin{minipage}[t]{2cm}integer\end{minipage} & \begin{minipage}[t]{2cm}1\end{minipage} & \begin{minipage}[t]{12cm}Switch for nucleon nucleon $\rightarrow$ nucleon nucleon cross sections:\begin{itemize}\leftmargin0em\itemindent0pt\item 0=old parametrization\item 1=new parametrization (Alexej Larionov, Cugnon)\end{itemize}\end{minipage}\\*
\midrule
use\_ND\_ND\_model & \begin{minipage}[t]{2cm}logical\end{minipage} & \begin{minipage}[t]{2cm}.false.\end{minipage} & \begin{minipage}[t]{12cm}Switch for delta nucleon $\rightarrow$ delta nucleon cross sections:\begin{itemize}\leftmargin0em\itemindent0pt\item false=old parametrization\item true =one pion exchange model (Effenberger, Buss)\end{itemize}\end{minipage}\\*
\midrule
new\_NR\_NR & \begin{minipage}[t]{2cm}logical\end{minipage} & \begin{minipage}[t]{2cm}.true.\end{minipage} & \begin{minipage}[t]{12cm}\begin{itemize}\leftmargin0em\itemindent0pt\item .false.= Switch off the NR$\rightarrow$ NR improvement   (improvement= better NN$\leftrightarrow$NN fit is being used)\item only for debugging or comparing\end{itemize}\end{minipage}\\*
\midrule
NR\_NR\_massSHIFT & \begin{minipage}[t]{2cm}logical\end{minipage} & \begin{minipage}[t]{2cm}.false.\end{minipage} & \begin{minipage}[t]{12cm}\begin{itemize}\leftmargin0em\itemindent0pt\item .true.= Shift the srts in NR$\rightarrow$ NR elastic collisions.\end{itemize}\end{minipage}\\*
\midrule
oldOset\_treatment & \begin{minipage}[t]{2cm}logical\end{minipage} & \begin{minipage}[t]{2cm}.false.\end{minipage} & \begin{minipage}[t]{12cm}\begin{itemize}\leftmargin0em\itemindent0pt\item .true.= Use the old treatment for the Oset Delta width:   Put everything into 3-body.\item only for debugging or comparing\end{itemize}\end{minipage}\\*
\midrule
etafac & \begin{minipage}[t]{2cm}real\end{minipage} & \begin{minipage}[t]{2cm}6.5\end{minipage} & \begin{minipage}[t]{12cm}Parameter for enhancement of p n $\rightarrow$ N*(1535) N, relative to p p $\rightarrow$ N*(1535) N, in order to enhance eta production in pn collisions. See Calen et al., PRC 58 (1998) 2667.\end{minipage}\\*
\midrule
rhofac & \begin{minipage}[t]{2cm}real\end{minipage} & \begin{minipage}[t]{2cm}1.\end{minipage} & \begin{minipage}[t]{12cm}Parameter for enhancement of p n $\rightarrow$ N*(1520) N, relative to p p $\rightarrow$ N*(1520) N, in order to enhance rho production in p n collisions.\end{minipage}\\*
\midrule
neufac & \begin{minipage}[t]{2cm}real\end{minipage} & \begin{minipage}[t]{2cm}1.\end{minipage} & \begin{minipage}[t]{12cm}Parameter for enhancement of p n $\rightarrow$ N R, relative to  p p $\rightarrow$ N R, affecting all resonances.\end{minipage}\\*
\midrule
neufac\_roper & \begin{minipage}[t]{2cm}real\end{minipage} & \begin{minipage}[t]{2cm}2.\end{minipage} & \begin{minipage}[t]{12cm}Parameter for enhancement of p n $\rightarrow$ N N*(1440), relative to p p $\rightarrow$ N N*(1440). See http://arxiv.org/abs/1203.3557.\end{minipage}\\*
\bottomrule
\end{longtable}
{ }



% barBar_barBarMes     code/collisions/twoBodyReactions/baryonBaryon/barBar_barBarMes.f90

\begin{longtable}{llll}
\toprule
\textbf{\large{barBar\_barBarMes}} & \multicolumn{3}{l}{\footnotesize{code/collisions/twoBodyReactions/baryonBaryon/barBar\_barBarMes.f90}}\\*
\midrule
\endfirsthead
\midrule
\endhead
NNpi & \begin{minipage}[t]{2cm}logical\end{minipage} & \begin{minipage}[t]{2cm}.true.\end{minipage} & \begin{minipage}[t]{12cm}Global switch for the N N $\rightarrow$ N N pi contibution\end{minipage}\\*
\midrule
NNpi\_BG & \begin{minipage}[t]{2cm}integer\end{minipage} & \begin{minipage}[t]{2cm}2\end{minipage} & \begin{minipage}[t]{12cm}Switch for the N N $\rightarrow$ N N pi background (s-channel):\begin{itemize}\leftmargin0em\itemindent0pt\item 0 = no BG\item 1 = BG according to Teis\item 2 = BG according to Buss (improves threshold behavior, default)\item 3 = BG according to Weil\end{itemize}\end{minipage}\\*
\midrule
NNV\_BG & \begin{minipage}[t]{2cm}logical\end{minipage} & \begin{minipage}[t]{2cm}.true.\end{minipage} & \begin{minipage}[t]{12cm}Incude a N N $\rightarrow$ N N V background term, where V=omega,phi (in addition to possible resonance contributions).\end{minipage}\\*
\midrule
isofac\_omega & \begin{minipage}[t]{2cm}real\end{minipage} & \begin{minipage}[t]{2cm}1.\end{minipage} & \begin{minipage}[t]{12cm}Isospin enhancement factor for p n $\rightarrow$ p n omega, relative to p p $\rightarrow$ p p omega. Data indicate that this is around 2, while theory predicts even larger values (up to 5). Reference: Barsov et al., EPJ A21 (2004) 521-527.\end{minipage}\\*
\midrule
isofac\_phi & \begin{minipage}[t]{2cm}real\end{minipage} & \begin{minipage}[t]{2cm}1.\end{minipage} & \begin{minipage}[t]{12cm}Isospin enhancement factor for p n $\rightarrow$ p n phi, relative to p p $\rightarrow$ p p phi. Theory predicts values of 3-4, cf.: Kaptari, Kaempfer, Eur.Phys.J. A23 (2005) 291-304.\end{minipage}\\*
\bottomrule
\end{longtable}
{ }



% BarBar_to_barBar_model     code/collisions/twoBodyReactions/baryonBaryon/barbar_to_barbar_model.f90

\begin{longtable}{llll}
\toprule
\textbf{\large{BarBar\_to\_barBar\_model}} & \multicolumn{3}{l}{\footnotesize{code/collisions/twoBodyReactions/baryonBaryon/barbar\_to\_barbar\_model.f90}}\\*
\midrule
\endfirsthead
\midrule
\endhead
couplings\_switch & \begin{minipage}[t]{2cm}integer\end{minipage} & \begin{minipage}[t]{2cm}2\end{minipage} & \begin{minipage}[t]{12cm}Possible values:\begin{itemize}\leftmargin0em\itemindent0pt\item 1 = use couplings according to Dmitriev\item 2 = use couplings according to Pascalutsa (default)\end{itemize}\end{minipage}\\*
\midrule
lambda\_cutoff & \begin{minipage}[t]{2cm}real\end{minipage} & \begin{minipage}[t]{2cm}0.6\end{minipage} & \begin{minipage}[t]{12cm}Cutoff parameter in the form factor for ND$\rightarrow$ND Possible values:\begin{itemize}\leftmargin0em\itemindent0pt\item 0.6 (Dmitriev, default)\item 1.2 (Doenges)\end{itemize}\end{minipage}\\*
\bottomrule
\end{longtable}
{ }



% baryonPotential     code/potential/baryonPotential.f90

\begin{longtable}{llll}
\toprule
\textbf{\large{baryonPotential}} & \multicolumn{3}{l}{\footnotesize{code/potential/baryonPotential.f90}}\\*
\midrule
\endfirsthead
\midrule
\endhead
EQS\_Type & \begin{minipage}[t]{2cm}integer\end{minipage} & \begin{minipage}[t]{2cm}5\end{minipage} & \begin{minipage}[t]{12cm}Switch for equation of state for nucleon resonances with spin 1/2.\\ Parameters for nucleon potentials:\begin{itemize}\leftmargin0em\itemindent0pt\item  0 = nucleon potential is set to zero\item  1 = soft,   momentum dependent, lambda = 2.130 (Teis PhD, K = 215 MeV)\item  2 = hard,   momentum dependent, lambda = 2.126 (Teis PhD, K = 380 MeV)\item  3 = soft,   momentum independent               (Teis PhD, K = 215 MeV)\item  4 = hard,   momentum independent               (Teis PhD, K = 380 MeV)\item  5 = medium, momentum dependent, lambda = 2.130 (Teis PhD, K = 290 MeV)\item  6 = LDA potential (Birger Steinmueller)\item  7 = Deuterium potential Argonne V18 (not usable for eventtypes 'heavyIon' and 'hadron')\item  8 = LDA Potential Welke\item  9 = Buss PhD, Set\#1 (K = 220 MeV, momentum dependent)\item 10 = Buss PhD, Set\#2 (K = 220 MeV, momentum dependent)\item 11 = Buss PhD, Set\#3 (K = 220 MeV, momentum dependent)\item 12 = Shanghai meeting 2014 (soft, momentum independent, K = 240 MeV)\item 13 = slightly modified Cooper potential, central depth = - 67.5 MeV at p=0 (see \#14)\item 14 = Potential fitted by Cooper et al, Fig. 6 in PRC 47 (1993) 297\item 98 = use pre-stored values\item 99 = variable Skyrme : E\_bind, p\_0, U\_0, rho\_0 must be defined!\end{itemize}NOTES\\ References:\begin{itemize}\leftmargin0em\itemindent0pt\item for 1-5,  see the PhD thesis of S. Teis, chapter 3.3.2 / table 3.1\item for 9-11, see the PhD thesis of O. Buss, chapter 7.2.3 / table 7.1\end{itemize}\end{minipage}\\*
\midrule
DeltaPot & \begin{minipage}[t]{2cm}integer\end{minipage} & \begin{minipage}[t]{2cm}1\end{minipage} & \begin{minipage}[t]{12cm}Switch for potential of spin=3/2 resonances:\begin{itemize}\leftmargin0em\itemindent0pt\item 0 = no potential\item 1 = nucleon (spin=1/2) potential times 2/3 [according to Ericson/Weise book]\item 2 = 100 MeV * rho/rhoNull\item 3 = nucleon (spin=1/2) potential\end{itemize}\end{minipage}\\*
\midrule
HypPot & \begin{minipage}[t]{2cm}integer\end{minipage} & \begin{minipage}[t]{2cm}1\end{minipage} & \begin{minipage}[t]{12cm}Switch for potential of hyperons:\begin{itemize}\leftmargin0em\itemindent0pt\item 0 = no potential\item 1 = nucleon (spin=1/2) potential times (3+S)/3 (i.e. according to the   share of the light quarks)\item 2 = nucleon (spin=1/2) potential\end{itemize}\end{minipage}\\*
\midrule
symmetryPotFlag & \begin{minipage}[t]{2cm}integer\end{minipage} & \begin{minipage}[t]{2cm}0\end{minipage} & \begin{minipage}[t]{12cm}Switch for the asymmetry term in the nucleon potential.\\NOTES\\ Possible values:\begin{itemize}\leftmargin0em\itemindent0pt\item 0 = none (default)\item 1 = linear (strength given by 'dsymm')\item 2 = stiffer, Esym=Esym\_rho\_0*U\^{}gamma=31.*U\^{}gamma, gamma=2\item 3 = stiff, linear increasing Esym=Esym\_rho\_0*U=31.*U\item 4 = soft, U\_c=3, can give negative Esym=Esym\_rho\_0*U*(U\_c-U)/(U\_c-1)\end{itemize}\end{minipage}\\*
\midrule
symmetryPotFlag\_Delta & \begin{minipage}[t]{2cm}logical\end{minipage} & \begin{minipage}[t]{2cm}.false.\end{minipage} & \begin{minipage}[t]{12cm}Switch for the asymmetry term in the Delta potential.\\NOTES\\ If .true., a symmetry potential will be used also for the Delta (but only if symmetryPotFlag$>$0). It is closely related to the symmetry potential of the nucleon.\end{minipage}\\*
\midrule
noPerturbativePotential & \begin{minipage}[t]{2cm}logical\end{minipage} & \begin{minipage}[t]{2cm}.false.\end{minipage} & \begin{minipage}[t]{12cm}Switch for potential of perturbative particles. If .true. then perturbative baryons feel no potential.\end{minipage}\\*
\midrule
rho\_0 & \begin{minipage}[t]{2cm}real\end{minipage} & \begin{minipage}[t]{2cm}0.16\end{minipage} & \begin{minipage}[t]{12cm}Nuclear matter density for EQS\_Type=99\\NOTES\begin{itemize}\leftmargin0em\itemindent0pt\item Units : fm\^{}{-3}\end{itemize}\end{minipage}\\*
\midrule
p\_0 & \begin{minipage}[t]{2cm}real\end{minipage} & \begin{minipage}[t]{2cm}0.8\end{minipage} & \begin{minipage}[t]{12cm}momentum for which U(p\_0,rho=rho\_0)=0 for EQS\_Type=99\\NOTES\begin{itemize}\leftmargin0em\itemindent0pt\item Units : GeV\end{itemize}\end{minipage}\\*
\midrule
U\_0 & \begin{minipage}[t]{2cm}real\end{minipage} & \begin{minipage}[t]{2cm}0.075\end{minipage} & \begin{minipage}[t]{12cm}U(p=0,rho=rho\_0) for EQS\_Type=99\\NOTES\begin{itemize}\leftmargin0em\itemindent0pt\item Units : GeV\end{itemize}\end{minipage}\\*
\midrule
bindingEnergy & \begin{minipage}[t]{2cm}real\end{minipage} & \begin{minipage}[t]{2cm}0.016\end{minipage} & \begin{minipage}[t]{12cm}Nuclear matter binding energy for EQS\_Type=99\\NOTES\begin{itemize}\leftmargin0em\itemindent0pt\item Units : GeV\end{itemize}\end{minipage}\\*
\midrule
compressibility & \begin{minipage}[t]{2cm}real\end{minipage} & \begin{minipage}[t]{2cm}0.290\end{minipage} & \begin{minipage}[t]{12cm}Nuclear matter compressibility for EQS\_Type=99\\NOTES\begin{itemize}\leftmargin0em\itemindent0pt\item Units : GeV\end{itemize}\end{minipage}\\*
\midrule
SurfacePotFlag & \begin{minipage}[t]{2cm}logical\end{minipage} & \begin{minipage}[t]{2cm}.false.\end{minipage} & \begin{minipage}[t]{12cm}Switch for the surface term in the nucleon potential.\\NOTES\begin{itemize}\leftmargin0em\itemindent0pt\item Do not use it together with yukawa!\end{itemize}\end{minipage}\\*
\midrule
nLoopReAdjust & \begin{minipage}[t]{2cm}integer\end{minipage} & \begin{minipage}[t]{2cm}10\end{minipage} & \begin{minipage}[t]{12cm}number of iterations, if density is readjusted (cf. type(nucleus)\%ReAdjustForConstBinding)\\NOTES\\ It is necessary to reiterate (at least for momentum dependent potentials), since we calculate the potential for a given pF and then calculate for the radjusting a new pF.\end{minipage}\\*
\midrule
dsymm & \begin{minipage}[t]{2cm}real\end{minipage} & \begin{minipage}[t]{2cm}0.03\end{minipage} & \begin{minipage}[t]{12cm}Parameter for symmetry potential in GeV.\\NOTES\\ Value is only used for symmetryPotFlag = 1\end{minipage}\\*
\bottomrule
\end{longtable}
{ }



% BaryonWidth     code/width/baryonWidth.f90

\begin{longtable}{llll}
\toprule
\textbf{\large{BaryonWidth}} & \multicolumn{3}{l}{\footnotesize{code/width/baryonWidth.f90}}\\*
\midrule
\endfirsthead
\midrule
\endhead
readTable & \begin{minipage}[t]{2cm}logical\end{minipage} & \begin{minipage}[t]{2cm}.true.\end{minipage} & \begin{minipage}[t]{12cm}There is a tabulation of the widths saved in buuinput which is used to initialize ('baryonWidthVacuum.dat.bz2'). If you don't want to use this pre-tabulated input, then you can set "readTable=.false". This is useful for runs on a cluster where you want to minimize input/output. Also it is necessary if the decay channels have been modified (cf. DecayChannels.dat).\end{minipage}\\*
\midrule
writeTable & \begin{minipage}[t]{2cm}logical\end{minipage} & \begin{minipage}[t]{2cm}.false.\end{minipage} & \begin{minipage}[t]{12cm}This flag determines whether we write out a new tabulation of the widths ('baryonWidthVacuum.dat.bz2'). It will only have an effects if readTable == .false. or reading of the tabulation file fails for some reason.\end{minipage}\\*
\bottomrule
\end{longtable}
{ }



% BaryonWidthMedium_tables     code/width/baryonWidthMedium_tables.f90

\begin{longtable}{llll}
\toprule
\textbf{\large{BaryonWidthMedium\_tables}} & \multicolumn{3}{l}{\footnotesize{code/width/baryonWidthMedium\_tables.f90}}\\*
\midrule
\endfirsthead
\midrule
\endhead
minRes & \begin{minipage}[t]{2cm}integer\end{minipage} & \begin{minipage}[t]{2cm}-1000\end{minipage} & \begin{minipage}[t]{12cm}Read the data table starting at this minimal resonance ID. ONLY FOR TESTING!!!\end{minipage}\\*
\midrule
maxRes & \begin{minipage}[t]{2cm}integer\end{minipage} & \begin{minipage}[t]{2cm}1000\end{minipage} & \begin{minipage}[t]{12cm}Read the data table up to a maximum resonance ID. ONLY FOR TESTING!!!\end{minipage}\\*
\midrule
inMediumParameterset & \begin{minipage}[t]{2cm}integer\end{minipage} & \begin{minipage}[t]{2cm}2\end{minipage} & \begin{minipage}[t]{12cm}chooses the parameters for the inMediumWidth (1 electron, 2 neutrino)\end{minipage}\\*
\midrule
onlyNucleon & \begin{minipage}[t]{2cm}logical\end{minipage} & \begin{minipage}[t]{2cm}.false.\end{minipage} & \begin{minipage}[t]{12cm}Only for debugging: only nucleon width is read in.\end{minipage}\\*
\midrule
deltaOset & \begin{minipage}[t]{2cm}logical\end{minipage} & \begin{minipage}[t]{2cm}.false.\end{minipage} & \begin{minipage}[t]{12cm}Use delta width according to Oset et al. NPA 468 (1987)\end{minipage}\\*
\midrule
extrapolateAbsP & \begin{minipage}[t]{2cm}logical\end{minipage} & \begin{minipage}[t]{2cm}.false.\end{minipage} & \begin{minipage}[t]{12cm}if(true) then set absP to maxAbsP if absP is larger\end{minipage}\\*
\midrule
FileNameMask & \begin{minipage}[t]{2cm}character(1000)\end{minipage} & \begin{minipage}[t]{2cm}''\end{minipage} & \begin{minipage}[t]{12cm}The absolute filename of the file containing the in-medium tables.\\ This is only the basename, the final name is given by\\<pre>   '[FileNameMask].iii\_nnn.dat.bz2' where iii is the ID of the particle and nnn=1 or 2\\ possible values:\begin{itemize}\leftmargin0em\itemindent0pt\item if not set, default is '[path\_To\_Input]/inMediumWidth/InMediumWidth'\item if given, but does not contain '/':   default is '[path\_To\_Input]/inMediumWidth/[FileNameMask]'\item otherwise: filename is absolute, including path\end{itemize} NOTE if you want to use the files 'XXX.iii\_nnn.dat.bz2' in the actual directory, give this mask as './XXX'\end{minipage}\\*
\bottomrule
\end{longtable}
{ }



% BaryonWidthVacuum     code/width/baryonWidthVacuum.f90

\begin{longtable}{llll}
\toprule
\textbf{\large{BaryonWidthVacuum}} & \multicolumn{3}{l}{\footnotesize{code/width/baryonWidthVacuum.f90}}\\*
\midrule
\endfirsthead
\midrule
\endhead
use\_cutoff & \begin{minipage}[t]{2cm}logical\end{minipage} & \begin{minipage}[t]{2cm}.true.\end{minipage} & \begin{minipage}[t]{12cm}\begin{itemize}\leftmargin0em\itemindent0pt\item Switch on and off the use of cut off parameters.\item These cut-offs are necessary when working with dispersion relations to   deduce the real part.\end{itemize}\end{minipage}\\*
\midrule
deltaRho\_cutoff & \begin{minipage}[t]{2cm}real\end{minipage} & \begin{minipage}[t]{2cm}0.85\end{minipage} & \begin{minipage}[t]{12cm}\begin{itemize}\leftmargin0em\itemindent0pt\item Cut off parameter for the decay of a resonance into delta rho.\item Units of GeV\end{itemize}\end{minipage}\\*
\midrule
baryon\_cutoff & \begin{minipage}[t]{2cm}real\end{minipage} & \begin{minipage}[t]{2cm}2.0\end{minipage} & \begin{minipage}[t]{12cm}\begin{itemize}\leftmargin0em\itemindent0pt\item Cut off parameter for the decay of a resonance into an unstable baryon   and a meson.\item Units of GeV\end{itemize}\end{minipage}\\*
\midrule
meson\_cutoff & \begin{minipage}[t]{2cm}real\end{minipage} & \begin{minipage}[t]{2cm}1.6\end{minipage} & \begin{minipage}[t]{12cm}\begin{itemize}\leftmargin0em\itemindent0pt\item Cut off parameter for the decay of a resonance into a baryon and an   unstable meson.\item Units of GeV\end{itemize}\end{minipage}\\*
\midrule
Delta\_width & \begin{minipage}[t]{2cm}integer\end{minipage} & \begin{minipage}[t]{2cm}1\end{minipage} & \begin{minipage}[t]{12cm}Select a parametrization for the Delta width:\begin{itemize}\leftmargin0em\itemindent0pt\item 1 = Manley   (GiBUU default, cf. Manley/Saleski, Phys. Rev. D 45, 1992)\item 2 = Dmitriev (Dmitriev/Sushkov/Gaarde, Nucl. Phys. A 459, 1986)\item 3 = Moniz    (Koch/Moniz/Ohtsuka, Ann. of Phys. 154, 1984)\item 4 = Verwest  (Phys. Lett. B 83, 1979)\item 5 = UrQMD    (Bass et al., Prog. Part. Nucl. Phys. 41, 1998)\end{itemize}\end{minipage}\\*
\bottomrule
\end{longtable}
{ }



% BB_BYK     code/collisions/twoBodyReactions/baryonBaryon/barBar_BarHypKaon.f90

\begin{longtable}{llll}
\toprule
\textbf{\large{BB\_BYK}} & \multicolumn{3}{l}{\footnotesize{code/collisions/twoBodyReactions/baryonBaryon/barBar\_BarHypKaon.f90}}\\*
\midrule
\endfirsthead
\midrule
\endhead
enable & \begin{minipage}[t]{2cm}logical\end{minipage} & \begin{minipage}[t]{2cm}.true.\end{minipage} & \begin{minipage}[t]{12cm}Enable the production of BB $\rightarrow$ B Hyperon Kaon channels. B=Nucleon\^{}{0,1},Delta\^{}{-,0.+,++}; Hyperon=Lambda\^{}{0},Sigma\^{}{0,-,+}; Kaon=K\^{}{+,0}\end{minipage}\\*
\midrule
parameter\_set & \begin{minipage}[t]{2cm}integer\end{minipage} & \begin{minipage}[t]{2cm}2\end{minipage} & \begin{minipage}[t]{12cm}Select a particular parameter set for BB$\rightarrow$BYK collisions. Possible values:\begin{itemize}\leftmargin0em\itemindent0pt\item 1 = original Tsushima model: Tsushima et al., PRC59 (1999) 369\item 2 = extended/adjusted model, fitted to HADES data:       Agakishiev et al., arXiv:1404.7011\item 3 = custom parameters based on Tsushima values (as given by the       array 'a' in the jobcard; those values not given in the jobcard       are adopted from Tsushima, i.e. parameter set 1)\item 4 = custom parameters based on HADES values (as given by the       array 'a' in the jobcard; those values not given in the jobcard       are adopted from HADES, i.e. parameter set 2)\end{itemize}\end{minipage}\\*
\midrule
a & \begin{minipage}[t]{2cm}real, dimension(1:Nch)\end{minipage} & \begin{minipage}[t]{2cm}-1.\end{minipage} & \begin{minipage}[t]{12cm}This array contains the "a" parameters (in microbarn) for the 30 primary channels, see:\begin{itemize}\leftmargin0em\itemindent0pt\item Tsushima et al., PRC59 (1999) 369, table III\item Agakishiev et al., arXiv:1404.7011, chapter IV\end{itemize} Note: The values given in the jobcard are only used for parameter\_set = 3 and 4.\end{minipage}\\*
\bottomrule
\end{longtable}
{ }



% Box     code/init/initBox.f90

\begin{longtable}{llll}
\toprule
\textbf{\large{Box}} & \multicolumn{3}{l}{\footnotesize{code/init/initBox.f90}}\\*
\midrule
\endfirsthead
\midrule
\endhead
thermalInit & \begin{minipage}[t]{2cm}logical\end{minipage} & \begin{minipage}[t]{2cm}.false.\end{minipage} & \begin{minipage}[t]{12cm}flag how to initialize\end{minipage}\\*
\midrule
nDens & \begin{minipage}[t]{2cm}real\end{minipage} & \begin{minipage}[t]{2cm}1.0\end{minipage} & \begin{minipage}[t]{12cm}particle density [fm\^{}-3]\end{minipage}\\*
\midrule
ChargeSelection & \begin{minipage}[t]{2cm}integer\end{minipage} & \begin{minipage}[t]{2cm}0\end{minipage} & \begin{minipage}[t]{12cm}define the type of the charge selection:\begin{itemize}\leftmargin0em\itemindent0pt\item 0: only pi0\item 1: 50\% pi+, 50\% pi-\item 2: 33\% for +,0,-\end{itemize}\end{minipage}\\*
\midrule
pInit & \begin{minipage}[t]{2cm}real\end{minipage} & \begin{minipage}[t]{2cm}0.5\end{minipage} & \begin{minipage}[t]{12cm}initial momentum of particles [GeV/c]\end{minipage}\\*
\midrule
BoostZ & \begin{minipage}[t]{2cm}real\end{minipage} & \begin{minipage}[t]{2cm}0.0\end{minipage} & \begin{minipage}[t]{12cm}additional boost for all particles in z-direction\end{minipage}\\*
\midrule
Temp & \begin{minipage}[t]{2cm}real, dimension(1:122)\end{minipage} & \begin{minipage}[t]{2cm}0.0\end{minipage} & \begin{minipage}[t]{12cm}for thermal init: temperature of every meson species in GeV, if larger than 0. otherwise this species is not initialized\end{minipage}\\*
\midrule
Fugacity & \begin{minipage}[t]{2cm}real, dimension(1:122)\end{minipage} & \begin{minipage}[t]{2cm}1.0\end{minipage} & \begin{minipage}[t]{12cm}for thermal init: fugacity of every hadron species.\end{minipage}\\*
\midrule
correctMovingBox & \begin{minipage}[t]{2cm}integer\end{minipage} & \begin{minipage}[t]{2cm}-2\end{minipage} & \begin{minipage}[t]{12cm}switch to indicate, whether a correction of the momenta after initialization should be done to enforce vanishing 3-momenta. possibilities are:\begin{itemize}\leftmargin0em\itemindent0pt\item -2: set to 3,4 according inputGeneral::fullEnsemble\item -1: set to 1,2 according inputGeneral::fullEnsemble\item 0 : no correction\item 1 : global correction (rescaling)\item 2 : per ensemble correction (rescaling)\item 3 : global correction (rotating)\item 4 : per ensemble correction (rotating)\end{itemize}\end{minipage}\\*
\midrule
noAnti & \begin{minipage}[t]{2cm}logical\end{minipage} & \begin{minipage}[t]{2cm}.false.\end{minipage} & \begin{minipage}[t]{12cm}if .true., no antiparticles will be initialized in thermal init\end{minipage}\\*
\midrule
coldMatter & \begin{minipage}[t]{2cm}logical\end{minipage} & \begin{minipage}[t]{2cm}.false.\end{minipage} & \begin{minipage}[t]{12cm}if .true., matter at T=0 will be initialised according rhoP and rhoN\end{minipage}\\*
\midrule
rhoP & \begin{minipage}[t]{2cm}real\end{minipage} & \begin{minipage}[t]{2cm}-99.9\end{minipage} & \begin{minipage}[t]{12cm}proton density for cold matter\end{minipage}\\*
\midrule
rhoN & \begin{minipage}[t]{2cm}real\end{minipage} & \begin{minipage}[t]{2cm}-99.0\end{minipage} & \begin{minipage}[t]{12cm}neutron density for cold matter\end{minipage}\\*
\midrule
useOffShell & \begin{minipage}[t]{2cm}logical\end{minipage} & \begin{minipage}[t]{2cm}.false.\end{minipage} & \begin{minipage}[t]{12cm}initialize cold matter box with offshell nucleons\end{minipage}\\*
\bottomrule
\end{longtable}
{ }



% BoxAnalysis     code/analysis/BoxAnalysis.f90

\begin{longtable}{llll}
\toprule
\textbf{\large{BoxAnalysis}} & \multicolumn{3}{l}{\footnotesize{code/analysis/BoxAnalysis.f90}}\\*
\midrule
\endfirsthead
\midrule
\endhead
do\_Tmunu & \begin{minipage}[t]{2cm}logical\end{minipage} & \begin{minipage}[t]{2cm}.false.\end{minipage} & \begin{minipage}[t]{12cm}Switch for Tmunu output. default: Only one file for all ensemble! you may change this with the flag perEnsemble\_Tmunu\end{minipage}\\*
\midrule
do\_TmunuZ & \begin{minipage}[t]{2cm}logical\end{minipage} & \begin{minipage}[t]{2cm}.false.\end{minipage} & \begin{minipage}[t]{12cm}switch for Tmunu for every z-coordinate\\NOTES\\ This is mutually exclusive with perEnsemble\_Tmunu\end{minipage}\\*
\midrule
perEnsemble\_Tmunu & \begin{minipage}[t]{2cm}logical\end{minipage} & \begin{minipage}[t]{2cm}.false.\end{minipage} & \begin{minipage}[t]{12cm}Switch for Tmunu output. One file per ensemble!\\NOTES\\ this may slow down the execution dramatically, since huge output to the hard drive is induced. You may observe this, if e.g the cpu load drops permanently to 30\%. Thus: switch it on, only if you want it!\\NOTES\\ This is mutually exclusive with do\_TmunuZ\end{minipage}\\*
\midrule
selectTmunuFormat & \begin{minipage}[t]{2cm}integer\end{minipage} & \begin{minipage}[t]{2cm}2\end{minipage} & \begin{minipage}[t]{12cm}select output format of Tmunu (binary encoded):\begin{itemize}\leftmargin0em\itemindent0pt\item 1: ASCII\item 2: Binary\item 3: ASCII + Binary\end{itemize}\end{minipage}\\*
\midrule
do\_P & \begin{minipage}[t]{2cm}logical\end{minipage} & \begin{minipage}[t]{2cm}.false.\end{minipage} & \begin{minipage}[t]{12cm}Switch for dN/p\^{}2 dp output\end{minipage}\\*
\midrule
do\_velrel & \begin{minipage}[t]{2cm}logical\end{minipage} & \begin{minipage}[t]{2cm}.false.\end{minipage} & \begin{minipage}[t]{12cm}Switch for calculating velrel\end{minipage}\\*
\midrule
do\_posrel & \begin{minipage}[t]{2cm}logical\end{minipage} & \begin{minipage}[t]{2cm}.false.\end{minipage} & \begin{minipage}[t]{12cm}Switch for calculating relative distance\end{minipage}\\*
\midrule
do\_Cumulants & \begin{minipage}[t]{2cm}logical\end{minipage} & \begin{minipage}[t]{2cm}.false.\end{minipage} & \begin{minipage}[t]{12cm}Switch for calculating cumulants\end{minipage}\\*
\midrule
useSet & \begin{minipage}[t]{2cm}logical, dimension(nSet)\end{minipage} & \begin{minipage}[t]{2cm}(/ .false., .false., .false., .true., .true. /)\end{minipage} & \begin{minipage}[t]{12cm}Array to indicate, which particle set will be used for output\end{minipage}\\*
\midrule
factorSubBoxVolume & \begin{minipage}[t]{2cm}real\end{minipage} & \begin{minipage}[t]{2cm}0.5\end{minipage} & \begin{minipage}[t]{12cm}Volume of the sub box relative to the full box\end{minipage}\\*
\midrule
do\_DumpPartVec & \begin{minipage}[t]{2cm}logical\end{minipage} & \begin{minipage}[t]{2cm}.false.\end{minipage} & \begin{minipage}[t]{12cm}Switch for writing the whole particle vector to file at the end of the run\end{minipage}\\*
\bottomrule
\end{longtable}
{ }



% Checks     code/run/checks.f90

\begin{longtable}{llll}
\toprule
\textbf{\large{Checks}} & \multicolumn{3}{l}{\footnotesize{code/run/checks.f90}}\\*
\midrule
\endfirsthead
\midrule
\endhead
Do\_CheckDensity & \begin{minipage}[t]{2cm}logical\end{minipage} & \begin{minipage}[t]{2cm}.false.\end{minipage} & \begin{minipage}[t]{12cm}Flag to indicate whether the density check routine should be called.\end{minipage}\\*
\midrule
Do\_CheckCoulomb & \begin{minipage}[t]{2cm}logical\end{minipage} & \begin{minipage}[t]{2cm}.false.\end{minipage} & \begin{minipage}[t]{12cm}Flag to indicate whether the Coulomb check routine should be called.\end{minipage}\\*
\midrule
Do\_CheckFermiSurface & \begin{minipage}[t]{2cm}logical\end{minipage} & \begin{minipage}[t]{2cm}.false.\end{minipage} & \begin{minipage}[t]{12cm}Flag to indicate whether the Fermi-surface check routine should be called.\end{minipage}\\*
\midrule
Do\_CheckRadius & \begin{minipage}[t]{2cm}logical\end{minipage} & \begin{minipage}[t]{2cm}.false.\end{minipage} & \begin{minipage}[t]{12cm}Flag to indicate whether the radius check routine should be called.\end{minipage}\\*
\midrule
Do\_CheckMomentumDensity & \begin{minipage}[t]{2cm}logical\end{minipage} & \begin{minipage}[t]{2cm}.false.\end{minipage} & \begin{minipage}[t]{12cm}Flag to indicate whether the momentum-density check routine should be called.\end{minipage}\\*
\midrule
Do\_CheckEnergyLDA & \begin{minipage}[t]{2cm}logical\end{minipage} & \begin{minipage}[t]{2cm}.false.\end{minipage} & \begin{minipage}[t]{12cm}Flag to indicate whether the local density approximation check routine should be called.\end{minipage}\\*
\midrule
Do\_OccupiedReal & \begin{minipage}[t]{2cm}logical\end{minipage} & \begin{minipage}[t]{2cm}.false.\end{minipage} & \begin{minipage}[t]{12cm}Flag to indicate whether the occupation check routine for the real particle vector should be called.\end{minipage}\\*
\midrule
Do\_OccupiedPert & \begin{minipage}[t]{2cm}logical\end{minipage} & \begin{minipage}[t]{2cm}.false.\end{minipage} & \begin{minipage}[t]{12cm}Flag to indicate whether the occupation check routine for the perturbative particle vector should be called.\end{minipage}\\*
\midrule
Do\_CheckEnergy & \begin{minipage}[t]{2cm}logical\end{minipage} & \begin{minipage}[t]{2cm}.false.\end{minipage} & \begin{minipage}[t]{12cm}Flag to indicate whether the energy check routine should be called.\end{minipage}\\*
\midrule
Do\_TachyonsReal & \begin{minipage}[t]{2cm}logical\end{minipage} & \begin{minipage}[t]{2cm}.false.\end{minipage} & \begin{minipage}[t]{12cm}Flag to indicate whether the tachyon check routine for the real particle vector should be called.\end{minipage}\\*
\midrule
Do\_TachyonsPert & \begin{minipage}[t]{2cm}logical\end{minipage} & \begin{minipage}[t]{2cm}.false.\end{minipage} & \begin{minipage}[t]{12cm}Flag to indicate whether the tachyon check routine for the perturbative particle vector should be called.\end{minipage}\\*
\midrule
TachyonIsBlocking & \begin{minipage}[t]{2cm}logical\end{minipage} & \begin{minipage}[t]{2cm}.false.\end{minipage} & \begin{minipage}[t]{12cm}Select whether the occurrence of a 'tachyon' in the check routines will stop the code or not (error messages will hopefully occur later in the code).\end{minipage}\\*
\midrule
Do\_CheckPertFlag & \begin{minipage}[t]{2cm}logical\end{minipage} & \begin{minipage}[t]{2cm}.true.\end{minipage} & \begin{minipage}[t]{12cm}Flag to indicate whether the flag '\%pert' is set correctly in the particle vectors\end{minipage}\\*
\midrule
Do\_CheckConservation & \begin{minipage}[t]{2cm}logical\end{minipage} & \begin{minipage}[t]{2cm}.false.\end{minipage} & \begin{minipage}[t]{12cm}Flag to indicate whether conservation of energy/momentum, baryon number and strangeness between time steps for the real particles should be checked\end{minipage}\\*
\bottomrule
\end{longtable}
{ }



% coll_BaB     code/collisions/twoBodyReactions/HiEnergy/DoColl_BaB.f90

\begin{longtable}{llll}
\toprule
\textbf{\large{coll\_BaB}} & \multicolumn{3}{l}{\footnotesize{code/collisions/twoBodyReactions/HiEnergy/DoColl\_BaB.f90}}\\*
\midrule
\endfirsthead
\midrule
\endhead
iset & \begin{minipage}[t]{2cm}integer\end{minipage} & \begin{minipage}[t]{2cm}1\end{minipage} & \begin{minipage}[t]{12cm}Switch to choose an initialization of jets:\begin{itemize}\leftmargin0em\itemindent0pt\item 1: phase space distribution, also the charge is conserved (new prescription)\item 2: first jet along inPart(1) momentum, 3-d jet opposite,      others orthogonal, charge is not conserved (old prescription)\end{itemize}\end{minipage}\\*
\bottomrule
\end{longtable}
{ }



% coll_Manni     code/collisions/twoBodyReactions/HiEnergy/DoColl_Manni.f90

\begin{longtable}{llll}
\toprule
\textbf{\large{coll\_Manni}} & \multicolumn{3}{l}{\footnotesize{code/collisions/twoBodyReactions/HiEnergy/DoColl\_Manni.f90}}\\*
\midrule
\endfirsthead
\midrule
\endhead
angDistribution & \begin{minipage}[t]{2cm}integer\end{minipage} & \begin{minipage}[t]{2cm}2\end{minipage} & \begin{minipage}[t]{12cm}Switch to select the angular distribution:\begin{itemize}\leftmargin0em\itemindent0pt\item 1: isotropic\item 2: diquark/quark aligned like baryon/meson\end{itemize}\end{minipage}\\*
\midrule
itry\_max & \begin{minipage}[t]{2cm}integer\end{minipage} & \begin{minipage}[t]{2cm}10\end{minipage} & \begin{minipage}[t]{12cm}maximum number of tries\end{minipage}\\*
\bottomrule
\end{longtable}
{ }



% collCriteria     code/collisions/twoBodyReactions/collisionCriteria.f90

\begin{longtable}{llll}
\toprule
\textbf{\large{collCriteria}} & \multicolumn{3}{l}{\footnotesize{code/collisions/twoBodyReactions/collisionCriteria.f90}}\\*
\midrule
\endfirsthead
\midrule
\endhead
kodama\_evalFrame & \begin{minipage}[t]{2cm}logical\end{minipage} & \begin{minipage}[t]{2cm}.false.\end{minipage} & \begin{minipage}[t]{12cm}Set to .true., this logical will cause the kodama criterion to be evaluated in the laboratory/evaluation frame, not CM frame.\end{minipage}\\*
\bottomrule
\end{longtable}
{ }



% collHistory     code/storage/CollHistory.f90

\begin{longtable}{llll}
\toprule
\textbf{\large{collHistory}} & \multicolumn{3}{l}{\footnotesize{code/storage/CollHistory.f90}}\\*
\midrule
\endfirsthead
\midrule
\endhead
DoCollHistory & \begin{minipage}[t]{2cm}logical\end{minipage} & \begin{minipage}[t]{2cm}.false.\end{minipage} & \begin{minipage}[t]{12cm}Flag to switch on/off the whole Collision History machinery.\\ You may set this variable via your jobcard, namelist "collHistory"\end{minipage}\\*
\bottomrule
\end{longtable}
{ }



% collisionterm     code/collisions/collisionTerm.f90

\begin{longtable}{llll}
\toprule
\textbf{\large{collisionterm}} & \multicolumn{3}{l}{\footnotesize{code/collisions/collisionTerm.f90}}\\*
\midrule
\endfirsthead
\midrule
\endhead
oneBodyProcesses & \begin{minipage}[t]{2cm}logical\end{minipage} & \begin{minipage}[t]{2cm}.true.\end{minipage} & \begin{minipage}[t]{12cm}Switch on/off one-body-induced processes.\end{minipage}\\*
\midrule
twoBodyProcesses & \begin{minipage}[t]{2cm}logical\end{minipage} & \begin{minipage}[t]{2cm}.true.\end{minipage} & \begin{minipage}[t]{12cm}Switch on/off two-body-induced processes.\end{minipage}\\*
\midrule
threeBodyProcesses & \begin{minipage}[t]{2cm}logical\end{minipage} & \begin{minipage}[t]{2cm}.true.\end{minipage} & \begin{minipage}[t]{12cm}Switch on/off three-body-induced processes.\end{minipage}\\*
\midrule
threeMesonProcesses & \begin{minipage}[t]{2cm}logical\end{minipage} & \begin{minipage}[t]{2cm}.false.\end{minipage} & \begin{minipage}[t]{12cm}Switch on/off three-meson-induced processes. These are the backreactions for e.g. omega $\rightarrow$ pi pi pi etc.\end{minipage}\\*
\midrule
threeBarMesProcesses & \begin{minipage}[t]{2cm}logical\end{minipage} & \begin{minipage}[t]{2cm}.false.\end{minipage} & \begin{minipage}[t]{12cm}Switch on/off baryon-meson-meson induced processes. These are the backreactions for e.g. N pi $\rightarrow$ N pi pi etc.\end{minipage}\\*
\midrule
twoPlusOneBodyProcesses & \begin{minipage}[t]{2cm}logical\end{minipage} & \begin{minipage}[t]{2cm}.false.\end{minipage} & \begin{minipage}[t]{12cm}Switch on/off 2+1 body processes (two really colliding particles plus one nearby).\end{minipage}\\*
\midrule
twoBodyProcessesRealReal & \begin{minipage}[t]{2cm}logical\end{minipage} & \begin{minipage}[t]{2cm}.true.\end{minipage} & \begin{minipage}[t]{12cm}Switch on/off two-body-induced processes between two real particles.\end{minipage}\\*
\midrule
twoBodyProcessesRealPert & \begin{minipage}[t]{2cm}logical\end{minipage} & \begin{minipage}[t]{2cm}.true.\end{minipage} & \begin{minipage}[t]{12cm}Switch on/off two-body-induced processes between a real and a perturbative particle.\end{minipage}\\*
\midrule
oneBodyAdditional & \begin{minipage}[t]{2cm}logical\end{minipage} & \begin{minipage}[t]{2cm}.true.\end{minipage} & \begin{minipage}[t]{12cm}Switch on/off additional Pythia one-body-induced processes.\end{minipage}\\*
\midrule
doForceDecay & \begin{minipage}[t]{2cm}logical\end{minipage} & \begin{minipage}[t]{2cm}.true.\end{minipage} & \begin{minipage}[t]{12cm}switch on/off the forced decays at the end of the simulation\\NOTES\begin{itemize}\leftmargin0em\itemindent0pt\item Do not touch, unless you know what you are doing!\item You may set this to .false., if you are e.g. running box calculations   with excited states. Decaying all these particles would need a much   larger particle vector...\end{itemize}\end{minipage}\\*
\midrule
energyCheck & \begin{minipage}[t]{2cm}real\end{minipage} & \begin{minipage}[t]{2cm}0.01\end{minipage} & \begin{minipage}[t]{12cm}Precision of energy check for each collision in GeV.\end{minipage}\\*
\midrule
maxOut & \begin{minipage}[t]{2cm}integer\end{minipage} & \begin{minipage}[t]{2cm}100\end{minipage} & \begin{minipage}[t]{12cm}Maximal number of produced particles in one process. Must not be smaller than 4.\\NOTES\begin{itemize}\leftmargin0em\itemindent0pt\item When using annihilation, must not be smaller than 6\item If using FRITIOF or PYTHIA, should probably larger than 6\item Code stops with an error message, if value chosen too small\end{itemize}\end{minipage}\\*
\midrule
collisionProtocol & \begin{minipage}[t]{2cm}logical\end{minipage} & \begin{minipage}[t]{2cm}.false.\end{minipage} & \begin{minipage}[t]{12cm}Write a protocol of all real-real collisions to the file 'fort.990'. Includes the time, IDs, charges, invariant masses and 3-momenta of both collision partners and an indicator for Pauli blocking.\end{minipage}\\*
\midrule
printPositions & \begin{minipage}[t]{2cm}logical\end{minipage} & \begin{minipage}[t]{2cm}.false.\end{minipage} & \begin{minipage}[t]{12cm}Switch on/off output of positions in real-pert collisions. Produces statistical output.\end{minipage}\\*
\midrule
useStatistics & \begin{minipage}[t]{2cm}logical\end{minipage} & \begin{minipage}[t]{2cm}.false.\end{minipage} & \begin{minipage}[t]{12cm}Generate statistical information using the module statistics.\end{minipage}\\*
\midrule
noNucNuc & \begin{minipage}[t]{2cm}logical\end{minipage} & \begin{minipage}[t]{2cm}.false.\end{minipage} & \begin{minipage}[t]{12cm}Switch on/off perturbative NN reactions.\end{minipage}\\*
\midrule
storeRho0Info & \begin{minipage}[t]{2cm}logical\end{minipage} & \begin{minipage}[t]{2cm}.false.\end{minipage} & \begin{minipage}[t]{12cm}Flag whether in a rho0 decay the particle numbers of the resulting charged pions are stored or not.\end{minipage}\\*
\midrule
storeRho0InfoOnlyDifr & \begin{minipage}[t]{2cm}logical\end{minipage} & \begin{minipage}[t]{2cm}.false.\end{minipage} & \begin{minipage}[t]{12cm}Flag, whether the flag storeRho0Info is valid for all decays or only for rho0, which are marked to be diffractive.\end{minipage}\\*
\midrule
DoJustAbsorptive & \begin{minipage}[t]{2cm}logical\end{minipage} & \begin{minipage}[t]{2cm}.false.\end{minipage} & \begin{minipage}[t]{12cm}If this flag is true, then: for perturbative simulations all final state particles in a collision are set to zero; for real simulations \%event index of incoming hadron is changed in the case of collision, but actual collision is not simulated. This is a way to mimick Glauber like calculations.\\NOTES\\ The "absorption" is done with sigmaTot, not just by sigmaInEl.\end{minipage}\\*
\midrule
annihilate & \begin{minipage}[t]{2cm}logical\end{minipage} & \begin{minipage}[t]{2cm}.false.\end{minipage} & \begin{minipage}[t]{12cm}If this flag is true, then an annihilation of the antibaryons with the closest baryons will be simulated (by hand) starting from annihilationTime.\end{minipage}\\*
\midrule
annihilationTime & \begin{minipage}[t]{2cm}real\end{minipage} & \begin{minipage}[t]{2cm}1000.\end{minipage} & \begin{minipage}[t]{12cm}Time moment (in fm/c) when the annihilation will be started.\\NOTES\\ This flag has an influence only when annihilate = .true. Before annihilationTime all the collision processes are not activated. They start to act (if the corresponding switches oneBodyProcesses,twoBodyProcesses etc. are .true.) only after annihilationTime.\end{minipage}\\*
\midrule
justDeleteDelta & \begin{minipage}[t]{2cm}logical\end{minipage} & \begin{minipage}[t]{2cm}.false.\end{minipage} & \begin{minipage}[t]{12cm}Deletes final-state products in Delta N N $\rightarrow$ NNN and Delta N $\rightarrow$ N N. NOTE: Only for testing and comparing with the old Effenberger code! DO NOT USE OTHERWISE: Violates energy conservation! This switch is meant to simulate the treatment of the Delta in the old code. Only implemented for perturbative runs.\end{minipage}\\*
\midrule
noRecollisions & \begin{minipage}[t]{2cm}logical\end{minipage} & \begin{minipage}[t]{2cm}.false.\end{minipage} & \begin{minipage}[t]{12cm}Outgoing particles of collisions are inserted somewhere in the particle vector. Due to implementation issues, these outgoing particles may interact during the same timestep.\\ Setting this flag to true, the parameter '\%lastCollTime' is checked against the actual time variable and collisions of these particles are excluded.\end{minipage}\\*
\bottomrule
\end{longtable}
{ }



% collReporter     code/collisions/collisionReporter.f90

\begin{longtable}{llll}
\toprule
\textbf{\large{collReporter}} & \multicolumn{3}{l}{\footnotesize{code/collisions/collisionReporter.f90}}\\*
\midrule
\endfirsthead
\midrule
\endhead
UseCollReporter & \begin{minipage}[t]{2cm}logical\end{minipage} & \begin{minipage}[t]{2cm}.false.\end{minipage} & \begin{minipage}[t]{12cm}Enable or disable the collision reporter.\end{minipage}\\*
\midrule
cR\_sizeT & \begin{minipage}[t]{2cm}integer\end{minipage} & \begin{minipage}[t]{2cm}200\end{minipage} & \begin{minipage}[t]{12cm}Number of timestep bins.\end{minipage}\\*
\midrule
cR\_sizeE & \begin{minipage}[t]{2cm}integer\end{minipage} & \begin{minipage}[t]{2cm}100\end{minipage} & \begin{minipage}[t]{12cm}Number of sqrt(s) bins.\end{minipage}\\*
\midrule
cR\_DeltaT & \begin{minipage}[t]{2cm}real\end{minipage} & \begin{minipage}[t]{2cm}0.1\end{minipage} & \begin{minipage}[t]{12cm}Size of timestep bins.\end{minipage}\\*
\midrule
cR\_DeltaE & \begin{minipage}[t]{2cm}real\end{minipage} & \begin{minipage}[t]{2cm}0.1\end{minipage} & \begin{minipage}[t]{12cm}Size of sqrt(s) bins.\end{minipage}\\*
\bottomrule
\end{longtable}
{ }



% ColStat     code/collisions/twoBodyReactions/twoBodyStatistics.f90

\begin{longtable}{llll}
\toprule
\textbf{\large{ColStat}} & \multicolumn{3}{l}{\footnotesize{code/collisions/twoBodyReactions/twoBodyStatistics.f90}}\\*
\midrule
\endfirsthead
\midrule
\endhead
flag\_sqrts & \begin{minipage}[t]{2cm}logical\end{minipage} & \begin{minipage}[t]{2cm}.false.\end{minipage} & \begin{minipage}[t]{12cm}If .true., then the calculation and output of the sqrts distributions from subroutine sqrts\_distribution will be done\end{minipage}\\*
\midrule
flag\_rate & \begin{minipage}[t]{2cm}logical\end{minipage} & \begin{minipage}[t]{2cm}.false.\end{minipage} & \begin{minipage}[t]{12cm}If .true., then the calculation and output of the collision rates from subroutine rate will be done\end{minipage}\\*
\midrule
flag\_varirate & \begin{minipage}[t]{2cm}logical\end{minipage} & \begin{minipage}[t]{2cm}.false.\end{minipage} & \begin{minipage}[t]{12cm}If .true., then the calculation and output of the collision rates from subroutine varirate will be done\end{minipage}\\*
\midrule
sqrts\_mode & \begin{minipage}[t]{2cm}integer\end{minipage} & \begin{minipage}[t]{2cm}1\end{minipage} & \begin{minipage}[t]{12cm}This flag determines the way how sqrt(s) is calculated (if flag\_sqrts = .true.):\begin{itemize}\leftmargin0em\itemindent0pt\item 1 = use vacuum sqrt(s)\item 2 = use in-medium, i.e. full sqrts\end{itemize}\end{minipage}\\*
\midrule
varirate\_chargeZero & \begin{minipage}[t]{2cm}logical\end{minipage} & \begin{minipage}[t]{2cm}.true.\end{minipage} & \begin{minipage}[t]{12cm}If .true., then all charge states are combined together\end{minipage}\\*
\midrule
varirate\_size & \begin{minipage}[t]{2cm}integer\end{minipage} & \begin{minipage}[t]{2cm}100\end{minipage} & \begin{minipage}[t]{12cm}size of array to hold all rates\end{minipage}\\*
\midrule
varirate\_filterPhi & \begin{minipage}[t]{2cm}logical\end{minipage} & \begin{minipage}[t]{2cm}.false.\end{minipage} & \begin{minipage}[t]{12cm}If .true., then only channels involving a phi meson are reported\end{minipage}\\*
\midrule
varirate\_format & \begin{minipage}[t]{2cm}integer\end{minipage} & \begin{minipage}[t]{2cm}1\end{minipage} & \begin{minipage}[t]{12cm}indicate the format to produce the output.\\ It is binary coded:\begin{itemize}\leftmargin0em\itemindent0pt\item 1 : use human readable output\item 2 : use machine readable output (csv format)\end{itemize} So valid values are 1,2,3.\end{minipage}\\*
\bottomrule
\end{longtable}
{ }



% coulomb     code/potential/coulomb/coulomb.f90

\begin{longtable}{llll}
\toprule
\textbf{\large{coulomb}} & \multicolumn{3}{l}{\footnotesize{code/potential/coulomb/coulomb.f90}}\\*
\midrule
\endfirsthead
\midrule
\endhead
coulombFlag & \begin{minipage}[t]{2cm}logical\end{minipage} & \begin{minipage}[t]{2cm}.false.\end{minipage} & \begin{minipage}[t]{12cm}Switch to turn on/off the Coulomb potential. If turned on, also 'symmetryPotFlag' (in namelist 'baryonPotential') needs to be turned on.\end{minipage}\\*
\midrule
magnetFieldFlag & \begin{minipage}[t]{2cm}logical\end{minipage} & \begin{minipage}[t]{2cm}.false.\end{minipage} & \begin{minipage}[t]{12cm}Switch to turn on/off elm. vector potential.\\NOTES\\ The vector potential is not yet fully implemented! Please do not use this option.\end{minipage}\\*
\midrule
cutMomentumPotential & \begin{minipage}[t]{2cm}real\end{minipage} & \begin{minipage}[t]{2cm}0.025\end{minipage} & \begin{minipage}[t]{12cm}If larger than 0, the coulomb potential is set to zero for all particles with momentum larger than minmass**2/(2*cutMomentumPotential) The cut-off is smeared out, if cutMomentumWidth$>$0\\ This cut is needed in non-RMF mode since\\<pre>   m\_eff\^{}2 = (sqrt(m\^{}2+p\^{}2)+U\_C)\^{}2-p\^{}2 can become smaller than zero for\\<pre>   p $>$ U\_C/2 - m\^{}2/(2*U\_C). In this case we have a 'tachyon'.\\NOTES\begin{itemize}\leftmargin0em\itemindent0pt\item for RMF mode you do not need this cut\item This value should correspond to the maximum vale of the Coulomb   potential. Therefore you should readjust this for every nucleus.\item For the pion, we take the mass (0.138) instead of minmass, since here   minmass is zero!\end{itemize}\end{minipage}\\*
\midrule
cutMomentumWidth & \begin{minipage}[t]{2cm}real\end{minipage} & \begin{minipage}[t]{2cm}0.100\end{minipage} & \begin{minipage}[t]{12cm}If cutMomentumPotential$>$0 and cutMomentumWidth$>$0, then the cut-off is smeared by some linear interpolation:\\<pre>   (At-Width) = 1.0 ... (At+Width) = 0.0 with At = minmass**2/(2*cutMomentumPotential) and the width given here in GeV.\\ This width is introduced in order to destroy numerical problems due to sharp edges.\end{minipage}\\*
\bottomrule
\end{longtable}
{ }



% DecayChannels     code/database/decayChannels.f90

\begin{longtable}{llll}
\toprule
\textbf{\large{DecayChannels}} & \multicolumn{3}{l}{\footnotesize{code/database/decayChannels.f90}}\\*
\midrule
\endfirsthead
\midrule
\endhead
rhoDelta\_is\_sigmaDelta & \begin{minipage}[t]{2cm}logical\end{minipage} & \begin{minipage}[t]{2cm}.false.\end{minipage} & \begin{minipage}[t]{12cm}If true, the rho-Delta decay channel will be replaced by sigma-Delta. For discussion, see e.g. Effenberger PhD, chapter 6.3.2.\end{minipage}\\*
\bottomrule
\end{longtable}
{ }



% deltawidth     code/width/deltaWidth.f90

\begin{longtable}{llll}
\toprule
\textbf{\large{deltawidth}} & \multicolumn{3}{l}{\footnotesize{code/width/deltaWidth.f90}}\\*
\midrule
\endfirsthead
\midrule
\endhead
deltaSwitch & \begin{minipage}[t]{2cm}integer\end{minipage} & \begin{minipage}[t]{2cm}3\end{minipage} & \begin{minipage}[t]{12cm}Switch for different prescriptions for the delta width.\\NOTES\begin{itemize}\leftmargin0em\itemindent0pt\item 1 = use Oset self energies+BUU input\item 2 = Spreading potential\item 3 = use Oset self energies\item 4 = density dependent with BUU input\end{itemize}\end{minipage}\\*
\bottomrule
\end{longtable}
{ }



% detailed_diff     code/analysis/neutrinoAnalysis.f90

\begin{longtable}{llll}
\toprule
\textbf{\large{detailed\_diff}} & \multicolumn{3}{l}{\footnotesize{code/analysis/neutrinoAnalysis.f90}}\\*
\midrule
\endfirsthead
\midrule
\endhead
EkinMin & \begin{minipage}[t]{2cm}real\end{minipage} & \begin{minipage}[t]{2cm}0.\end{minipage} & \begin{minipage}[t]{12cm}if detailed\_diff\_output is TRUE: minimal kinetic energy for dsigma/dEkin for hadrons\end{minipage}\\*
\midrule
EkinMax & \begin{minipage}[t]{2cm}real\end{minipage} & \begin{minipage}[t]{2cm}2.\end{minipage} & \begin{minipage}[t]{12cm}if detailed\_diff\_output is TRUE: Maximal kinetic energy for dsigma/dEkin for hadrons\end{minipage}\\*
\midrule
dEkin & \begin{minipage}[t]{2cm}real\end{minipage} & \begin{minipage}[t]{2cm}0.01\end{minipage} & \begin{minipage}[t]{12cm}if detailed\_diff\_output is TRUE: Delta(eKin) for dsigma/dEKin  for hadrons\end{minipage}\\*
\midrule
forPion & \begin{minipage}[t]{2cm}logical\end{minipage} & \begin{minipage}[t]{2cm}.true.\end{minipage} & \begin{minipage}[t]{12cm}If .true. then also the detailed output of differential cross sections is produced\end{minipage}\\*
\midrule
forEta & \begin{minipage}[t]{2cm}logical\end{minipage} & \begin{minipage}[t]{2cm}.false.\end{minipage} & \begin{minipage}[t]{12cm}If .true. then also the detailed output of differential cross sections is produced\end{minipage}\\*
\midrule
forKaon & \begin{minipage}[t]{2cm}logical\end{minipage} & \begin{minipage}[t]{2cm}.false.\end{minipage} & \begin{minipage}[t]{12cm}If .true. then also the detailed output of differential cross sections is produced\end{minipage}\\*
\midrule
forKaonBar & \begin{minipage}[t]{2cm}logical\end{minipage} & \begin{minipage}[t]{2cm}.false.\end{minipage} & \begin{minipage}[t]{12cm}If .true. then also the detailed output of differential cross sections is produced\end{minipage}\\*
\midrule
forDmeson & \begin{minipage}[t]{2cm}logical\end{minipage} & \begin{minipage}[t]{2cm}.false.\end{minipage} & \begin{minipage}[t]{12cm}If .true. then also the detailed output of differential cross sections is produced\end{minipage}\\*
\midrule
forDbar & \begin{minipage}[t]{2cm}logical\end{minipage} & \begin{minipage}[t]{2cm}.false.\end{minipage} & \begin{minipage}[t]{12cm}If .true. then also the detailed output of differential cross sections is produced\end{minipage}\\*
\midrule
forDs\_plus & \begin{minipage}[t]{2cm}logical\end{minipage} & \begin{minipage}[t]{2cm}.false.\end{minipage} & \begin{minipage}[t]{12cm}If .true. then also the detailed output of differential cross sections is produced\end{minipage}\\*
\midrule
forDs\_minus & \begin{minipage}[t]{2cm}logical\end{minipage} & \begin{minipage}[t]{2cm}.false.\end{minipage} & \begin{minipage}[t]{12cm}If .true. then also the detailed output of differential cross sections is produced\end{minipage}\\*
\midrule
forNucleon & \begin{minipage}[t]{2cm}logical\end{minipage} & \begin{minipage}[t]{2cm}.true.\end{minipage} & \begin{minipage}[t]{12cm}If .true. then also the detailed output of differential cross sections is produced\end{minipage}\\*
\midrule
forLambda & \begin{minipage}[t]{2cm}logical\end{minipage} & \begin{minipage}[t]{2cm}.false.\end{minipage} & \begin{minipage}[t]{12cm}If .true. then also the detailed output of differential cross sections is produced\end{minipage}\\*
\midrule
forSigmaResonance & \begin{minipage}[t]{2cm}logical\end{minipage} & \begin{minipage}[t]{2cm}.false.\end{minipage} & \begin{minipage}[t]{12cm}If .true. then also the detailed output of differential cross sections is produced\end{minipage}\\*
\midrule
forXi & \begin{minipage}[t]{2cm}logical\end{minipage} & \begin{minipage}[t]{2cm}.false.\end{minipage} & \begin{minipage}[t]{12cm}If .true. then also the detailed output of differential cross sections is produced\end{minipage}\\*
\midrule
forOmegaResonance & \begin{minipage}[t]{2cm}logical\end{minipage} & \begin{minipage}[t]{2cm}.false.\end{minipage} & \begin{minipage}[t]{12cm}If .true. then also the detailed output of differential cross sections is produced\end{minipage}\\*
\bottomrule
\end{longtable}
{ }



% deuteriumFermi     code/init/deuterium.f90

\begin{longtable}{llll}
\toprule
\textbf{\large{deuteriumFermi}} & \multicolumn{3}{l}{\footnotesize{code/init/deuterium.f90}}\\*
\midrule
\endfirsthead
\midrule
\endhead
waveFunction\_switch & \begin{minipage}[t]{2cm}integer\end{minipage} & \begin{minipage}[t]{2cm}1\end{minipage} & \begin{minipage}[t]{12cm}Possible values are:\begin{itemize}\leftmargin0em\itemindent0pt\item 0 -- No Wave functions! Pointlike Deuterium\item 1 -- Wave functions according to Bonn potential\item 2 -- Wave functions according to Argonne V18\end{itemize}\end{minipage}\\*
\midrule
iParam & \begin{minipage}[t]{2cm}integer\end{minipage} & \begin{minipage}[t]{2cm}1\end{minipage} & \begin{minipage}[t]{12cm}Choose parameterization of momentum distribution when using the Bonn potential. Possible values:\begin{itemize}\leftmargin0em\itemindent0pt\item 1 -- Full Bonn  (MaH87)\item 2 -- OBEPQ      (MaH87)\item 3 -- OBEPQ-A    (Mac89)\item 4 -- OBEPQ-B    (Mac89)\item 5 -- OBEPQ-!    (Mac89)\item 6 -- OBEPR      (MaH87)  self-made\item 7 -- Paris\end{itemize} References:\begin{itemize}\leftmargin0em\itemindent0pt\item MaH87: R. Machleidt et al. Phys. Rep. 149, 1 (1987)\item Mac89: R. Machleidt, Advances in Nucl. Phys. Vol 19\end{itemize}\end{minipage}\\*
\midrule
pMax & \begin{minipage}[t]{2cm}real\end{minipage} & \begin{minipage}[t]{2cm}0.5\end{minipage} & \begin{minipage}[t]{12cm}Cut-off parameter for Fermi momentum\end{minipage}\\*
\midrule
scaleMomentum & \begin{minipage}[t]{2cm}real\end{minipage} & \begin{minipage}[t]{2cm}1.0\end{minipage} & \begin{minipage}[t]{12cm}The selected momentum is multiplied by this factor afterwards, i.e. some rescaling is done\end{minipage}\\*
\bottomrule
\end{longtable}
{ }



% DileptonAnalysis     code/analysis/DileptonAnalysis.f90

\begin{longtable}{llll}
\toprule
\textbf{\large{DileptonAnalysis}} & \multicolumn{3}{l}{\footnotesize{code/analysis/DileptonAnalysis.f90}}\\*
\midrule
\endfirsthead
\midrule
\endhead
Enable & \begin{minipage}[t]{2cm}logical\end{minipage} & \begin{minipage}[t]{2cm}.false.\end{minipage} & \begin{minipage}[t]{12cm}If .true. the dilepton analysis will be performed, otherwise not.\end{minipage}\\*
\midrule
Extra & \begin{minipage}[t]{2cm}logical\end{minipage} & \begin{minipage}[t]{2cm}.false.\end{minipage} & \begin{minipage}[t]{12cm}If .true. an extended analysis will be performed, writing out many extra histograms (beyond the basic ones: mass, pT and rapidity).\end{minipage}\\*
\midrule
DeltaDalitz & \begin{minipage}[t]{2cm}integer\end{minipage} & \begin{minipage}[t]{2cm}2\end{minipage} & \begin{minipage}[t]{12cm}Choose between different parametrizations of the Delta Dalitz decay width (Delta $\rightarrow$ N e+e-):\begin{itemize}\leftmargin0em\itemindent0pt\item 1 = Wolf, http://inspirehep.net/record/306273\item 2 = Krivoruchenko (default), http://inspirehep.net/record/555421\item 3 = HadronTensor\item 4 = Ernst, http://inspirehep.net/record/452782\end{itemize}\end{minipage}\\*
\midrule
DeltaDalitzFF & \begin{minipage}[t]{2cm}integer\end{minipage} & \begin{minipage}[t]{2cm}1\end{minipage} & \begin{minipage}[t]{12cm}Choose a parametrization of the electromagnetic N-Delta transition form factor for the Delta Dalitz decay (only used for DeltaDalitz = 2):\begin{itemize}\leftmargin0em\itemindent0pt\item 1 = constant (default)\item 2 = Dipole\item 3 = MAID 2005\item 4 = simple VMD\item 5 = Wan/Iachello, Int. J. Mod. Phys. A 20 (2005) 1846,   http://inspirehep.net/record/689265\item 6 = Ramalho/Pena, Phys.Rev. D85 (2012) 113014,   http://inspirehep.net/record/1114321\end{itemize}\end{minipage}\\*
\midrule
omegaDalitzFF & \begin{minipage}[t]{2cm}integer\end{minipage} & \begin{minipage}[t]{2cm}1\end{minipage} & \begin{minipage}[t]{12cm}Choose between different parametrizations of the omega Dalitz decay (omega $\rightarrow$ pi\^{}0 e+e-) form factor:\begin{itemize}\leftmargin0em\itemindent0pt\item 0 = constant\item 1 = Effenberger/Bratkovskaya (default)\item 2 = standard VMD\item 3 = Terschluesen/Leupold\end{itemize}\end{minipage}\\*
\midrule
b\_pi & \begin{minipage}[t]{2cm}real\end{minipage} & \begin{minipage}[t]{2cm}5.5\end{minipage} & \begin{minipage}[t]{12cm}This constant represents the b parameter in the form factor of the pi0 Dalitz decay (in GeV\^{}-2), cf. Effenberger Diss. eq. (2.141). Originally taken from L.G. Landsberg, Phys. Rep. 128, 301 (1985).\end{minipage}\\*
\midrule
lambda\_eta & \begin{minipage}[t]{2cm}real\end{minipage} & \begin{minipage}[t]{2cm}0.716\end{minipage} & \begin{minipage}[t]{12cm}This constant represents the Lambda parameter in the form factor of the eta Dalitz decay in GeV. Values:\begin{itemize}\leftmargin0em\itemindent0pt\item L.G. Landsberg, Phys. Rep. 128, 301 (1985): Lambda = 720 MeV\item HADES pp@2.2, B. Spruck, Diss. (2008):      Lambda = 676 MeV\item NA60, Arnaldi et al, PLB 677 (2009):        Lambda = 716 MeV (default)\item CB/TAPS, Berghäuser et al, PLB 701 (2011):  Lambda = 722 MeV\end{itemize}\end{minipage}\\*
\midrule
etaPrimeDalitzFF & \begin{minipage}[t]{2cm}integer\end{minipage} & \begin{minipage}[t]{2cm}0\end{minipage} & \begin{minipage}[t]{12cm}Choose between different parametrizations of the eta' Dalitz decay (eta' $\rightarrow$ e+e- gamma) form factor:\begin{itemize}\leftmargin0em\itemindent0pt\item 0 = constant (default)\item 1 = eta FF (cf. lambda\_eta)\item 2 = generic VMD\item 3 = Genesis / Lepton-G\item 4 = standard VMD (Terschluesen)\end{itemize}\end{minipage}\\*
\midrule
angDist & \begin{minipage}[t]{2cm}integer\end{minipage} & \begin{minipage}[t]{2cm}1\end{minipage} & \begin{minipage}[t]{12cm}This switch determines the angular distribution of the pseudoscalar Dalitz decays P $\rightarrow$ e+ e- gamma (with P=pi0,eta,etaPrime):\begin{itemize}\leftmargin0em\itemindent0pt\item 0 = isotropic decay\item 1 = anisotropic decay according to 1 + B*cos**2(theta) with B=1\item 2 = the Dalitz decays of pi0 and eta will be done via Pythia.\end{itemize}\end{minipage}\\*
\midrule
brems & \begin{minipage}[t]{2cm}integer\end{minipage} & \begin{minipage}[t]{2cm}1\end{minipage} & \begin{minipage}[t]{12cm}This switch determines how the bremsstrahlung contribution is obtained:\begin{itemize}\leftmargin0em\itemindent0pt\item 0 = none\item 1 = soft-photon approximation (SPA)\item 2 = according to the one-boson-exchange (OBE) model by R. Shyam       (for NN bremstrahlung only, no em. form factors)\item 3 = as 2, but with pion em. form factor (for pn)\item 4 = as 3, but times correction factor (for pn), see arXiv:2009.11702\end{itemize}\end{minipage}\\*
\midrule
nEvent & \begin{minipage}[t]{2cm}integer\end{minipage} & \begin{minipage}[t]{2cm}10\end{minipage} & \begin{minipage}[t]{12cm}Number of events to generate for each dilepton decay (to enhance statistics).\end{minipage}\\*
\midrule
nEvent\_BH & \begin{minipage}[t]{2cm}integer\end{minipage} & \begin{minipage}[t]{2cm}1000\end{minipage} & \begin{minipage}[t]{12cm}Number of events for Bethe-Heitler simulation. BH typically needs a lot more statistics than the other dilepton channels. Therefore nEvent\_BH should be much bigger than nEvent.\end{minipage}\\*
\midrule
kp\_cut & \begin{minipage}[t]{2cm}logical\end{minipage} & \begin{minipage}[t]{2cm}.false.\end{minipage} & \begin{minipage}[t]{12cm}Perform a cut on (k*p) in the dilepton analysis, where k is the photon 4-momentum, and p is the electron or positron 4-momentum. This is useful for suppressing the BH contribution. Cf. "kp\_min".\end{minipage}\\*
\midrule
kp\_min & \begin{minipage}[t]{2cm}real\end{minipage} & \begin{minipage}[t]{2cm}0.01\end{minipage} & \begin{minipage}[t]{12cm}If kp\_cut=.true. a cut on (k*p) is performed. kp\_min determines the position of this cut. Only events with (k*p)$>$kp\_min are taken into account.\end{minipage}\\*
\midrule
binsz & \begin{minipage}[t]{2cm}real\end{minipage} & \begin{minipage}[t]{2cm}0.01\end{minipage} & \begin{minipage}[t]{12cm}This determines the bin size of the dilepton mass spectrum in GeV. Default is 10 MeV.\end{minipage}\\*
\midrule
filter & \begin{minipage}[t]{2cm}integer\end{minipage} & \begin{minipage}[t]{2cm}0\end{minipage} & \begin{minipage}[t]{12cm}If filter is nonzero, a filtering algorithm will be applied to the dilepton pairs, otherwise they will be written to the histograms unfiltered. For details on the filtering parameters see routine 'CS'. Choices:\begin{itemize}\leftmargin0em\itemindent0pt\item  0 = no filter\item  1 = DLS\item  2 (removed)\item  3 = HADES (full acceptance filter, using pair acceptance)\item  4 = HADES (full acceptance filter, using single-particle acceptance)\item  5 = g7/CLAS @ JLab\item  6 = KEK E325 (cuts on rapidity, transverse momentum and opening angle)\item  7 = JPARC E16\item 10 = HADES (simple cuts on polar angle, abs. momentum and opening angle)\item 11 = HADES (as 10, but modified by Jan Otto for AgAg@1.58)\item 12 = HADES (as 10, but modified by Karina Scharmann, pp@1.58)\item 20 = as 10, but with smearing (needs dummy HAFT file)\item 21 = as 11, but with smearing (needs dummy HAFT file)\item 22 = as 12, but with smearing (needs dummy HAFT file)\item 30 = as 10, but with smearing according SmearFile\item 31 = as 11, but with smearing according SmearFile\item 32 = as 12, but with smearing according SmearFile\end{itemize}NOTES\\ For filtering modes 3 and 4, the file containing the acceptance matrices must be specified (cf. hadesFilterFile).\\ The old filter 2 has been renamed to 10\end{minipage}\\*
\midrule
hadesFilterFile & \begin{minipage}[t]{2cm}character(len=1000)\end{minipage} & \begin{minipage}[t]{2cm}""\end{minipage} & \begin{minipage}[t]{12cm}This character string determines the location of the file containing the HADES acceptance matrices (filename with absolute or relative path).\\ For filter=4, some default files are selected, if this input parameter is left empty.\\ possible values:\begin{itemize}\leftmargin0em\itemindent0pt\item if not set, default is '[path\_To\_Input]/hades/XXX' (for filter=4)\item if given, but does not contain '/':   default is '[path\_To\_Input]/hades/[hadesFilterFile]'\item otherwise: filename is absolute, including path\end{itemize} NOTE if you want to use the file 'XXX.dat' in the actual directory, give it as './XXX.dat'\end{minipage}\\*
\midrule
hadesSmearFile & \begin{minipage}[t]{2cm}character(len=1000)\end{minipage} & \begin{minipage}[t]{2cm}""\end{minipage} & \begin{minipage}[t]{12cm}This character string determines the location of the file containing the HADES smearing matrices (filename with absolute or relative path).\\ for filter=30,31,32, a value must be given:\begin{itemize}\leftmargin0em\itemindent0pt\item if given, but does not contain '/':   default is '[path\_To\_Input]/hades/[hadesSmearFile]'\item otherwise: filename is absolute, including path\end{itemize} NOTE if you want to use the file 'XXX.dat' in the actual directory, give it as './XXX.dat'\end{minipage}\\*
\midrule
WriteEvents & \begin{minipage}[t]{2cm}integer\end{minipage} & \begin{minipage}[t]{2cm}0\end{minipage} & \begin{minipage}[t]{12cm}This switch decides whether we write out the simulated events. Possible values:\begin{itemize}\leftmargin0em\itemindent0pt\item 0: Don't write events (default).\item 1: We write out only the lepton pair information (including charge,   four-momentum, perturbative weight, source channel and filter result).   All of this will be written to a file called 'Dilepton\_Events.dat'.\item 2: As 1, but only writing exclusive events (NN$\rightarrow$NNe+e-).\item 3: We write out all produced particles in the event   (including the lepton pair) to a file called 'Dilepton\_FullEvents.dat'.\item 4: As 3, but only writing exclusive events (NN$\rightarrow$NNe+e-).\item 5: As 4, but only writing out R$\rightarrow$Ne+e- events (with R=N*,Delta*).\end{itemize}\end{minipage}\\*
\midrule
p\_lep\_min & \begin{minipage}[t]{2cm}real\end{minipage} & \begin{minipage}[t]{2cm}0.\end{minipage} & \begin{minipage}[t]{12cm}This switch sets a lower bound on the lepton momentum. Only leptons with momenta larger than this threshold will pass the filter. This switch is only used for filtering mode 5 (JLab).\end{minipage}\\*
\midrule
beta\_gamma\_cut & \begin{minipage}[t]{2cm}real\end{minipage} & \begin{minipage}[t]{2cm}1.25\end{minipage} & \begin{minipage}[t]{12cm}This is an upper bound on the beta*gamma value of the lepton pair. Since beta*gamma = p/m, it cuts on slow pairs.\end{minipage}\\*
\midrule
massBinning & \begin{minipage}[t]{2cm}real, dimension(1:4)\end{minipage} & \begin{minipage}[t]{2cm}(/ 0.150, 0.550, 9.999, 9.999 /)\end{minipage} & \begin{minipage}[t]{12cm}We produce several histograms (e.g. p,pT,mT,y,theta\_cm) not only for the full mass range, but also for (up to 5) different mass bins. The borders of these bins are given by this array.\end{minipage}\\*
\midrule
particle\_source & \begin{minipage}[t]{2cm}logical\end{minipage} & \begin{minipage}[t]{2cm}.true.\end{minipage} & \begin{minipage}[t]{12cm}This switch determines whether the mass spectrum will contain separate channels for different sources of particles, such as decays (R$\rightarrow$rho N) or collisions (pi pi $\rightarrow$ rho, K K $\rightarrow$ phi). Currently this is only done for the rho and phi mesons. Note: If using this switch, the "sum" channel in the mass histogram should not be used, since the rho and phi contributions will enter twice.\end{minipage}\\*
\midrule
missMass\_min & \begin{minipage}[t]{2cm}real\end{minipage} & \begin{minipage}[t]{2cm}-99.9\end{minipage} & \begin{minipage}[t]{12cm}if $>$ 0, only events with a missing mass larger than this value are taken into account\end{minipage}\\*
\midrule
missMass\_max & \begin{minipage}[t]{2cm}real\end{minipage} & \begin{minipage}[t]{2cm}-99.9\end{minipage} & \begin{minipage}[t]{12cm}if $>$ 0, only events with a missing mass smaller than this value are taken into account\end{minipage}\\*
\midrule
JanOttoOmega & \begin{minipage}[t]{2cm}logical\end{minipage} & \begin{minipage}[t]{2cm}.false.\end{minipage} & \begin{minipage}[t]{12cm}If .true. an extended analysis will be performed, writing out dilepton spectra focussing on the omega region\end{minipage}\\*
\bottomrule
\end{longtable}
{ }



% elementary     code/init/initElementary.f90

\begin{longtable}{llll}
\toprule
\textbf{\large{elementary}} & \multicolumn{3}{l}{\footnotesize{code/init/initElementary.f90}}\\*
\midrule
\endfirsthead
\midrule
\endhead
impactParameter & \begin{minipage}[t]{2cm}real\end{minipage} & \begin{minipage}[t]{2cm}-1.\end{minipage} & \begin{minipage}[t]{12cm}\begin{itemize}\leftmargin0em\itemindent0pt\item $>$=0: this is the actual impact parameter\item $<$0 : Impact parameter integration up to an automatically determined b\_max.   The actual impact parameter is randomly sampled in the interval   [0.,b\_max] with a proper geometrical weight.\end{itemize}\end{minipage}\\*
\midrule
particleId & \begin{minipage}[t]{2cm}integer, dimension(2)\end{minipage} & \begin{minipage}[t]{2cm}(/1,1/)\end{minipage} & \begin{minipage}[t]{12cm}Id of particles\end{minipage}\\*
\midrule
particleAnti & \begin{minipage}[t]{2cm}logical, dimension(2)\end{minipage} & \begin{minipage}[t]{2cm}(/.false.,.false./)\end{minipage} & \begin{minipage}[t]{12cm}if .true. then particles are antiparticles\end{minipage}\\*
\midrule
particleCharge & \begin{minipage}[t]{2cm}integer, dimension(2)\end{minipage} & \begin{minipage}[t]{2cm}(/0,0/)\end{minipage} & \begin{minipage}[t]{12cm}Charge of the particles\end{minipage}\\*
\midrule
srtsRaiseFlag & \begin{minipage}[t]{2cm}logical\end{minipage} & \begin{minipage}[t]{2cm}.false.\end{minipage} & \begin{minipage}[t]{12cm}\begin{itemize}\leftmargin0em\itemindent0pt\item if .true. then the srts stepping is done\item if .false. then the ekin\_lab stepping is done\end{itemize}\end{minipage}\\*
\midrule
ekin\_lab & \begin{minipage}[t]{2cm}real\end{minipage} & \begin{minipage}[t]{2cm}1.\end{minipage} & \begin{minipage}[t]{12cm}kin. energy of first particle in the rest frame of second particle (starting value for the energy scan: the number of different energies is set by parameter num\_Energies in the namelist "input")\end{minipage}\\*
\midrule
delta\_ekin\_lab & \begin{minipage}[t]{2cm}real\end{minipage} & \begin{minipage}[t]{2cm}0.03\end{minipage} & \begin{minipage}[t]{12cm}kin. energy step for energy scan\end{minipage}\\*
\midrule
srts & \begin{minipage}[t]{2cm}real\end{minipage} & \begin{minipage}[t]{2cm}3.\end{minipage} & \begin{minipage}[t]{12cm}invariant energy (starting value for the energy scan)\end{minipage}\\*
\midrule
delta\_srts & \begin{minipage}[t]{2cm}real\end{minipage} & \begin{minipage}[t]{2cm}1.\end{minipage} & \begin{minipage}[t]{12cm}srts step for srts scan\end{minipage}\\*
\bottomrule
\end{longtable}
{ }



% Elementary_Analysis     code/analysis/ElementaryAnalysis.f90

\begin{longtable}{llll}
\toprule
\textbf{\large{Elementary\_Analysis}} & \multicolumn{3}{l}{\footnotesize{code/analysis/ElementaryAnalysis.f90}}\\*
\midrule
\endfirsthead
\midrule
\endhead
DoOutChannels & \begin{minipage}[t]{2cm}logical\end{minipage} & \begin{minipage}[t]{2cm}.false.\end{minipage} & \begin{minipage}[t]{12cm}switch on/off: reporting of all final state channels\end{minipage}\\*
\midrule
DoH2d & \begin{minipage}[t]{2cm}logical\end{minipage} & \begin{minipage}[t]{2cm}.false.\end{minipage} & \begin{minipage}[t]{12cm}if .true. than make output of 2-dimensional histograms (they could produce files of size 240 mb)\end{minipage}\\*
\midrule
Do45ForAllEvents & \begin{minipage}[t]{2cm}logical\end{minipage} & \begin{minipage}[t]{2cm}.false.\end{minipage} & \begin{minipage}[t]{12cm}flag to decide, whether DoElementaryAnalysis4(5).dat is written for all events or just for events, where the output channel consist of pions\end{minipage}\\*
\midrule
DodNNbar & \begin{minipage}[t]{2cm}logical\end{minipage} & \begin{minipage}[t]{2cm}.false.\end{minipage} & \begin{minipage}[t]{12cm}\end{minipage}\\*
\midrule
DoPanda & \begin{minipage}[t]{2cm}logical\end{minipage} & \begin{minipage}[t]{2cm}.false.\end{minipage} & \begin{minipage}[t]{12cm}if .true., elementary analysis for channels with S = -2 and -3 (Xi, Omega)\end{minipage}\\*
\midrule
Dodsigdt & \begin{minipage}[t]{2cm}logical\end{minipage} & \begin{minipage}[t]{2cm}.false.\end{minipage} & \begin{minipage}[t]{12cm}\end{minipage}\\*
\midrule
Do2Part & \begin{minipage}[t]{2cm}logical\end{minipage} & \begin{minipage}[t]{2cm}.false.\end{minipage} & \begin{minipage}[t]{12cm}\end{minipage}\\*
\bottomrule
\end{longtable}
{ }



% eN_Event     code/init/ElectronGenerator/eN_event.f90

\begin{longtable}{llll}
\toprule
\textbf{\large{eN\_Event}} & \multicolumn{3}{l}{\footnotesize{code/init/ElectronGenerator/eN\_event.f90}}\\*
\midrule
\endfirsthead
\midrule
\endhead
selectFrame & \begin{minipage}[t]{2cm}integer\end{minipage} & \begin{minipage}[t]{2cm}2\end{minipage} & \begin{minipage}[t]{12cm}select frame, in which the calculaton of W\_free is done:\begin{itemize}\leftmargin0em\itemindent0pt\item 0 = doNOT   ---  do NOT remove\item 1 = CM\item 2 = CALC\item 3 = THRE  prescription from correct threshold behaviour, used in heavy ion collisions\item 4 = NucleonRest :  boost nucleon in the rest frame, set free mN, recalculate boson momentum\item 5 = THRE2 threshold with m\^{}2: sfree=s+m\^{}2-m*\^{}2\end{itemize}\end{minipage}\\*
\midrule
restingNucleon & \begin{minipage}[t]{2cm}logical\end{minipage} & \begin{minipage}[t]{2cm}.true.\end{minipage} & \begin{minipage}[t]{12cm}if this flag is .false., we use the momentum of the target nucleon in the calculation of the flux\end{minipage}\\*
\bottomrule
\end{longtable}
{ }



% EventOutput     code/analysis/EventOutputAnalysis.f90

\begin{longtable}{llll}
\toprule
\textbf{\large{EventOutput}} & \multicolumn{3}{l}{\footnotesize{code/analysis/EventOutputAnalysis.f90}}\\*
\midrule
\endfirsthead
\midrule
\endhead
WritePerturbativeParticles & \begin{minipage}[t]{2cm}logical\end{minipage} & \begin{minipage}[t]{2cm}.false.\end{minipage} & \begin{minipage}[t]{12cm}Flag to write out the perturbative particle vector to an output file. The switch 'EventFormat' determines which format is used.\end{minipage}\\*
\midrule
WriteRealParticles & \begin{minipage}[t]{2cm}logical\end{minipage} & \begin{minipage}[t]{2cm}.false.\end{minipage} & \begin{minipage}[t]{12cm}Flag to write out the real particle vector to an output file. The switch 'EventFormat' determines which format is used.\end{minipage}\\*
\midrule
EventFormat & \begin{minipage}[t]{2cm}integer\end{minipage} & \begin{minipage}[t]{2cm}1\end{minipage} & \begin{minipage}[t]{12cm}This switch determines the format of the event output files. Possible values:\begin{itemize}\leftmargin0em\itemindent0pt\item 1 = Les Houches format (default)\item 2 = OSCAR 2013 format\item 3 = Shanghai 2014 format\item 4 = ROOT\item 5 = OSCAR 2013 extended format\end{itemize}NOTES\begin{itemize}\leftmargin0em\itemindent0pt\item For Les Houches, the output will be written to files called   EventOutput.Pert.lhe and EventOutput.Real.lhe.\item For OSCAR, the output files are called EventOutput.Pert.oscar and   EventOutput.Real.oscar.\item For Shanghai, the output files are called EventOutput.Pert.dat and   EventOutput.Real.dat.\item For ROOT, the output files are called EventOutput.Pert.root and   EventOutput.Real.root.\end{itemize}\end{minipage}\\*
\midrule
Interval & \begin{minipage}[t]{2cm}integer\end{minipage} & \begin{minipage}[t]{2cm}0\end{minipage} & \begin{minipage}[t]{12cm}Interval for event output, i.e. number of timesteps after which output is written. If zero, only final output at the end of the time evolution is produced.\end{minipage}\\*
\bottomrule
\end{longtable}
{ }



% externalSystem     code/init/initExternal.f90

\begin{longtable}{llll}
\toprule
\textbf{\large{externalSystem}} & \multicolumn{3}{l}{\footnotesize{code/init/initExternal.f90}}\\*
\midrule
\endfirsthead
\midrule
\endhead
inputFile & \begin{minipage}[t]{2cm}character*1000\end{minipage} & \begin{minipage}[t]{2cm}'./source.inp'\end{minipage} & \begin{minipage}[t]{12cm}the absolute name of the input file with hadrons to be propagated.\\ possible values:\begin{itemize}\leftmargin0em\itemindent0pt\item if not set, default is './source.inp'\item if given, but does not contain '/':   default is './[inputFile]'\item otherwise: filename is absolute, including path\end{itemize} NOTE if you want to use the file 'XXX.inp' in the actual directory, give it as './XXX.inp'\end{minipage}\\*
\midrule
DoPerturbative & \begin{minipage}[t]{2cm}logical\end{minipage} & \begin{minipage}[t]{2cm}.false.\end{minipage} & \begin{minipage}[t]{12cm}if true, the particles will be inserted into the perturbative particle vector, the real particles have to be initialized via some nucleus definition\end{minipage}\\*
\midrule
NumberingScheme & \begin{minipage}[t]{2cm}integer\end{minipage} & \begin{minipage}[t]{2cm}1\end{minipage} & \begin{minipage}[t]{12cm}The way, how particles\%event will be numbered:\begin{itemize}\leftmargin0em\itemindent0pt\item 1: event = iPart, i.e. the particle number in the ensemble   (historical, but does not work for fullensemble)\item 2: event = pert\_numbering() or real\_numbering()   (good both for perturbative and real mode)\end{itemize}\end{minipage}\\*
\midrule
posSRC & \begin{minipage}[t]{2cm}logical\end{minipage} & \begin{minipage}[t]{2cm}.false.\end{minipage} & \begin{minipage}[t]{12cm}If true, the position vectors of the proton and neutron from SRC will be sampled by Monte-Carlo. Relevant when the target nucleus was initialized before calling initializeExternal and if there are only proton and neutron in the external source.\end{minipage}\\*
\midrule
flagPH & \begin{minipage}[t]{2cm}logical\end{minipage} & \begin{minipage}[t]{2cm}.false.\end{minipage} & \begin{minipage}[t]{12cm}If true, a particle-hole excitation will be added to the target nucleus. The momentum of the particle is obtained by adding transverse momentum transfer along x-axis and from "-" momentum conservation.\end{minipage}\\*
\midrule
pt & \begin{minipage}[t]{2cm}real\end{minipage} & \begin{minipage}[t]{2cm}0.\end{minipage} & \begin{minipage}[t]{12cm}Transverse momentum transfer to the struck nucleon (GeV/c). Relevant when flagPH=.true.\end{minipage}\\*
\bottomrule
\end{longtable}
{ }



% ff_QE     code/init/lepton/formfactors_QE_nucleon/FF_QE_nucleonScattering.f90

\begin{longtable}{llll}
\toprule
\textbf{\large{ff\_QE}} & \multicolumn{3}{l}{\footnotesize{code/init/lepton/formfactors\_QE\_nucleon/FF\_QE\_nucleonScattering.f90}}\\*
\midrule
\endfirsthead
\midrule
\endhead
parametrization & \begin{minipage}[t]{2cm}integer\end{minipage} & \begin{minipage}[t]{2cm}3\end{minipage} & \begin{minipage}[t]{12cm}\begin{itemize}\leftmargin0em\itemindent0pt\item 0 = dipole approximation\item 1 = BBA03 parametrization\item 2 = BBBA05 parametrization\item 3 = BBBA07 parametrization\end{itemize}\end{minipage}\\*
\midrule
MV2 & \begin{minipage}[t]{2cm}real\end{minipage} & \begin{minipage}[t]{2cm}0.71\end{minipage} & \begin{minipage}[t]{12cm}vector mass squared in the dipole parametrization of the vector form factors\end{minipage}\\*
\midrule
MA\_in & \begin{minipage}[t]{2cm}real\end{minipage} & \begin{minipage}[t]{2cm}1.0\end{minipage} & \begin{minipage}[t]{12cm}axial mass (only if useNonStandardMA=.true.)\end{minipage}\\*
\midrule
useNonStandardMA & \begin{minipage}[t]{2cm}logical\end{minipage} & \begin{minipage}[t]{2cm}.false.\end{minipage} & \begin{minipage}[t]{12cm}if one wants to use a specific axial mass, set this to true and choose value for MA\_in\end{minipage}\\*
\midrule
deltas & \begin{minipage}[t]{2cm}real\end{minipage} & \begin{minipage}[t]{2cm}-0.15\end{minipage} & \begin{minipage}[t]{12cm}strange contribution to the axial ff.\end{minipage}\\*
\midrule
axialMonopole & \begin{minipage}[t]{2cm}logical\end{minipage} & \begin{minipage}[t]{2cm}.false.\end{minipage} & \begin{minipage}[t]{12cm}use axial ff. of Gari, Kaulfuss PLB 138 (1984)\end{minipage}\\*
\bottomrule
\end{longtable}
{ }



% FinalState_Full     code/collisions/phaseSpace/finalState_Full.f90

\begin{longtable}{llll}
\toprule
\textbf{\large{FinalState\_Full}} & \multicolumn{3}{l}{\footnotesize{code/collisions/phaseSpace/finalState\_Full.f90}}\\*
\midrule
\endfirsthead
\midrule
\endhead
maxbwd\_scalingFactor & \begin{minipage}[t]{2cm}real\end{minipage} & \begin{minipage}[t]{2cm}1.\end{minipage} & \begin{minipage}[t]{12cm}\begin{itemize}\leftmargin0em\itemindent0pt\item Rescales maxBWD\end{itemize}\end{minipage}\\*
\midrule
silentMode & \begin{minipage}[t]{2cm}logical\end{minipage} & \begin{minipage}[t]{2cm}.true.\end{minipage} & \begin{minipage}[t]{12cm}\begin{itemize}\leftmargin0em\itemindent0pt\item Switches error messages off in massAss. Errors can still be seen   looking at massAssStatus.dat\end{itemize}\end{minipage}\\*
\midrule
NYK\_isotropic & \begin{minipage}[t]{2cm}logical\end{minipage} & \begin{minipage}[t]{2cm}.false.\end{minipage} & \begin{minipage}[t]{12cm}If .true., the angular distribution in Nucleon-Hyperon-Kaon production is assumed to be isotropic. If .false., a non-isotropic distribution is used, as described in Larionov/Mosel, Phys.Rev. C 72 (2005) 014901. See also momenta\_in\_3Body\_BYK.\end{minipage}\\*
\bottomrule
\end{longtable}
{ }



% formfactors_pion     code/init/lepton/formfactors_pionProduction/formfactors_A_input.f90

\begin{longtable}{llll}
\toprule
\textbf{\large{formfactors\_pion}} & \multicolumn{3}{l}{\footnotesize{code/init/lepton/formfactors\_pionProduction/formfactors\_A\_input.f90}}\\*
\midrule
\endfirsthead
\midrule
\endhead
which\_MaidVersion & \begin{minipage}[t]{2cm}integer\end{minipage} & \begin{minipage}[t]{2cm}2\end{minipage} & \begin{minipage}[t]{12cm}choice of MAID version:  1=2003, 2=2007\end{minipage}\\*
\bottomrule
\end{longtable}
{ }



% Freezeout     code/analysis/FreezeoutAnalysis.f90

\begin{longtable}{llll}
\toprule
\textbf{\large{Freezeout}} & \multicolumn{3}{l}{\footnotesize{code/analysis/FreezeoutAnalysis.f90}}\\*
\midrule
\endfirsthead
\midrule
\endhead
FreezeoutAnalysis\_Pert & \begin{minipage}[t]{2cm}logical\end{minipage} & \begin{minipage}[t]{2cm}.false.\end{minipage} & \begin{minipage}[t]{12cm}Flag to do freeze out analysis for perturbative particles\end{minipage}\\*
\midrule
FreezeoutAnalysis\_Real & \begin{minipage}[t]{2cm}logical\end{minipage} & \begin{minipage}[t]{2cm}.false.\end{minipage} & \begin{minipage}[t]{12cm}Flag to do freeze out analysis for real particles\end{minipage}\\*
\midrule
potThreshold & \begin{minipage}[t]{2cm}real\end{minipage} & \begin{minipage}[t]{2cm}0.005\end{minipage} & \begin{minipage}[t]{12cm}threshold value in GeV. If the absolute value of the potential is below this value, the particle is considered to be 'free', e.g. it 'escaped'\end{minipage}\\*
\bottomrule
\end{longtable}
{ }



% gamma_2Pi_Xsections     code/init/lowPhoton/twoPi_production/gamma2Pi_Xsections.f90

\begin{longtable}{llll}
\toprule
\textbf{\large{gamma\_2Pi\_Xsections}} & \multicolumn{3}{l}{\footnotesize{code/init/lowPhoton/twoPi\_production/gamma2Pi\_Xsections.f90}}\\*
\midrule
\endfirsthead
\midrule
\endhead
experimentalXsections & \begin{minipage}[t]{2cm}logical\end{minipage} & \begin{minipage}[t]{2cm}.true.\end{minipage} & \begin{minipage}[t]{12cm}\begin{itemize}\leftmargin0em\itemindent0pt\item If .true. then the Xsections are taken from the experiment\item If .false. then the theoretical values are given\end{itemize}\end{minipage}\\*
\bottomrule
\end{longtable}
{ }



% hadron     code/init/initHadron.f90

\begin{longtable}{llll}
\toprule
\textbf{\large{hadron}} & \multicolumn{3}{l}{\footnotesize{code/init/initHadron.f90}}\\*
\midrule
\endfirsthead
\midrule
\endhead
impactParameter & \begin{minipage}[t]{2cm}real\end{minipage} & \begin{minipage}[t]{2cm}0.\end{minipage} & \begin{minipage}[t]{12cm}smaller 0: Impact parameter will be chosen randomly in the interval [0;abs(impactParameter)] (see subroutine setGeometry). It is recommended to take very large negative value of impactParameter in order to have good automatic random choice, e.g. impactParameter=-100.\end{minipage}\\*
\midrule
bRaiseFlag & \begin{minipage}[t]{2cm}logical\end{minipage} & \begin{minipage}[t]{2cm}.false.\end{minipage} & \begin{minipage}[t]{12cm}if .true.: actual impact parameter will be raised by deltaB after nRunsPerB subsequent runs. Starting value is given by the impactParameter variable.\end{minipage}\\*
\midrule
deltaB & \begin{minipage}[t]{2cm}real\end{minipage} & \begin{minipage}[t]{2cm}0.\end{minipage} & \begin{minipage}[t]{12cm}impact parameter step (relevant if bRaiseFlag=.true.)\end{minipage}\\*
\midrule
nRunsPerB & \begin{minipage}[t]{2cm}integer\end{minipage} & \begin{minipage}[t]{2cm}1\end{minipage} & \begin{minipage}[t]{12cm}number of subsequent runs per impact parameter (relevant if bRaiseFlag=.true.)\end{minipage}\\*
\midrule
perturbative & \begin{minipage}[t]{2cm}logical\end{minipage} & \begin{minipage}[t]{2cm}.false.\end{minipage} & \begin{minipage}[t]{12cm}if .true. then the hadron is a perturbative particle\end{minipage}\\*
\midrule
numberParticles & \begin{minipage}[t]{2cm}integer\end{minipage} & \begin{minipage}[t]{2cm}200\end{minipage} & \begin{minipage}[t]{12cm}Number of projectile testparticles per ensemble in the case of a perturbative treatment\end{minipage}\\*
\midrule
particleId & \begin{minipage}[t]{2cm}integer\end{minipage} & \begin{minipage}[t]{2cm}1\end{minipage} & \begin{minipage}[t]{12cm}Identity of the projectile hadron.\end{minipage}\\*
\midrule
antiParticle & \begin{minipage}[t]{2cm}logical\end{minipage} & \begin{minipage}[t]{2cm}.false.\end{minipage} & \begin{minipage}[t]{12cm}if .true. then the hadron is an antiparticle\end{minipage}\\*
\midrule
particleCharge & \begin{minipage}[t]{2cm}integer\end{minipage} & \begin{minipage}[t]{2cm}0\end{minipage} & \begin{minipage}[t]{12cm}Charge of the hadron\end{minipage}\\*
\midrule
ekin\_lab & \begin{minipage}[t]{2cm}real\end{minipage} & \begin{minipage}[t]{2cm}1.\end{minipage} & \begin{minipage}[t]{12cm}kinetic energy of the hadron in the rest frame of the target nucleus (GeV)\\NOTES\\ If ekin\_lab $<$ 0.  --- initialization according to the binding energy\end{minipage}\\*
\midrule
E\_bind & \begin{minipage}[t]{2cm}real\end{minipage} & \begin{minipage}[t]{2cm}0.\end{minipage} & \begin{minipage}[t]{12cm}binding energy of initialized hadron (GeV)\\NOTES\\ Active for iniType= 0,2 if  ekin\_lab $<$ 0. is set.\end{minipage}\\*
\midrule
iniType & \begin{minipage}[t]{2cm}integer\end{minipage} & \begin{minipage}[t]{2cm}0\end{minipage} & \begin{minipage}[t]{12cm}\begin{itemize}\leftmargin0em\itemindent0pt\item 0:   usual initialization for the hadron-nucleus collision\item 1:   position and momentum of the hadron is chosen according to        the Gaussians centered, resp., at the centre of the nucleus and at zero momentum        (impactParameter, distance and ekin\_lab have no effect in this case)\item 2:   gaussian in coordinate space, but usual sharp momentum choice        (impactParameter, distance and ekin\_lab work as usual)\end{itemize}\end{minipage}\\*
\midrule
zChoice & \begin{minipage}[t]{2cm}integer\end{minipage} & \begin{minipage}[t]{2cm}1\end{minipage} & \begin{minipage}[t]{12cm}\begin{itemize}\leftmargin0em\itemindent0pt\item 1:    hadron is initialised at fixed distance delta from nuclear surface\item 2:    hadron is initialised at fixed z\end{itemize} Relevant for iniType=0 or iniType=2.\end{minipage}\\*
\midrule
delta & \begin{minipage}[t]{2cm}real\end{minipage} & \begin{minipage}[t]{2cm}0.5\end{minipage} & \begin{minipage}[t]{12cm}\begin{itemize}\leftmargin0em\itemindent0pt\item for zChoice=1: distance from nuclear surface, at which the hadron is   initialised [fm]\item for zChoice=2: maximum distance from the edge of nucleus in transverse   direction which restricts the choice of actual impact parameter for   impactParameter $<$ 0 (for impactParameter $>$ 0 no restriction)\end{itemize} Relevant for iniType=0 or  iniType=2.\end{minipage}\\*
\midrule
deltaZ & \begin{minipage}[t]{2cm}real\end{minipage} & \begin{minipage}[t]{2cm}5.\end{minipage} & \begin{minipage}[t]{12cm}z = -deltaZ - R\_nucleus, where z is z-coordinate of the hadron Relevant for iniType=0,2 and zChoice=2.\end{minipage}\\*
\midrule
width & \begin{minipage}[t]{2cm}real\end{minipage} & \begin{minipage}[t]{2cm}1.\end{minipage} & \begin{minipage}[t]{12cm}Width of a gaussian density profile [fm]. Only relevant for iniType= 1 and 2.\end{minipage}\\*
\bottomrule
\end{longtable}
{ }



% HadronAnalysis     code/analysis/hadronAnalysis.f90

\begin{longtable}{llll}
\toprule
\textbf{\large{HadronAnalysis}} & \multicolumn{3}{l}{\footnotesize{code/analysis/hadronAnalysis.f90}}\\*
\midrule
\endfirsthead
\midrule
\endhead
flagAnalysis & \begin{minipage}[t]{2cm}logical\end{minipage} & \begin{minipage}[t]{2cm}.false.\end{minipage} & \begin{minipage}[t]{12cm}If true, perform the output of a hadron at the latest time step before the hadron disappeared (file DoHadronAnalysisTime.dat). The output hadron has the same baryon/meson type and antiparticle-flag as the beam particle. In case if the hadron did not disappear, the output is done at the end of the time evolution. The output for the hadron is also done in three other files if its momentum becomes for the first time less than the cut values pCut1 and pCut2 (files DoHadronAnalysisTime1.dat and  DoHadronAnalysisTime2.dat) and if it becomes bound (DoHadronAnalysisTime3.dat)\\NOTES\\ Presently feasible only for real particle simulations.\end{minipage}\\*
\bottomrule
\end{longtable}
{ }



% hadronformation     code/collisions/twoBodyReactions/hadronFormation.f90

\begin{longtable}{llll}
\toprule
\textbf{\large{hadronformation}} & \multicolumn{3}{l}{\footnotesize{code/collisions/twoBodyReactions/hadronFormation.f90}}\\*
\midrule
\endfirsthead
\midrule
\endhead
tauProda & \begin{minipage}[t]{2cm}real\end{minipage} & \begin{minipage}[t]{2cm}0.5\end{minipage} & \begin{minipage}[t]{12cm}in formation time concept 2) and 3) for "error particles": production time of non-leading in rest frame of hadron (in fm)\end{minipage}\\*
\midrule
tauForma & \begin{minipage}[t]{2cm}real\end{minipage} & \begin{minipage}[t]{2cm}0.8\end{minipage} & \begin{minipage}[t]{12cm}in formation time concept 1) and in concept 2),3) for "error particles": formation time in rest frame of hadron (in fm)\end{minipage}\\*
\midrule
tauFormaFak & \begin{minipage}[t]{2cm}real\end{minipage} & \begin{minipage}[t]{2cm}1.0\end{minipage} & \begin{minipage}[t]{12cm}in formation time concept 1): scale factor for constituent quark model, rescales \#(lead quarks)/\#quarks\end{minipage}\\*
\midrule
useJetSetVec & \begin{minipage}[t]{2cm}logical\end{minipage} & \begin{minipage}[t]{2cm}.true.\end{minipage} & \begin{minipage}[t]{12cm}Flag to select fragmentation time estimates:\begin{itemize}\leftmargin0em\itemindent0pt\item false $\rightarrow$ old concept 1)\item true $\rightarrow$ new concepts 2) and 3)\end{itemize}NOTES\\ select false in case of calculations on a nucleon (speed!).\end{minipage}\\*
\midrule
powerCS & \begin{minipage}[t]{2cm}real\end{minipage} & \begin{minipage}[t]{2cm}1.0\end{minipage} & \begin{minipage}[t]{12cm}in formation time concept 2): power of 't' (constant, linear, quadratic)\end{minipage}\\*
\midrule
useTimeFrom & \begin{minipage}[t]{2cm}integer\end{minipage} & \begin{minipage}[t]{2cm}1\end{minipage} & \begin{minipage}[t]{12cm}in formation time concept 2): encode time XS starts to evolve: 1: tP\_min, 2: tP\_max, 3: tF\end{minipage}\\*
\midrule
useTimeTo & \begin{minipage}[t]{2cm}integer\end{minipage} & \begin{minipage}[t]{2cm}3\end{minipage} & \begin{minipage}[t]{12cm}in formation time concept 2): encode time XS stops to evolve: 1: tP\_min, 2: tP\_max, 3: tF\end{minipage}\\*
\midrule
GuessDiffrTimes & \begin{minipage}[t]{2cm}logical\end{minipage} & \begin{minipage}[t]{2cm}.true.\end{minipage} & \begin{minipage}[t]{12cm}if true, then the times for diffractive particles are treated like them of all other particles, otherwise particles from "diffractive" events hadronize immediately.\end{minipage}\\*
\midrule
useJetSetVec\_Q & \begin{minipage}[t]{2cm}logical\end{minipage} & \begin{minipage}[t]{2cm}.true.\end{minipage} & \begin{minipage}[t]{12cm}if useJetSetVec, then also use Q2 as measure for XS-pedestal, i.e. select concept 3) instead of concept 2)\end{minipage}\\*
\midrule
useJetSetVec\_R & \begin{minipage}[t]{2cm}logical\end{minipage} & \begin{minipage}[t]{2cm}.true.\end{minipage} & \begin{minipage}[t]{12cm}if not useJetSetVec\_Q, then use rLead as measure for XS-pedestal\end{minipage}\\*
\midrule
pedestalCS & \begin{minipage}[t]{2cm}real\end{minipage} & \begin{minipage}[t]{2cm}0.0\end{minipage} & \begin{minipage}[t]{12cm}in formation time concept 2): encode time XS stops to evolve: 1: tP\_min, 2: tP\_max, 3: tF\end{minipage}\\*
\midrule
useQDM & \begin{minipage}[t]{2cm}logical\end{minipage} & \begin{minipage}[t]{2cm}.false.\end{minipage} & \begin{minipage}[t]{12cm}If true, then use the quantum diffusion model of G.R. Farrar et al., PRL 61, 686 (1988).\\ It means that the cross section grows as (t-t\_int)**powerCS for t\_int $<$  t $<$ t\_form, where t\_int is the interaction time (=0 for electron-nucleus case) and t\_form= t\_int + 2*p/dM2. So hadrons with equal momenta have equal formation times (lengths). Also allows to control the space-time scale of hadronization. Attention: setting useQDM = .true. overrides other switches of this module.\end{minipage}\\*
\midrule
dM2 & \begin{minipage}[t]{2cm}real\end{minipage} & \begin{minipage}[t]{2cm}0.7\end{minipage} & \begin{minipage}[t]{12cm}Mass denominator in the coherence length.\\ Relevant only for quantum diffusion model (when useQDM =.true.)\end{minipage}\\*
\midrule
use\_pCut & \begin{minipage}[t]{2cm}logical\end{minipage} & \begin{minipage}[t]{2cm}.false.\end{minipage} & \begin{minipage}[t]{12cm}If true, then only particles with momentum p $<$ pCut will interact.\end{minipage}\\*
\midrule
pCut & \begin{minipage}[t]{2cm}real\end{minipage} & \begin{minipage}[t]{2cm}1.\end{minipage} & \begin{minipage}[t]{12cm}Momentum cutoff. Relevant only when use\_pCut =.true.\end{minipage}\\*
\bottomrule
\end{longtable}
{ }



% HadronTensor_ResProd     code/init/lepton/hadronTensor_ResProd.f90

\begin{longtable}{llll}
\toprule
\textbf{\large{HadronTensor\_ResProd}} & \multicolumn{3}{l}{\footnotesize{code/init/lepton/hadronTensor\_ResProd.f90}}\\*
\midrule
\endfirsthead
\midrule
\endhead
speedup & \begin{minipage}[t]{2cm}logical\end{minipage} & \begin{minipage}[t]{2cm}.true.\end{minipage} & \begin{minipage}[t]{12cm}\end{minipage}\\*
\bottomrule
\end{longtable}
{ }



% heavyIon     code/init/initHeavyIon.f90

\begin{longtable}{llll}
\toprule
\textbf{\large{heavyIon}} & \multicolumn{3}{l}{\footnotesize{code/init/initHeavyIon.f90}}\\*
\midrule
\endfirsthead
\midrule
\endhead
impact\_parameter & \begin{minipage}[t]{2cm}real\end{minipage} & \begin{minipage}[t]{2cm}0.\end{minipage} & \begin{minipage}[t]{12cm}Impact parameter b [fm]. There are three options:\begin{itemize}\leftmargin0em\itemindent0pt\item b$>$=0: The impact parameter is fixed to the given value.\item -100$<$b$<$0: The impact parameter will be chosen randomly in each run   between 0 and abs(b).\item b$<$=-100: "Minimum bias". The impact parameter will be chosen randomly   in each run (maximum = sum of radii plus twice the sum of surfaces).\end{itemize}\end{minipage}\\*
\midrule
impact\_profile & \begin{minipage}[t]{2cm}integer\end{minipage} & \begin{minipage}[t]{2cm}0\end{minipage} & \begin{minipage}[t]{12cm}This switch provides impact-parameter distributions for trigger-biased setups. Only used for impact\_parameter $<$ 0. Possible values:\begin{itemize}\leftmargin0em\itemindent0pt\item  0: minimum bias (default)\item  1: HADES C+C    at 1.00 AGeV\item  2: HADES C+C    at 2.00 AGeV\item  3: HADES Ar+KCl at 1.76 AGeV\item  4: HADES Au+Au  at 1.23 AGeV (all)\item  5: HADES Au+Au  at 1.23 AGeV ( 0-10\% central)\item  6: HADES Au+Au  at 1.23 AGeV (10-20\% central)\item  7: HADES Au+Au  at 1.23 AGeV (20-30\% central)\item  8: HADES Au+Au  at 1.23 AGeV (30-40\% central)\item  9: HADES Ag+Ag  at 1.58 AGeV ( 0-40\% central) [not in release]\item 10: HADES Ag+Ag  at 1.58 AGeV ( 0-10\% central) [not in release]\item 11: HADES Ag+Ag  at 1.58 AGeV (10-20\% central) [not in release]\item 12: HADES Ag+Ag  at 1.58 AGeV (20-30\% central) [not in release]\item 13: HADES Ag+Ag  at 1.58 AGeV (30-40\% central) [not in release]\end{itemize}\end{minipage}\\*
\midrule
distance & \begin{minipage}[t]{2cm}real\end{minipage} & \begin{minipage}[t]{2cm}0.\end{minipage} & \begin{minipage}[t]{12cm}Distance between centers of nuclei along z (i.e. beam)-direction [fm]. This will be readjusted automatically in case it is too small.\end{minipage}\\*
\midrule
coulomb & \begin{minipage}[t]{2cm}logical\end{minipage} & \begin{minipage}[t]{2cm}.false.\end{minipage} & \begin{minipage}[t]{12cm}If .true., then a Coulomb propagation from coulombDistance = 10000 fm to distance is performed.\end{minipage}\\*
\midrule
ekin\_lab\_Target & \begin{minipage}[t]{2cm}real\end{minipage} & \begin{minipage}[t]{2cm}0.\end{minipage} & \begin{minipage}[t]{12cm}Kinetic energy per nucleon of target nucleus in lab frame [GeV].\end{minipage}\\*
\midrule
ekin\_lab\_Projectile & \begin{minipage}[t]{2cm}real\end{minipage} & \begin{minipage}[t]{2cm}0.\end{minipage} & \begin{minipage}[t]{12cm}Kinetic energy per nucleon of projectile nucleus in lab frame [GeV].\end{minipage}\\*
\midrule
adjustGridFlag & \begin{minipage}[t]{2cm}logical\end{minipage} & \begin{minipage}[t]{2cm}.false.\end{minipage} & \begin{minipage}[t]{12cm}If .true., the grid spacing in z-direction will be readjusted.\end{minipage}\\*
\midrule
cmsFlag & \begin{minipage}[t]{2cm}logical\end{minipage} & \begin{minipage}[t]{2cm}.true.\end{minipage} & \begin{minipage}[t]{12cm}If .true.,  the collision takes place in the CM frame of the two nuclei (default option). If .false., the collision takes place in the LAB frame (target at rest).\end{minipage}\\*
\bottomrule
\end{longtable}
{ }



% HICanalysis_Input     code/analysis/HeavyIonAnalysis.f90

\begin{longtable}{llll}
\toprule
\textbf{\large{HICanalysis\_Input}} & \multicolumn{3}{l}{\footnotesize{code/analysis/HeavyIonAnalysis.f90}}\\*
\midrule
\endfirsthead
\midrule
\endhead
flag\_outputReal & \begin{minipage}[t]{2cm}logical\end{minipage} & \begin{minipage}[t]{2cm}.false.\end{minipage} & \begin{minipage}[t]{12cm}If .true., then the output of the real particle vector will be written to the file 'DoHIA.dat'.\end{minipage}\\*
\midrule
flag\_outputPert & \begin{minipage}[t]{2cm}logical\end{minipage} & \begin{minipage}[t]{2cm}.false.\end{minipage} & \begin{minipage}[t]{12cm}If .false., then the output of the perturbative particle vector will be written to the file 'DoHIA\_pert.dat'.\end{minipage}\\*
\midrule
flag\_outputDetailed & \begin{minipage}[t]{2cm}logical\end{minipage} & \begin{minipage}[t]{2cm}.false.\end{minipage} & \begin{minipage}[t]{12cm}Print out more detailed information at each time step from subroutine HeavyIon\_evol:\begin{itemize}\leftmargin0em\itemindent0pt\item rhorad\_*.dat\item rhoz\_*.dat\item rhozx\_*.dat\item Fields\_*.dat\item pauli\_*.dat\item dens\_max.dat\end{itemize}\end{minipage}\\*
\midrule
pionAnalysis & \begin{minipage}[t]{2cm}logical\end{minipage} & \begin{minipage}[t]{2cm}.false.\end{minipage} & \begin{minipage}[t]{12cm}This flag generates various pion spectra (p\_T, m\_T, y, etc). The analysis operates under the assumption of a fixed target, and expects the collision to be performed in the CMS system (cf. cmsFlag in namelist /heavyIon/). The analysis matches the one applied to the HADES data in Agakishiev et al., Eur.Phys.J. A40 (2009) 45-49.\end{minipage}\\*
\midrule
etaAnalysis & \begin{minipage}[t]{2cm}logical\end{minipage} & \begin{minipage}[t]{2cm}.false.\end{minipage} & \begin{minipage}[t]{12cm}This flag generates various eta spectra and eta-related analyses.\end{minipage}\\*
\midrule
rapBinningMode & \begin{minipage}[t]{2cm}integer\end{minipage} & \begin{minipage}[t]{2cm}1\end{minipage} & \begin{minipage}[t]{12cm}Select the variable the 'rapBinning' is given for:\begin{itemize}\leftmargin0em\itemindent0pt\item 1: variable is y0 = y/y\_cms (the normalised rapidity)\item 2: variable is y\end{itemize}\end{minipage}\\*
\midrule
rapBinning & \begin{minipage}[t]{2cm}real, dimension(0:13)\end{minipage} & \begin{minipage}[t]{2cm}(/ -0.75, -0.45, -0.15, 0.15, 0.45, 0.75, 1.05, 1.35, -99.9, -99.9, -99.9, -99.9, -99.9, -99.9 /)\end{minipage} & \begin{minipage}[t]{12cm}Rapidity binning for the pion and eta analysis (only used if pionAnalysis = .true. or etaAnalysis = .true. ). The numbers represent the binning borders in y (or y0, see rapBinningMode). For each of the bins, a separate pT and/or mT spectrum will be generated.\\ Only bins, where the upper bound is larger than the lower one are considered\\NOTES\begin{itemize}\leftmargin0em\itemindent0pt\item for the Hades AuAu analysis, you should set the bins to   -0.65,-0.55,...0.75\end{itemize}\end{minipage}\\*
\midrule
KaonAnalysis & \begin{minipage}[t]{2cm}logical\end{minipage} & \begin{minipage}[t]{2cm}.false.\end{minipage} & \begin{minipage}[t]{12cm}This flag generates various Kaon spectra and Kaon-related analyses.\end{minipage}\\*
\midrule
DensityPlot & \begin{minipage}[t]{2cm}logical\end{minipage} & \begin{minipage}[t]{2cm}.false.\end{minipage} & \begin{minipage}[t]{12cm}This flag select printing the density for several time steps\end{minipage}\\*
\midrule
NucleonMassPlot & \begin{minipage}[t]{2cm}logical\end{minipage} & \begin{minipage}[t]{2cm}.false.\end{minipage} & \begin{minipage}[t]{12cm}This flag select printing the (invariant) mass of the nucleons for several time steps\end{minipage}\\*
\midrule
do\_QRvector & \begin{minipage}[t]{2cm}logical\end{minipage} & \begin{minipage}[t]{2cm}.false.\end{minipage} & \begin{minipage}[t]{12cm}Switch for QRvector output.\end{minipage}\\*
\midrule
do\_Glauber & \begin{minipage}[t]{2cm}logical\end{minipage} & \begin{minipage}[t]{2cm}.false.\end{minipage} & \begin{minipage}[t]{12cm}Switch for Glauber-MC analysis at timestep 0\end{minipage}\\*
\midrule
do\_Tmunu & \begin{minipage}[t]{2cm}logical\end{minipage} & \begin{minipage}[t]{2cm}.false.\end{minipage} & \begin{minipage}[t]{12cm}Switch for Tmunu output.\end{minipage}\\*
\midrule
BarMes\_Tmunu & \begin{minipage}[t]{2cm}logical\end{minipage} & \begin{minipage}[t]{2cm}.false.\end{minipage} & \begin{minipage}[t]{12cm}If .true., then Tmunu is calculated for baryons and mesons separately.\end{minipage}\\*
\midrule
rotateZ\_Tmunu & \begin{minipage}[t]{2cm}logical\end{minipage} & \begin{minipage}[t]{2cm}.false.\end{minipage} & \begin{minipage}[t]{12cm}select, whether the particles are first rotated to be aligned to the z-axis\end{minipage}\\*
\midrule
correctPot\_Tmunu & \begin{minipage}[t]{2cm}integer\end{minipage} & \begin{minipage}[t]{2cm}0\end{minipage} & \begin{minipage}[t]{12cm}select, whether the energy is corrected for the potential or not:\begin{itemize}\leftmargin0em\itemindent0pt\item 0: no correction\item 1: full potential added to p0\item 2: only U\_b/2+U\_r added to p0\item 3: U\_b/2+U\_r added to p0 in the LRF\end{itemize}\end{minipage}\\*
\midrule
selectTmunuFormat & \begin{minipage}[t]{2cm}integer\end{minipage} & \begin{minipage}[t]{2cm}2\end{minipage} & \begin{minipage}[t]{12cm}select output format of Tmunu (binary encoded):\begin{itemize}\leftmargin0em\itemindent0pt\item 1: ASCII\item 2: Binary\item 3: ASCII + Binary\end{itemize}\end{minipage}\\*
\midrule
useSet & \begin{minipage}[t]{2cm}logical, dimension(nSet)\end{minipage} & \begin{minipage}[t]{2cm}(/ .false., .true., .false., .true., .true. /)\end{minipage} & \begin{minipage}[t]{12cm}Array to indicate, which particle set will be used for output\end{minipage}\\*
\midrule
nPartAnalysis & \begin{minipage}[t]{2cm}logical\end{minipage} & \begin{minipage}[t]{2cm}.false.\end{minipage} & \begin{minipage}[t]{12cm}This flag generates output about impact parameter and N\_part\end{minipage}\\*
\bottomrule
\end{longtable}
{ }



% HiGammaNucleus     code/init/ElectronGenerator/eventGenerator_eN_HiEnergy.f90

\begin{longtable}{llll}
\toprule
\textbf{\large{HiGammaNucleus}} & \multicolumn{3}{l}{\footnotesize{code/init/ElectronGenerator/eventGenerator\_eN\_HiEnergy.f90}}\\*
\midrule
\endfirsthead
\midrule
\endhead
DoLowEv & \begin{minipage}[t]{2cm}logical\end{minipage} & \begin{minipage}[t]{2cm}.true.\end{minipage} & \begin{minipage}[t]{12cm}If this flag is set true, then for W\_free$<$HighEnergyThreshold we will call the low energy model routines.\end{minipage}\\*
\midrule
DoTransEv & \begin{minipage}[t]{2cm}logical\end{minipage} & \begin{minipage}[t]{2cm}.false.\end{minipage} & \begin{minipage}[t]{12cm}flag: use transitionEvent in order to replace PYTHIA events by events where we give the cross section explicitely and do the remaining stuff by FRITIOF\\NOTES\\ this replaces the flag "FRITIOF" in the namelists "HiLeptonNucleus" and "HiPhotonNucleus"\end{minipage}\\*
\midrule
useHermesPythiaPars & \begin{minipage}[t]{2cm}logical\end{minipage} & \begin{minipage}[t]{2cm}.false.\end{minipage} & \begin{minipage}[t]{12cm}flag: Use "PYTHIA tuning done by HERMES collab"\end{minipage}\\*
\midrule
DoDiffr & \begin{minipage}[t]{2cm}logical\end{minipage} & \begin{minipage}[t]{2cm}.true.\end{minipage} & \begin{minipage}[t]{12cm}flag: Generate diffractive events\end{minipage}\\*
\midrule
PYTHIAthresh & \begin{minipage}[t]{2cm}real\end{minipage} & \begin{minipage}[t]{2cm}2.0\end{minipage} & \begin{minipage}[t]{12cm}Below this value for W, PYTHIA is not used to generate (G)VMD events\\NOTES\\ This value is transferred to PyVP.f. you can access this value by the function "GetPYTHIAthresh()".\end{minipage}\\*
\midrule
useVMD\_VM & \begin{minipage}[t]{2cm}logical, dimension(4)\end{minipage} & \begin{minipage}[t]{2cm}.true.\end{minipage} & \begin{minipage}[t]{12cm}These flags can be used to switch on/off some VM in the VMD description of the events generated by "transitionevent"\\NOTES\begin{itemize}\leftmargin0em\itemindent0pt\item The VMD events of PYTHIA are not affected. (We could change this!)\end{itemize}\end{minipage}\\*
\midrule
useRes & \begin{minipage}[t]{2cm}logical, dimension(2:nres+1)\end{minipage} & \begin{minipage}[t]{2cm}.true.\end{minipage} & \begin{minipage}[t]{12cm}Switch for including/excluding specific resonances\end{minipage}\\*
\midrule
allowRes & \begin{minipage}[t]{2cm}logical\end{minipage} & \begin{minipage}[t]{2cm}.true.\end{minipage} & \begin{minipage}[t]{12cm}Switch for including/excluding resonance contribution.\\ If this is set to .true., 1pion events will just be generated as for the background, but according the full MAID cross section (if at all)\end{minipage}\\*
\midrule
allow1pi & \begin{minipage}[t]{2cm}logical\end{minipage} & \begin{minipage}[t]{2cm}.true.\end{minipage} & \begin{minipage}[t]{12cm}Switch for including/excluding 1pion contribution.\\ Depending on the switch allowRes, 1 pion events will be done according the full cross section or just as a background.\end{minipage}\\*
\midrule
allow2piBack & \begin{minipage}[t]{2cm}logical\end{minipage} & \begin{minipage}[t]{2cm}.true.\end{minipage} & \begin{minipage}[t]{12cm}Switch for including/excluding additional 2pion background.\end{minipage}\\*
\midrule
allowDIS & \begin{minipage}[t]{2cm}logical\end{minipage} & \begin{minipage}[t]{2cm}.true.\end{minipage} & \begin{minipage}[t]{12cm}Switch for including/excluding DIS contribution\end{minipage}\\*
\midrule
allowVMDrho & \begin{minipage}[t]{2cm}logical\end{minipage} & \begin{minipage}[t]{2cm}.true.\end{minipage} & \begin{minipage}[t]{12cm}Switch for including/excluding the VMD gamma N $\rightarrow$ rho0 N contribution\end{minipage}\\*
\midrule
DoToyModel\_pi & \begin{minipage}[t]{2cm}logical\end{minipage} & \begin{minipage}[t]{2cm}.false.\end{minipage} & \begin{minipage}[t]{12cm}flag: Use a Toy model instead of realistic event generation. Only a single pion is generated\end{minipage}\\*
\midrule
DoToyModel\_rho & \begin{minipage}[t]{2cm}logical\end{minipage} & \begin{minipage}[t]{2cm}.false.\end{minipage} & \begin{minipage}[t]{12cm}flag: Use a Toy model instead of realistic event generation Only rho0 N events are generated.\\ Additional assumptions: (c.f.UseFormTime\_ToyModel\_rho) * tau\_F = 0 * tau\_F = m with t\_F = E (boost according E/m, not E\_string/M\_string)\\ In the latter case we suffer also the following simplifications:\begin{itemize}\leftmargin0em\itemindent0pt\item no Q2 dependance\item XS starts with n\_L/n = 0.5 (should be 0.66 for the nucleon)\end{itemize}\end{minipage}\\*
\midrule
UseFormTime\_ToyModel\_rho & \begin{minipage}[t]{2cm}logical\end{minipage} & \begin{minipage}[t]{2cm}.false.\end{minipage} & \begin{minipage}[t]{12cm}flag: if .true., we set the formation times of the particles produced in the ToyModel\_rho equals to the energy of the particle (t\_f/fm=E/GeV) representing the assumption tau\_f/fm = m/GeV plus a boost according E/m. (Otherwise the formation time is set to zero.)\end{minipage}\\*
\midrule
DoExclPiModel & \begin{minipage}[t]{2cm}logical\end{minipage} & \begin{minipage}[t]{2cm}.false.\end{minipage} & \begin{minipage}[t]{12cm}flag: Use a model for exclusive pion production. Only those events are generated\end{minipage}\\*
\midrule
ExclPiCharge & \begin{minipage}[t]{2cm}integer\end{minipage} & \begin{minipage}[t]{2cm}1\end{minipage} & \begin{minipage}[t]{12cm}variable to specify the charge of the pion produced, if DoExclPiModel is selected\end{minipage}\\*
\midrule
flagTwoJets & \begin{minipage}[t]{2cm}logical\end{minipage} & \begin{minipage}[t]{2cm}.false.\end{minipage} & \begin{minipage}[t]{12cm}If .true. - the events without two jets with large transverse momentum are marked with XS\_tot=100000 mub.\end{minipage}\\*
\bottomrule
\end{longtable}
{ }



% HiLepton_Analysis     code/analysis/HiLeptonAnalysis.f90

\begin{longtable}{llll}
\toprule
\textbf{\large{HiLepton\_Analysis}} & \multicolumn{3}{l}{\footnotesize{code/analysis/HiLeptonAnalysis.f90}}\\*
\midrule
\endfirsthead
\midrule
\endhead
DoTimes & \begin{minipage}[t]{2cm}logical\end{minipage} & \begin{minipage}[t]{2cm}.false.\end{minipage} & \begin{minipage}[t]{12cm}switch on/off: reporting of times\end{minipage}\\*
\midrule
DoOutChannels & \begin{minipage}[t]{2cm}logical\end{minipage} & \begin{minipage}[t]{2cm}.false.\end{minipage} & \begin{minipage}[t]{12cm}switch on/off: reporting of all final state channels\end{minipage}\\*
\midrule
DoInvMasses & \begin{minipage}[t]{2cm}logical\end{minipage} & \begin{minipage}[t]{2cm}.false.\end{minipage} & \begin{minipage}[t]{12cm}switch on/off: reporting of pairwise-invariant-masses\end{minipage}\\*
\midrule
DoFindRho0 & \begin{minipage}[t]{2cm}logical\end{minipage} & \begin{minipage}[t]{2cm}.false.\end{minipage} & \begin{minipage}[t]{12cm}switch on/off: reconstructing rho0 from final pions\end{minipage}\\*
\midrule
DoClasie & \begin{minipage}[t]{2cm}logical\end{minipage} & \begin{minipage}[t]{2cm}.false.\end{minipage} & \begin{minipage}[t]{12cm}switch on/off: Do pion analysis as Clasie et al., arXiv:0701.1481\end{minipage}\\*
\midrule
DoMorrow & \begin{minipage}[t]{2cm}logical\end{minipage} & \begin{minipage}[t]{2cm}.false.\end{minipage} & \begin{minipage}[t]{12cm}switch on/off: Do pion analysis as Morrow et al., Morrow:2008ek\end{minipage}\\*
\midrule
DoBrooks & \begin{minipage}[t]{2cm}logical\end{minipage} & \begin{minipage}[t]{2cm}.false.\end{minipage} & \begin{minipage}[t]{12cm}switch on/off: Do pi+ pT2 spectra for Brooks delta pT2\end{minipage}\\*
\midrule
DoMandelT & \begin{minipage}[t]{2cm}logical\end{minipage} & \begin{minipage}[t]{2cm}.false.\end{minipage} & \begin{minipage}[t]{12cm}switch on/off: Do pion analysis with Mandelstam t.\end{minipage}\\*
\midrule
DoClassifyFirst & \begin{minipage}[t]{2cm}logical\end{minipage} & \begin{minipage}[t]{2cm}.false.\end{minipage} & \begin{minipage}[t]{12cm}Classifying 'FirstEvent' into some classes\\ Needs DoEventAdd.\end{minipage}\\*
\midrule
DoFSIsqrts & \begin{minipage}[t]{2cm}logical\end{minipage} & \begin{minipage}[t]{2cm}.false.\end{minipage} & \begin{minipage}[t]{12cm}switch on/off: Estimate potential/future final state interactions\\ Plot the sqrt(s) distribution of potential final state interactions of perturbative particles with the nucleus (real) particles). (The interactions do not happen, this is calculated before every propagation.) In order to select the particle class for which one wants to report the FSI, please change directly the code.\end{minipage}\\*
\midrule
DoCentralN & \begin{minipage}[t]{2cm}logical\end{minipage} & \begin{minipage}[t]{2cm}.false.\end{minipage} & \begin{minipage}[t]{12cm}switch on/off: Do centrality analysis with slow nucleons\end{minipage}\\*
\midrule
DoLeptonKinematics & \begin{minipage}[t]{2cm}logical\end{minipage} & \begin{minipage}[t]{2cm}.false.\end{minipage} & \begin{minipage}[t]{12cm}switch on/off lepton kinematics output\end{minipage}\\*
\midrule
DoHadronKinematics & \begin{minipage}[t]{2cm}logical\end{minipage} & \begin{minipage}[t]{2cm}.false.\end{minipage} & \begin{minipage}[t]{12cm}switch on/off hadron kinematics output\end{minipage}\\*
\midrule
flagDoIt & \begin{minipage}[t]{2cm}logical\end{minipage} & \begin{minipage}[t]{2cm}.true.\end{minipage} & \begin{minipage}[t]{12cm}switch on/off using DoHiLeptonAnalysis\end{minipage}\\*
\bottomrule
\end{longtable}
{ }



% HiLeptonNucleus     code/init/initHiLepton.f90

\begin{longtable}{llll}
\toprule
\textbf{\large{HiLeptonNucleus}} & \multicolumn{3}{l}{\footnotesize{code/init/initHiLepton.f90}}\\*
\midrule
\endfirsthead
\midrule
\endhead
iExperiment & \begin{minipage}[t]{2cm}integer\end{minipage} & \begin{minipage}[t]{2cm}0\end{minipage} & \begin{minipage}[t]{12cm}choice of experiment, detector and energy\\ possible values are:\begin{itemize}\leftmargin0em\itemindent0pt\item  0: no experiment/fixed kinematics\item  1: Hermes, 27GeV, D,N,Kr\item  2: Hermes, 27GeV, Ne\item  3: Hermes, 27GeV, H\item  4: JLAB, 12GeV\item  5: JLAB,  5GeV\item  6: EMC, 100GeV\item  7: EMC, 120GeV\item  8: EMC, 200GeV\item  9: EMC, 280GeV\item 10: Hermes, 12GeV\item 11: Hermes, 27GeV, arXiv:0704.3270\item 12: Mainz, Yoon: Ebeam=1.5GeV\item 13: Hermes, 27GeV, arXiv:0704.3712 (pT-broadening)\item 14: JLAB,  5GeV, rho0 experiment\item 15: JLAB,  4GeV, rho0 experiment\item 16: EIC, E\_e and E\_A given explicit (3+30,11+30,4+100)\item 17: no detector, total cross section, Ebeam\item 18: E665, 470GeV\item 19: CLAS/JLAB, 12GeV RunGroupA optimized 10.6 GeV\item 20: CLAS/JLAB, 12GeV RunGroupA theoterical\end{itemize} please note: The entry "iExperiment == 0" replaces the old HiPhoton event type.\end{minipage}\\*
\midrule
shadow & \begin{minipage}[t]{2cm}logical\end{minipage} & \begin{minipage}[t]{2cm}.true.\end{minipage} & \begin{minipage}[t]{12cm}flag: Consider shadowing or not\end{minipage}\\*
\midrule
minimumMomentum & \begin{minipage}[t]{2cm}real\end{minipage} & \begin{minipage}[t]{2cm}0.1\end{minipage} & \begin{minipage}[t]{12cm}minimal momentum considered. (in GeV)\end{minipage}\\*
\midrule
ModusCalcFluxNorm & \begin{minipage}[t]{2cm}logical\end{minipage} & \begin{minipage}[t]{2cm}.false.\end{minipage} & \begin{minipage}[t]{12cm}if this flag is true, than we do not really generate events. We only select nu and Q2 according an equal distribution and plot the flux (and the flux multiplied with AccWeight). Normally we choose nu and Q2 according flux*Accweight via von-Neumann- rejection method (where we loose access to the absolute normalisation).\end{minipage}\\*
\midrule
iDetector & \begin{minipage}[t]{2cm}integer\end{minipage} & \begin{minipage}[t]{2cm}-1\end{minipage} & \begin{minipage}[t]{12cm}This sets the treatment of the detector:\begin{itemize}\leftmargin0em\itemindent0pt\item -1 : not valid/not initialized/use default\item  0 : no detector, as AccFlag=.false.\item  1 : HERMES, full efficiency\item  2 : EMC, full efficiency\item  3 : CLAS, only cuts (th\_e=12°..50°, th\_hadron=6°..143°)\item  4 : CLAS, full efficiency + cuts as for 5GeV\item  5 : CLAS, electron: cuts (th\_e=12°..50°),              hadrons: efficiency+cuts as for 5GeV\item 90 : full acceptance\end{itemize}\end{minipage}\\*
\midrule
EIC\_Ee & \begin{minipage}[t]{2cm}real\end{minipage} & \begin{minipage}[t]{2cm}-99.9\end{minipage} & \begin{minipage}[t]{12cm}the electron beam energy, if iExperiment=EIC\end{minipage}\\*
\midrule
EIC\_EA & \begin{minipage}[t]{2cm}real\end{minipage} & \begin{minipage}[t]{2cm}-99.9\end{minipage} & \begin{minipage}[t]{12cm}the hadron beam energy, if iExperiment=EIC\end{minipage}\\*
\midrule
realRun & \begin{minipage}[t]{2cm}logical\end{minipage} & \begin{minipage}[t]{2cm}.false.\end{minipage} & \begin{minipage}[t]{12cm}Flag to indicate, whether we produce real or perturbative particles.\\NOTES\\ run with real particles untested !!!\end{minipage}\\*
\midrule
DoStatistics & \begin{minipage}[t]{2cm}logical\end{minipage} & \begin{minipage}[t]{2cm}.false.\end{minipage} & \begin{minipage}[t]{12cm}switch on/off statistical output of init routines\end{minipage}\\*
\midrule
user\_numin & \begin{minipage}[t]{2cm}real\end{minipage} & \begin{minipage}[t]{2cm}-99.9\end{minipage} & \begin{minipage}[t]{12cm}user given value for numin, overrides default value if reasonable\end{minipage}\\*
\midrule
user\_numax & \begin{minipage}[t]{2cm}real\end{minipage} & \begin{minipage}[t]{2cm}-99.9\end{minipage} & \begin{minipage}[t]{12cm}user given value for numax, overrides default value if reasonable\end{minipage}\\*
\midrule
user\_costmin & \begin{minipage}[t]{2cm}real\end{minipage} & \begin{minipage}[t]{2cm}-99.9\end{minipage} & \begin{minipage}[t]{12cm}user given value for costmin, overrides default value if reasonable\end{minipage}\\*
\midrule
user\_costmax & \begin{minipage}[t]{2cm}real\end{minipage} & \begin{minipage}[t]{2cm}99.9\end{minipage} & \begin{minipage}[t]{12cm}user given value for costmax, overrides default value if reasonable\end{minipage}\\*
\midrule
user\_ymax & \begin{minipage}[t]{2cm}real\end{minipage} & \begin{minipage}[t]{2cm}-99.9\end{minipage} & \begin{minipage}[t]{12cm}user given value for ymax, overrides default value if reasonable\end{minipage}\\*
\midrule
user\_smin & \begin{minipage}[t]{2cm}real\end{minipage} & \begin{minipage}[t]{2cm}-99.9\end{minipage} & \begin{minipage}[t]{12cm}user given value for smin, overrides default value if reasonable\end{minipage}\\*
\midrule
user\_xBmin & \begin{minipage}[t]{2cm}real\end{minipage} & \begin{minipage}[t]{2cm}-99.9\end{minipage} & \begin{minipage}[t]{12cm}user given value for xBmin, overrides default value if reasonable\end{minipage}\\*
\midrule
user\_qsqmin & \begin{minipage}[t]{2cm}real\end{minipage} & \begin{minipage}[t]{2cm}-99.9\end{minipage} & \begin{minipage}[t]{12cm}user given value for qsqmin, overrides default value if reasonable\end{minipage}\\*
\midrule
user\_qsqmax & \begin{minipage}[t]{2cm}real\end{minipage} & \begin{minipage}[t]{2cm}-99.9\end{minipage} & \begin{minipage}[t]{12cm}user given value for qsqmax, overrides default value if reasonable\end{minipage}\\*
\midrule
user\_maxw & \begin{minipage}[t]{2cm}real\end{minipage} & \begin{minipage}[t]{2cm}-99.9\end{minipage} & \begin{minipage}[t]{12cm}user given value for maxw, overrides default value if reasonable\end{minipage}\\*
\midrule
earlyPauli & \begin{minipage}[t]{2cm}logical\end{minipage} & \begin{minipage}[t]{2cm}.false.\end{minipage} & \begin{minipage}[t]{12cm}Flag to indicate, whether we should check Pauli blocking already during generation or only at the end.\\ if .false. (default), events will be generated in a first stage without Pauli blocking. This is then tested afterwards. If the generated event is blocked, it will be redone! Thus Pauli blocking does *not* change the total cross section, only the relative strength will be reshuffled.\\ if .true., then blocked events will be excluded from the Monte Carlo decision and the total cross section will be reduced.\\NOTES\\ The behaviour, if no event at all is possible, is at the moment a little bit unpredictable ;)\end{minipage}\\*
\midrule
equalWeights\_Mode & \begin{minipage}[t]{2cm}integer\end{minipage} & \begin{minipage}[t]{2cm}0\end{minipage} & \begin{minipage}[t]{12cm}possible values are:\begin{itemize}\leftmargin0em\itemindent0pt\item 0: default perweight mode is used (default)\item 1: default perweight mode is used, but max is printed\item 2: MC rejection method is used.\end{itemize} In the default mode, the perweights of the final particles are given by cross section/(A * numEnsembles)\\ If equalWeightsMode==2, then the perweights are given by equalWeights\_Max/(A * numEnsembles)\end{minipage}\\*
\midrule
equalWeights\_Max & \begin{minipage}[t]{2cm}real\end{minipage} & \begin{minipage}[t]{2cm}-1e99\end{minipage} & \begin{minipage}[t]{12cm}The maximum value the MC-rejection method is done against.\end{minipage}\\*
\bottomrule
\end{longtable}
{ }



% HiPhotonKinematics     code/init/initHiLepton.f90

\begin{longtable}{llll}
\toprule
\textbf{\large{HiPhotonKinematics}} & \multicolumn{3}{l}{\footnotesize{code/init/initHiLepton.f90}}\\*
\midrule
\endfirsthead
\midrule
\endhead
nu & \begin{minipage}[t]{2cm}real\end{minipage} & \begin{minipage}[t]{2cm}-99.9\end{minipage} & \begin{minipage}[t]{12cm}Photon energy [GeV]\end{minipage}\\*
\midrule
Q2 & \begin{minipage}[t]{2cm}real\end{minipage} & \begin{minipage}[t]{2cm}-99.9\end{minipage} & \begin{minipage}[t]{12cm}transfer four momentum squared [GeV\^{}2]\end{minipage}\\*
\midrule
eps & \begin{minipage}[t]{2cm}real\end{minipage} & \begin{minipage}[t]{2cm}-99.9\end{minipage} & \begin{minipage}[t]{12cm}Photon polarisation [1]\end{minipage}\\*
\midrule
srts & \begin{minipage}[t]{2cm}real\end{minipage} & \begin{minipage}[t]{2cm}-99.9\end{minipage} & \begin{minipage}[t]{12cm}sqrt(s) of electron nucleon system [GeV]\end{minipage}\\*
\midrule
W & \begin{minipage}[t]{2cm}real\end{minipage} & \begin{minipage}[t]{2cm}-99.9\end{minipage} & \begin{minipage}[t]{12cm}sqrt(s) of photon nucleon system [GeV]\end{minipage}\\*
\midrule
xBj & \begin{minipage}[t]{2cm}real\end{minipage} & \begin{minipage}[t]{2cm}-99.9\end{minipage} & \begin{minipage}[t]{12cm}Bjorken x [1]\end{minipage}\\*
\midrule
Ebeam & \begin{minipage}[t]{2cm}real\end{minipage} & \begin{minipage}[t]{2cm}\end{minipage} & \begin{minipage}[t]{12cm}electron beam energy [GeV]\end{minipage}\\*
\bottomrule
\end{longtable}
{ }



% HiPion_Analysis     code/analysis/HiPionAnalysis.f90

\begin{longtable}{llll}
\toprule
\textbf{\large{HiPion\_Analysis}} & \multicolumn{3}{l}{\footnotesize{code/analysis/HiPionAnalysis.f90}}\\*
\midrule
\endfirsthead
\midrule
\endhead
Enable & \begin{minipage}[t]{2cm}logical\end{minipage} & \begin{minipage}[t]{2cm}.true.\end{minipage} & \begin{minipage}[t]{12cm}If .true. the HiPion analysis will be performed, otherwise not.\end{minipage}\\*
\midrule
EnablePerTime & \begin{minipage}[t]{2cm}logical\end{minipage} & \begin{minipage}[t]{2cm}.false.\end{minipage} & \begin{minipage}[t]{12cm}If .true. the HiPion analysis per timestep will be performed, otherwise not.\end{minipage}\\*
\midrule
DoSimpleKin & \begin{minipage}[t]{2cm}logical\end{minipage} & \begin{minipage}[t]{2cm}.false.\end{minipage} & \begin{minipage}[t]{12cm}switch on/off: Analysis for simple kinematics: pZ-, pT-spectra etc.\end{minipage}\\*
\midrule
DoHarp & \begin{minipage}[t]{2cm}logical\end{minipage} & \begin{minipage}[t]{2cm}.false.\end{minipage} & \begin{minipage}[t]{12cm}switch on/off: Analysis for the HARP experiment\end{minipage}\\*
\midrule
DoBlobel & \begin{minipage}[t]{2cm}logical\end{minipage} & \begin{minipage}[t]{2cm}.false.\end{minipage} & \begin{minipage}[t]{12cm}switch on/off: Analysis according Blobel et al.\end{minipage}\\*
\midrule
DoInvMasses & \begin{minipage}[t]{2cm}logical\end{minipage} & \begin{minipage}[t]{2cm}.false.\end{minipage} & \begin{minipage}[t]{12cm}switch on/off: reporting of pairwise-invariant-masses\end{minipage}\\*
\midrule
DoDOmega & \begin{minipage}[t]{2cm}logical\end{minipage} & \begin{minipage}[t]{2cm}.false.\end{minipage} & \begin{minipage}[t]{12cm}switch on/off: Analysis for dSigma/dOmega\end{minipage}\\*
\midrule
DoOutChannels & \begin{minipage}[t]{2cm}logical\end{minipage} & \begin{minipage}[t]{2cm}.false.\end{minipage} & \begin{minipage}[t]{12cm}switch on/off: reporting of all final state channels\end{minipage}\\*
\bottomrule
\end{longtable}
{ }



% HiPionNucleus     code/init/initHiPion.f90

\begin{longtable}{llll}
\toprule
\textbf{\large{HiPionNucleus}} & \multicolumn{3}{l}{\footnotesize{code/init/initHiPion.f90}}\\*
\midrule
\endfirsthead
\midrule
\endhead
distance & \begin{minipage}[t]{2cm}real\end{minipage} & \begin{minipage}[t]{2cm}15.\end{minipage} & \begin{minipage}[t]{12cm}Distance in z-direction from the nucleus center in fm, where the projectiles are initialzed. If negative, the distance will be chosen automatically.\end{minipage}\\*
\midrule
impact\_parameter & \begin{minipage}[t]{2cm}real\end{minipage} & \begin{minipage}[t]{2cm}0.\end{minipage} & \begin{minipage}[t]{12cm}Impact parameter of the projectiles in fm. If positive (or zero), this fixed value is used for all projectiles. If negative, the impact parameter is chosen by Monte Carlo, so that the projectiles are distributed over a certain disk. Cf. 'setPosition'.\end{minipage}\\*
\midrule
ProjectileCharge & \begin{minipage}[t]{2cm}integer\end{minipage} & \begin{minipage}[t]{2cm}0\end{minipage} & \begin{minipage}[t]{12cm}Charge of projectile particles.\end{minipage}\\*
\midrule
ProjectileID & \begin{minipage}[t]{2cm}integer\end{minipage} & \begin{minipage}[t]{2cm}pion\end{minipage} & \begin{minipage}[t]{12cm}ID of projectile particles.\end{minipage}\\*
\midrule
ProjectileAnti & \begin{minipage}[t]{2cm}logical\end{minipage} & \begin{minipage}[t]{2cm}.false.\end{minipage} & \begin{minipage}[t]{12cm}Antiparticle flag of projectile particles.\end{minipage}\\*
\midrule
nTestparticles & \begin{minipage}[t]{2cm}integer\end{minipage} & \begin{minipage}[t]{2cm}200\end{minipage} & \begin{minipage}[t]{12cm}Number of projectile testparticles per ensemble.\end{minipage}\\*
\midrule
ekin\_lab & \begin{minipage}[t]{2cm}real\end{minipage} & \begin{minipage}[t]{2cm}-99.9\end{minipage} & \begin{minipage}[t]{12cm}Kinetic energy of projectile particles in lab frame [GeV].\end{minipage}\\*
\midrule
p\_lab & \begin{minipage}[t]{2cm}real\end{minipage} & \begin{minipage}[t]{2cm}-99.9\end{minipage} & \begin{minipage}[t]{12cm}Momentum of projectile particles in lab frame [GeV/c].\end{minipage}\\*
\midrule
DoPerturbativeInit & \begin{minipage}[t]{2cm}logical\end{minipage} & \begin{minipage}[t]{2cm}.false.\end{minipage} & \begin{minipage}[t]{12cm}If this flag is set to .true., the first collision of the projectile particles with a nucleon in the target nucleus will be done in this init routine (at timestep 0). This enables you to treat the first (hard) collision different from those in the FSI.\\ If this flag is set to .false., the projectile particles have to be propagated onto the nucleus as in the default transport treatment.\\ See documentation of 'initHiPionInducedCollide' and 'initHiPionInducedCollideFull' for further information.\end{minipage}\\*
\midrule
DoOnlyOne & \begin{minipage}[t]{2cm}logical\end{minipage} & \begin{minipage}[t]{2cm}.true.\end{minipage} & \begin{minipage}[t]{12cm}If the first interaction of beam and target particles is treated already here in the init (cf. DoPerturbativeInit), you may select whether a beam particle may interact only once (flag set to .true.) or with all other target nucleons (flag set to .false.).\\ See documentation of 'initHiPionInducedCollide' and 'initHiPionInducedCollideFull' for further information.\end{minipage}\\*
\midrule
minimumMomentum & \begin{minipage}[t]{2cm}real\end{minipage} & \begin{minipage}[t]{2cm}1.0\end{minipage} & \begin{minipage}[t]{12cm}Minimal momentum of particles (in GeV) produced in the init routines. Only particles with an absolute momentum larger than this will be further propagated.\end{minipage}\\*
\midrule
useHermesPythiaPars & \begin{minipage}[t]{2cm}logical\end{minipage} & \begin{minipage}[t]{2cm}.false.\end{minipage} & \begin{minipage}[t]{12cm}flag: Use "PYTHIA tuning done by HERMES collab"\end{minipage}\\*
\midrule
NucCharge & \begin{minipage}[t]{2cm}integer\end{minipage} & \begin{minipage}[t]{2cm}-1\end{minipage} & \begin{minipage}[t]{12cm}Select charge state of nucleons to scatter on. If this value is $>$=0, then we only scatter on nucleons with the respective charge, i.e. only on neutrons if NucCharge==0 and only on protons if NucCharge==1. Useful e.g. for selecting only pn events in a pd collision.\end{minipage}\\*
\midrule
flagOffShell & \begin{minipage}[t]{2cm}logical\end{minipage} & \begin{minipage}[t]{2cm}.false.\end{minipage} & \begin{minipage}[t]{12cm}If this flag is set to .true., the struck nucleon in the deuteron will be off vacuum mass shell to agree with total energy conservation. Relevant for the deuteron target only.\end{minipage}\\*
\midrule
flagFlux & \begin{minipage}[t]{2cm}logical\end{minipage} & \begin{minipage}[t]{2cm}.false.\end{minipage} & \begin{minipage}[t]{12cm}If this flag is set to .true., the Moeller flux factor will be included in calculation of the cross section with nuclear target.\end{minipage}\\*
\midrule
flagLC & \begin{minipage}[t]{2cm}logical\end{minipage} & \begin{minipage}[t]{2cm}.false.\end{minipage} & \begin{minipage}[t]{12cm}If this flag is set to .true., the deuteron wave function will be treated according to the light cone formalism.\end{minipage}\\*
\bottomrule
\end{longtable}
{ }



% History     code/collisions/history.f90

\begin{longtable}{llll}
\toprule
\textbf{\large{History}} & \multicolumn{3}{l}{\footnotesize{code/collisions/history.f90}}\\*
\midrule
\endfirsthead
\midrule
\endhead
IncGeneration\_Decay & \begin{minipage}[t]{2cm}logical\end{minipage} & \begin{minipage}[t]{2cm}.true.\end{minipage} & \begin{minipage}[t]{12cm}This flag determines whether we will increase the stored 'generation' of the daughter particles in a resonance decay.\end{minipage}\\*
\midrule
IncGeneration\_Elastic & \begin{minipage}[t]{2cm}logical\end{minipage} & \begin{minipage}[t]{2cm}.true.\end{minipage} & \begin{minipage}[t]{12cm}This flag determines whether we will increase the stored 'generation' of particles in an elastic collision. Setting it to .false. will also prevent elastic collisions from showing up as parents in the history.\end{minipage}\\*
\bottomrule
\end{longtable}
{ }



% InABoxAnalysis     code/analysis/InABoxAnalysis.f90

\begin{longtable}{llll}
\toprule
\textbf{\large{InABoxAnalysis}} & \multicolumn{3}{l}{\footnotesize{code/analysis/InABoxAnalysis.f90}}\\*
\midrule
\endfirsthead
\midrule
\endhead
Enable & \begin{minipage}[t]{2cm}logical\end{minipage} & \begin{minipage}[t]{2cm}.true.\end{minipage} & \begin{minipage}[t]{12cm}Flag to enable or disable the box analysis alltogether.\end{minipage}\\*
\midrule
Interval & \begin{minipage}[t]{2cm}integer\end{minipage} & \begin{minipage}[t]{2cm}20\end{minipage} & \begin{minipage}[t]{12cm}Interval for output, i.e. number of timesteps after which output is written.\end{minipage}\\*
\bottomrule
\end{longtable}
{ }



% initDatabase     code/database/particleProperties.f90

\begin{longtable}{llll}
\toprule
\textbf{\large{initDatabase}} & \multicolumn{3}{l}{\footnotesize{code/database/particleProperties.f90}}\\*
\midrule
\endfirsthead
\midrule
\endhead
propagationSwitch & \begin{minipage}[t]{2cm}integer\end{minipage} & \begin{minipage}[t]{2cm}3\end{minipage} & \begin{minipage}[t]{12cm}\begin{itemize}\leftmargin0em\itemindent0pt\item 0 = propagate resonances with more than 1 star in their rating   (irrespar=0 in old code)\item 1 = propagate just the Delta (irrespar=2 in old code)\item 2 = propagate no resonance (irrespar=3 in old code)\item 3 = propagate all resonances (default)\end{itemize}\end{minipage}\\*
\midrule
usageForXsectionSwitch & \begin{minipage}[t]{2cm}integer\end{minipage} & \begin{minipage}[t]{2cm}2\end{minipage} & \begin{minipage}[t]{12cm}\begin{itemize}\leftmargin0em\itemindent0pt\item 0 = use resonances with more than 1 star rating for cross sections\item 1 = use all resonances for cross sections\item 2 = use all resonances besides the 1* star I=1/2 resonances\item 3 = use only the Delta\end{itemize}\end{minipage}\\*
\midrule
rho\_dilep & \begin{minipage}[t]{2cm}logical\end{minipage} & \begin{minipage}[t]{2cm}.false.\end{minipage} & \begin{minipage}[t]{12cm}If .false. (default), the rho meson width will be exclusively given by the 2pi decay and its minmass will be 2m\_pi. If .true., the dilepton width will be included in the width and spectral function of the rho, and the minmass will be 2m\_e. This is important for dilepton spectra, in order to get contributions from the rho below the 2pi threshold.\end{minipage}\\*
\midrule
FileNameDecayChannels & \begin{minipage}[t]{2cm}character(1000)\end{minipage} & \begin{minipage}[t]{2cm}''\end{minipage} & \begin{minipage}[t]{12cm}The absolute filename of the file containing decay channel infos.\\ possible values:\begin{itemize}\leftmargin0em\itemindent0pt\item if not set, default is '[path\_To\_Input]/DecayChannels.dat'\item if given, but does not contain '/':   default is '[path\_To\_Input]/[FileNameDecayChannels]'\item otherwise: filename is absolute, including path\end{itemize} NOTE if you want to use the file 'XXX.dat' in the actual directory, give it as './XXX.dat'\end{minipage}\\*
\bottomrule
\end{longtable}
{ }



% initDensity     code/density/density.f90

\begin{longtable}{llll}
\toprule
\textbf{\large{initDensity}} & \multicolumn{3}{l}{\footnotesize{code/density/density.f90}}\\*
\midrule
\endfirsthead
\midrule
\endhead
densitySwitch & \begin{minipage}[t]{2cm}integer\end{minipage} & \begin{minipage}[t]{2cm}1\end{minipage} & \begin{minipage}[t]{12cm}This switch decides whether the density is static or dynamic during the run. ("Static" makes sense only for fixed target scenarios!)\\ One can use a static density if the nucleus stays roughly in its ground state during the collision.\\ possible values:\begin{itemize}\leftmargin0em\itemindent0pt\item 0: Density is set to 0.\item 1: Dynamic density according to test-particle distribution.\item 2: Static density (not for heavy-ion collisions).\item 3: Resting matter: Density is given by the two input parameters   "densityInput\_neutron" and "densityInput\_proton".\item 4: Dynamic density in a box. Assumes the same density everywhere, but   also calculates the momentum distribution\end{itemize}\end{minipage}\\*
\midrule
linearInterpolation & \begin{minipage}[t]{2cm}logical\end{minipage} & \begin{minipage}[t]{2cm}.true.\end{minipage} & \begin{minipage}[t]{12cm}If this switch is 'true', then the dynamic-density mode uses linear interpolation to determine the density in between the gridpoints.\end{minipage}\\*
\midrule
densityInput\_proton & \begin{minipage}[t]{2cm}real\end{minipage} & \begin{minipage}[t]{2cm}0.084\end{minipage} & \begin{minipage}[t]{12cm}Assumed proton density if densitySwitch=3\end{minipage}\\*
\midrule
densityInput\_neutron & \begin{minipage}[t]{2cm}real\end{minipage} & \begin{minipage}[t]{2cm}0.084\end{minipage} & \begin{minipage}[t]{12cm}Assumed neutron density if densitySwitch=3\end{minipage}\\*
\midrule
gridSize & \begin{minipage}[t]{2cm}real, dimension(1:3)\end{minipage} & \begin{minipage}[t]{2cm}(/12.,12.,12./)\end{minipage} & \begin{minipage}[t]{12cm}Size of density grid in fm.\end{minipage}\\*
\midrule
gridPoints & \begin{minipage}[t]{2cm}integer, dimension(1:3)\end{minipage} & \begin{minipage}[t]{2cm}(/30,30,30/)\end{minipage} & \begin{minipage}[t]{12cm}Number of gridpoints in each space direction.\end{minipage}\\*
\midrule
setnewsmearing & \begin{minipage}[t]{2cm}logical\end{minipage} & \begin{minipage}[t]{2cm}.false.\end{minipage} & \begin{minipage}[t]{12cm}Readjust the smearing to a different width if .true.\end{minipage}\\*
\midrule
newsmearing & \begin{minipage}[t]{2cm}real\end{minipage} & \begin{minipage}[t]{2cm}1.\end{minipage} & \begin{minipage}[t]{12cm}Use a smearing width as in a grid wtih newsmearing times the gridspacing\end{minipage}\\*
\midrule
nLargePoints & \begin{minipage}[t]{2cm}integer\end{minipage} & \begin{minipage}[t]{2cm}2\end{minipage} & \begin{minipage}[t]{12cm}Number of points which are considered to the left and right to smear density on\end{minipage}\\*
\bottomrule
\end{longtable}
{ }



% initInABox     code/init/initInABox.f90

\begin{longtable}{llll}
\toprule
\textbf{\large{initInABox}} & \multicolumn{3}{l}{\footnotesize{code/init/initInABox.f90}}\\*
\midrule
\endfirsthead
\midrule
\endhead
proton\_Density & \begin{minipage}[t]{2cm}real\end{minipage} & \begin{minipage}[t]{2cm}0.084\end{minipage} & \begin{minipage}[t]{12cm}proton Density [fm\^{}-3]\end{minipage}\\*
\midrule
neutron\_Density & \begin{minipage}[t]{2cm}real\end{minipage} & \begin{minipage}[t]{2cm}0.084\end{minipage} & \begin{minipage}[t]{12cm}neutron Density [fm\^{}-3]\end{minipage}\\*
\midrule
fermiMotion & \begin{minipage}[t]{2cm}logical\end{minipage} & \begin{minipage}[t]{2cm}.true.\end{minipage} & \begin{minipage}[t]{12cm}switch on/off Fermi motion\end{minipage}\\*
\midrule
temp & \begin{minipage}[t]{2cm}real\end{minipage} & \begin{minipage}[t]{2cm}0.\end{minipage} & \begin{minipage}[t]{12cm}If fermiMotion is true, this switch determines the temperature (in GeV) used in the Fermi distribution.\end{minipage}\\*
\midrule
energy\_density & \begin{minipage}[t]{2cm}real\end{minipage} & \begin{minipage}[t]{2cm}0.\end{minipage} & \begin{minipage}[t]{12cm}Energy density in GeV/fm\^{}3. If a finite positive number is given, the box will be boosted to a frame with the given energy density.\end{minipage}\\*
\midrule
standing\_wave\_number & \begin{minipage}[t]{2cm}integer\end{minipage} & \begin{minipage}[t]{2cm}0\end{minipage} & \begin{minipage}[t]{12cm}If this number is larger than zero, the initial density distribution will not be uniform, but is modulated with a standing wave in z direction. The given number determines the number of oscillations throughout the box. The amplitude of the oscillations is currently fixed to be 20\% of the (average) nucleon density.\end{minipage}\\*
\bottomrule
\end{longtable}
{ }



% initNbarN_to_NbarDelta     code/collisions/twoBodyReactions/baryonBaryon/NbarN_to_NbarDelta.f90

\begin{longtable}{llll}
\toprule
\textbf{\large{initNbarN\_to\_NbarDelta}} & \multicolumn{3}{l}{\footnotesize{code/collisions/twoBodyReactions/baryonBaryon/NbarN\_to\_NbarDelta.f90}}\\*
\midrule
\endfirsthead
\midrule
\endhead
delta\_mass & \begin{minipage}[t]{2cm}real\end{minipage} & \begin{minipage}[t]{2cm}0.01\end{minipage} & \begin{minipage}[t]{12cm}\begin{itemize}\leftmargin0em\itemindent0pt\item grid step on a delta mass (GeV)\end{itemize}\end{minipage}\\*
\midrule
maxPoints\_mass & \begin{minipage}[t]{2cm}integer\end{minipage} & \begin{minipage}[t]{2cm}150\end{minipage} & \begin{minipage}[t]{12cm}\begin{itemize}\leftmargin0em\itemindent0pt\item number of the grid points on the delta mass\end{itemize}\end{minipage}\\*
\midrule
delta\_srts & \begin{minipage}[t]{2cm}real\end{minipage} & \begin{minipage}[t]{2cm}0.01\end{minipage} & \begin{minipage}[t]{12cm}\begin{itemize}\leftmargin0em\itemindent0pt\item grid step on an invariant energy (GeV)\end{itemize}\end{minipage}\\*
\midrule
maxPoints\_srts & \begin{minipage}[t]{2cm}integer\end{minipage} & \begin{minipage}[t]{2cm}100\end{minipage} & \begin{minipage}[t]{12cm}\begin{itemize}\leftmargin0em\itemindent0pt\item number of the grid points on the invariant energy\end{itemize}\end{minipage}\\*
\bottomrule
\end{longtable}
{ }



% InitNucleus_in_PS     code/init/initNucPhaseSpace.f90

\begin{longtable}{llll}
\toprule
\textbf{\large{InitNucleus\_in\_PS}} & \multicolumn{3}{l}{\footnotesize{code/init/initNucPhaseSpace.f90}}\\*
\midrule
\endfirsthead
\midrule
\endhead
improvedMC & \begin{minipage}[t]{2cm}logical\end{minipage} & \begin{minipage}[t]{2cm}.false.\end{minipage} & \begin{minipage}[t]{12cm}\begin{itemize}\leftmargin0em\itemindent0pt\item If this flag is set to .true. then we use the information of the already   initialized nucleons to decide on the position of a nucleon which has to   be initialized.\item This prescription only works properly if the smearing width is really   small. Therefore it is switched off by default.\end{itemize}\end{minipage}\\*
\midrule
improvedMC\_speedup & \begin{minipage}[t]{2cm}integer\end{minipage} & \begin{minipage}[t]{2cm}500\end{minipage} & \begin{minipage}[t]{12cm}\begin{itemize}\leftmargin0em\itemindent0pt\item If improvedMC  is set to .true. then this variable defines   the speedup of the algorithm.\item The number defines how often the density field is updated.\item A large value of  this parameter yields a less accurate test-particle   distribution and a faster initialization.\end{itemize}\end{minipage}\\*
\midrule
HiTail & \begin{minipage}[t]{2cm}logical\end{minipage} & \begin{minipage}[t]{2cm}.false.\end{minipage} & \begin{minipage}[t]{12cm}If HiTail is set to .true., then a simple parametrization of n(p) is used to initialize the nucleon momenta (cf. function chooseAbsMomentum for details).\end{minipage}\\*
\midrule
determine\_Fermi\_momentum\_by\_binding\_energy & \begin{minipage}[t]{2cm}logical\end{minipage} & \begin{minipage}[t]{2cm}.false.\end{minipage} & \begin{minipage}[t]{12cm}If set to .true., the Fermi momentum will determined by E\_B=p\_f**2/(2m)+U(rho,p\_F), where E\_B is the binding energy per nucleon.\end{minipage}\\*
\midrule
determine\_Fermi\_new\_NucDLDA & \begin{minipage}[t]{2cm}logical\end{minipage} & \begin{minipage}[t]{2cm}.false.\end{minipage} & \begin{minipage}[t]{12cm}If set to .true., the Fermi momentum will be set to a value such that there are no unbound nucleons at the initialisation.\end{minipage}\\*
\midrule
useEnergySF & \begin{minipage}[t]{2cm}logical\end{minipage} & \begin{minipage}[t]{2cm}.false.\end{minipage} & \begin{minipage}[t]{12cm}If set to .true., then a spectral function is used to choose the energy.\end{minipage}\\*
\midrule
compressedFlag & \begin{minipage}[t]{2cm}logical\end{minipage} & \begin{minipage}[t]{2cm}.false.\end{minipage} & \begin{minipage}[t]{12cm}If set to .true., then a spherically deformed nucleus is initialized (isotropic compression/expansion; protons  and  neutrons in phase). This type of deformation corresponds to a giant-monopol resonance mode.\end{minipage}\\*
\midrule
ScaleFactor & \begin{minipage}[t]{2cm}real\end{minipage} & \begin{minipage}[t]{2cm}1.\end{minipage} & \begin{minipage}[t]{12cm}If compressedFlag=.true., then rescale coordinates by ScaleFactor.\end{minipage}\\*
\midrule
useCdA & \begin{minipage}[t]{2cm}logical\end{minipage} & \begin{minipage}[t]{2cm}.false.\end{minipage} & \begin{minipage}[t]{12cm}Instead of the usual momentum distribution according a fermi gas, use the momentum parametrizations as given in:\begin{itemize}\leftmargin0em\itemindent0pt\item C. Ciofi degli Ati, S. Simula, PRC 53, 1689 (1996)\end{itemize} These exist only for 2H,3He,4He,12C,160 40Ca,56Fe,208Pb\end{minipage}\\*
\midrule
zeroNucleusMomentum & \begin{minipage}[t]{2cm}logical\end{minipage} & \begin{minipage}[t]{2cm}.true.\end{minipage} & \begin{minipage}[t]{12cm}Indicate whether a procedure should be called to try to find a momentum init where the sum of all nucleon momenta (per ensemble) is zero (or at least close to zero).\\ At the moment, only a hill climbing algorithm is available, which changes the directions of the momenta randomly. The resulting averaged nucleus momentum is in the order of 10 MeV.\\ Without that, the average nucleus momentum goes \~{}0.17GeV*sqrt(A).\\ (Applies only for A$>$2.)\end{minipage}\\*
\bottomrule
\end{longtable}
{ }



% initPauli     code/density/pauliBlocking.f90

\begin{longtable}{llll}
\toprule
\textbf{\large{initPauli}} & \multicolumn{3}{l}{\footnotesize{code/density/pauliBlocking.f90}}\\*
\midrule
\endfirsthead
\midrule
\endhead
pauliSwitch & \begin{minipage}[t]{2cm}integer\end{minipage} & \begin{minipage}[t]{2cm}1\end{minipage} & \begin{minipage}[t]{12cm}\begin{itemize}\leftmargin0em\itemindent0pt\item 0 : No Pauli blocking\item 1 : dynamic Pauli blocking (use actual phase space densities)\item 2 : analytic Pauli blocking (use ground state assumption)   (not possible for Heavy Ions!)\item 3 : dynamic Pauli blocking in a box\end{itemize}\end{minipage}\\*
\midrule
densDepMomCutFlag & \begin{minipage}[t]{2cm}logical\end{minipage} & \begin{minipage}[t]{2cm}.false.\end{minipage} & \begin{minipage}[t]{12cm}if .true. - the radius in momentum space for selecting nucleons around given nucleon will depend on local Fermi momentum\\NOTES\\ Used only for dynamic pauli blocking.\end{minipage}\\*
\midrule
Gauss & \begin{minipage}[t]{2cm}real\end{minipage} & \begin{minipage}[t]{2cm}1.0\end{minipage} & \begin{minipage}[t]{12cm}Smearing for dynamic pauli blocking\end{minipage}\\*
\midrule
cutGauss & \begin{minipage}[t]{2cm}real\end{minipage} & \begin{minipage}[t]{2cm}2.2\end{minipage} & \begin{minipage}[t]{12cm}Cutoff for gauss Smearing\end{minipage}\\*
\midrule
cutMom & \begin{minipage}[t]{2cm}real\end{minipage} & \begin{minipage}[t]{2cm}0.08\end{minipage} & \begin{minipage}[t]{12cm}\begin{itemize}\leftmargin0em\itemindent0pt\item for densDepMomCutFlag=.false. ---   radius of phase space box in momentum space\item for densDepMomCutFlag=.true. ---   minimum radius of phase space box in momentum space\end{itemize}\end{minipage}\\*
\midrule
cutPos & \begin{minipage}[t]{2cm}real\end{minipage} & \begin{minipage}[t]{2cm}1.86\end{minipage} & \begin{minipage}[t]{12cm}Radius of phase space box in position space\end{minipage}\\*
\midrule
nGridPos & \begin{minipage}[t]{2cm}integer\end{minipage} & \begin{minipage}[t]{2cm}30\end{minipage} & \begin{minipage}[t]{12cm}number of points in position space to save weights on\end{minipage}\\*
\midrule
ensembleJump & \begin{minipage}[t]{2cm}integer\end{minipage} & \begin{minipage}[t]{2cm}5\end{minipage} & \begin{minipage}[t]{12cm}Parameter for speedup. Only every "ensemblejump"th ensemble is considered to evaluate the probability for pauli blocking.\end{minipage}\\*
\midrule
DoHistogram & \begin{minipage}[t]{2cm}logical\end{minipage} & \begin{minipage}[t]{2cm}.false.\end{minipage} & \begin{minipage}[t]{12cm}if .true., a historgram is filled representing the blocking probability as function of the fermi momentum. You have to call 'WriteBlockMom' explicitely for writing the histogram\end{minipage}\\*
\bottomrule
\end{longtable}
{ }



% initRandom     code/numerics/random.f90

\begin{longtable}{llll}
\toprule
\textbf{\large{initRandom}} & \multicolumn{3}{l}{\footnotesize{code/numerics/random.f90}}\\*
\midrule
\endfirsthead
\midrule
\endhead
Seed & \begin{minipage}[t]{2cm}integer\end{minipage} & \begin{minipage}[t]{2cm}0\end{minipage} & \begin{minipage}[t]{12cm}Random Seed (used to initialize the random number generator), accessible through the namelist 'initRandom'. If Seed is zero (default), then it is set via "SYSTEM\_CLOCK()".\end{minipage}\\*
\midrule
resetRandom & \begin{minipage}[t]{2cm}logical\end{minipage} & \begin{minipage}[t]{2cm}.false.\end{minipage} & \begin{minipage}[t]{12cm}Reread random generator, used by setRandom, useful for debugging.\end{minipage}\\*
\bottomrule
\end{longtable}
{ }



% initThermoDynamics     code/density/thermoDyn.f90

\begin{longtable}{llll}
\toprule
\textbf{\large{initThermoDynamics}} & \multicolumn{3}{l}{\footnotesize{code/density/thermoDyn.f90}}\\*
\midrule
\endfirsthead
\midrule
\endhead
temperatureSwitch & \begin{minipage}[t]{2cm}integer\end{minipage} & \begin{minipage}[t]{2cm}1\end{minipage} & \begin{minipage}[t]{12cm}\begin{itemize}\leftmargin0em\itemindent0pt\item 1=groundstate calculations (T=0,mu=E\_F)\item 2=the full procedure\end{itemize}\end{minipage}\\*
\midrule
linearExtrapolation & \begin{minipage}[t]{2cm}logical\end{minipage} & \begin{minipage}[t]{2cm}.true.\end{minipage} & \begin{minipage}[t]{12cm}\begin{itemize}\leftmargin0em\itemindent0pt\item .true.= Use linear extrapolation for temperature between gridPoints\item .false.= Do not use it\end{itemize}\end{minipage}\\*
\bottomrule
\end{longtable}
{ }



% input     code/inputOutput/inputGeneral.f90

\begin{longtable}{llll}
\toprule
\textbf{\large{input}} & \multicolumn{3}{l}{\footnotesize{code/inputOutput/inputGeneral.f90}}\\*
\midrule
\endfirsthead
\midrule
\endhead
path\_To\_Input & \begin{minipage}[t]{2cm}character(1000)\end{minipage} & \begin{minipage}[t]{2cm}'../../buuinput'\end{minipage} & \begin{minipage}[t]{12cm}Path to input files. This switch needs to be set to the local path of the 'buuinput' directory, which contains various input files for GiBUU.\end{minipage}\\*
\midrule
numEnsembles & \begin{minipage}[t]{2cm}integer\end{minipage} & \begin{minipage}[t]{2cm}300\end{minipage} & \begin{minipage}[t]{12cm}Number of parallel ensembles\end{minipage}\\*
\midrule
eventtype & \begin{minipage}[t]{2cm}integer\end{minipage} & \begin{minipage}[t]{2cm}3\end{minipage} & \begin{minipage}[t]{12cm}Switch for the type of event\\ possible values: see module eventtypes\end{minipage}\\*
\midrule
fullEnsemble & \begin{minipage}[t]{2cm}logical\end{minipage} & \begin{minipage}[t]{2cm}.false.\end{minipage} & \begin{minipage}[t]{12cm}Switch for the type of simulation:\begin{itemize}\leftmargin0em\itemindent0pt\item .false.=parallel ensembles\item .true.=full ensemble\end{itemize} See also "localEnsemble".\end{minipage}\\*
\midrule
localEnsemble & \begin{minipage}[t]{2cm}logical\end{minipage} & \begin{minipage}[t]{2cm}.false.\end{minipage} & \begin{minipage}[t]{12cm}Switch for the type of simulation:\begin{itemize}\leftmargin0em\itemindent0pt\item .false. = parallel or full ensembles (depending on the value of the fullEnsemble switch).\item .true. = fullEnsemble with "local collisionCriteria", see Lang/Babovsky et al., J. Comput. Phys. 106 (1993) 391-396.\end{itemize} Setting localEnsemble = .true. will implicitly set fullEnsemble = .true. (disregarding its value in the jobcard).\end{minipage}\\*
\midrule
delta\_T & \begin{minipage}[t]{2cm}real\end{minipage} & \begin{minipage}[t]{2cm}0.2\end{minipage} & \begin{minipage}[t]{12cm}time difference for time stepping\end{minipage}\\*
\midrule
numTimeSteps & \begin{minipage}[t]{2cm}integer\end{minipage} & \begin{minipage}[t]{2cm}100\end{minipage} & \begin{minipage}[t]{12cm}Number of time steps\end{minipage}\\*
\midrule
variableTimeStep & \begin{minipage}[t]{2cm}logical\end{minipage} & \begin{minipage}[t]{2cm}.false.\end{minipage} & \begin{minipage}[t]{12cm}Switch for using of variable time step:\begin{itemize}\leftmargin0em\itemindent0pt\item .false.= use constant time step delta\_T (see above).\item .true.= use time step computed from the frequency of collisions.   In this case the input delta\_T is used as the maximum   allowed time step.\end{itemize}\end{minipage}\\*
\midrule
time\_max & \begin{minipage}[t]{2cm}real\end{minipage} & \begin{minipage}[t]{2cm}30.\end{minipage} & \begin{minipage}[t]{12cm}Maximum time until which the time evolution will be computed in the case of variableTimeStep = .true.\end{minipage}\\*
\midrule
num\_energies & \begin{minipage}[t]{2cm}integer\end{minipage} & \begin{minipage}[t]{2cm}1\end{minipage} & \begin{minipage}[t]{12cm}Number of different energies for energy scans\end{minipage}\\*
\midrule
num\_runs\_sameEnergy & \begin{minipage}[t]{2cm}integer\end{minipage} & \begin{minipage}[t]{2cm}1\end{minipage} & \begin{minipage}[t]{12cm}Number of runs with the same energy in the initialization.\end{minipage}\\*
\midrule
checkGridSize\_Flag & \begin{minipage}[t]{2cm}logical\end{minipage} & \begin{minipage}[t]{2cm}.false.\end{minipage} & \begin{minipage}[t]{12cm}Switch for checking if particles escape out of grid. possible values:\begin{itemize}\leftmargin0em\itemindent0pt\item .false.= no check.\item .true. = check is performed, and a warning flag is printed out,   in case that particles are outside of the grid.\item check valid only for real particles.\end{itemize}\end{minipage}\\*
\midrule
continousBoundaries & \begin{minipage}[t]{2cm}logical\end{minipage} & \begin{minipage}[t]{2cm}.false.\end{minipage} & \begin{minipage}[t]{12cm}\begin{itemize}\leftmargin0em\itemindent0pt\item Switch to turn on continous boundary conditions.\item Implications for density and propagation.\item This means that particles are propagated according to continous   boundaries. A particle leaving the grid will move back in from the   opposite side. The densities are carefully constructed such that places   at the opposite side contribute to places on the near side.\item What is still missing is the full implementation in collision criteria,   this is not done yet for the two body collisions!   Be careful therefore with the 2-Body-collisions at the edges.   A particle at one edge does not see its scattering partner at the   opposite edge.\end{itemize}\end{minipage}\\*
\midrule
FinalCoulombCorrection & \begin{minipage}[t]{2cm}logical\end{minipage} & \begin{minipage}[t]{2cm}.false.\end{minipage} & \begin{minipage}[t]{12cm}Switch for Coulomb correction at the end of each run of the outgoing particles\end{minipage}\\*
\midrule
length\_perturbative & \begin{minipage}[t]{2cm}integer\end{minipage} & \begin{minipage}[t]{2cm}-1\end{minipage} & \begin{minipage}[t]{12cm}Length of perturbative particle vector (per ensemble). If negative, it will be determined automatically by event type.\end{minipage}\\*
\midrule
length\_real & \begin{minipage}[t]{2cm}integer\end{minipage} & \begin{minipage}[t]{2cm}-1\end{minipage} & \begin{minipage}[t]{12cm}Length of real particle vector (per ensemble). If negative, it will be determined automatically by event type.\end{minipage}\\*
\midrule
freezeRealParticles & \begin{minipage}[t]{2cm}logical\end{minipage} & \begin{minipage}[t]{2cm}.false.\end{minipage} & \begin{minipage}[t]{12cm}Switch for not propagating real particles\end{minipage}\\*
\midrule
printParticleVectors & \begin{minipage}[t]{2cm}logical\end{minipage} & \begin{minipage}[t]{2cm}.false.\end{minipage} & \begin{minipage}[t]{12cm}Switch to turn on the printing of the particle vector at the start and end of a run.\end{minipage}\\*
\midrule
printParticleVectorTime & \begin{minipage}[t]{2cm}logical\end{minipage} & \begin{minipage}[t]{2cm}.false.\end{minipage} & \begin{minipage}[t]{12cm}\begin{itemize}\leftmargin0em\itemindent0pt\item Switch to turn on the printing of the particle vector   as function of time.\item Useful for event classes using real particles (HeavyIon,Hadron).\item See also 'timeForOutput' and 'timeSequence'.\end{itemize}\end{minipage}\\*
\midrule
printParticleVectorsFormat & \begin{minipage}[t]{2cm}integer\end{minipage} & \begin{minipage}[t]{2cm}1\end{minipage} & \begin{minipage}[t]{12cm}Select the format for printing the particle vectors. Possible values are:\begin{itemize}\leftmargin0em\itemindent0pt\item 1: ASCII\item 2: binary\end{itemize}\end{minipage}\\*
\midrule
timeForOutput & \begin{minipage}[t]{2cm}real\end{minipage} & \begin{minipage}[t]{2cm}50.\end{minipage} & \begin{minipage}[t]{12cm}\begin{itemize}\leftmargin0em\itemindent0pt\item Time (fm/c) after which the particle vector   is printed during run (see also variable "timeSequence").\item valid only if printParticleVectorTime = .true.\end{itemize}\end{minipage}\\*
\midrule
timeSequence & \begin{minipage}[t]{2cm}real\end{minipage} & \begin{minipage}[t]{2cm}10.\end{minipage} & \begin{minipage}[t]{12cm}\begin{itemize}\leftmargin0em\itemindent0pt\item Time sequence (fm/c) of time dependent printing of the particle vector\item valid only if printParticleVectorTime = .true.\end{itemize}\end{minipage}\\*
\midrule
DoPrLevel & \begin{minipage}[t]{2cm}\end{minipage} & \begin{minipage}[t]{2cm}\end{minipage} & \begin{minipage}[t]{12cm}\end{minipage}\\*
\midrule
povray\_switch & \begin{minipage}[t]{2cm}logical\end{minipage} & \begin{minipage}[t]{2cm}.false.\end{minipage} & \begin{minipage}[t]{12cm}Switch for generating Povray-Output\end{minipage}\\*
\midrule
LRF\_equals\_CALC\_frame & \begin{minipage}[t]{2cm}logical\end{minipage} & \begin{minipage}[t]{2cm}.false.\end{minipage} & \begin{minipage}[t]{12cm}\begin{itemize}\leftmargin0em\itemindent0pt\item Switch to turn on the assumption that calculation frame and LRF frame coincide\item Only useful for reactions close to ground state !!!\end{itemize}\end{minipage}\\*
\midrule
DoFragmentNucleons & \begin{minipage}[t]{2cm}logical\end{minipage} & \begin{minipage}[t]{2cm}.false.\end{minipage} & \begin{minipage}[t]{12cm}\begin{itemize}\leftmargin0em\itemindent0pt\item Switch to turn on/off adding of nucleons stemming from fragmentation   of bound clusters.\end{itemize}\end{minipage}\\*
\midrule
PrintCollisionList & \begin{minipage}[t]{2cm}logical\end{minipage} & \begin{minipage}[t]{2cm}.false.\end{minipage} & \begin{minipage}[t]{12cm}Switch to turn on the printing of the particles and the collisions continously during the run\end{minipage}\\*
\midrule
onlyFirstEvent & \begin{minipage}[t]{2cm}logical\end{minipage} & \begin{minipage}[t]{2cm}.false.\end{minipage} & \begin{minipage}[t]{12cm}Switch to turn off all final state colisions by only allowing particles to suffer a 2-body od 3-body collisions, if their field '\%firstEvent=0'\end{minipage}\\*
\midrule
version & \begin{minipage}[t]{2cm}integer, private\end{minipage} & \begin{minipage}[t]{2cm}-1\end{minipage} & \begin{minipage}[t]{12cm}Indicator, for which code version this jobcard is suitable.\\ possible value (at the moment):\begin{itemize}\leftmargin0em\itemindent0pt\item 2023: for the use with the release version v2023\end{itemize} if not given, or wrong value, the code will stop execution\end{minipage}\\*
\bottomrule
\end{longtable}
{ }



% input_FF_Delta     code/init/lepton/formfactors_Delta/FF_Delta_production.f90

\begin{longtable}{llll}
\toprule
\textbf{\large{input\_FF\_Delta}} & \multicolumn{3}{l}{\footnotesize{code/init/lepton/formfactors\_Delta/FF\_Delta\_production.f90}}\\*
\midrule
\endfirsthead
\midrule
\endhead
FF\_Delta & \begin{minipage}[t]{2cm}integer\end{minipage} & \begin{minipage}[t]{2cm}1\end{minipage} & \begin{minipage}[t]{12cm}This switch decides whether the Paschos form factors (FF\_Delta=1) or the Maid form factors (FF\_Delta=0) are used. Default is FF\_Delta=1.\end{minipage}\\*
\bottomrule
\end{longtable}
{ }



% input_FF_ResProd     code/init/lepton/formfactors_ResProd/formFactor_ResProd.f90

\begin{longtable}{llll}
\toprule
\textbf{\large{input\_FF\_ResProd}} & \multicolumn{3}{l}{\footnotesize{code/init/lepton/formfactors\_ResProd/formFactor\_ResProd.f90}}\\*
\midrule
\endfirsthead
\midrule
\endhead
FF\_ResProd & \begin{minipage}[t]{2cm}integer\end{minipage} & \begin{minipage}[t]{2cm}0\end{minipage} & \begin{minipage}[t]{12cm}select how the form factors are calculated:\begin{itemize}\leftmargin0em\itemindent0pt\item 0: MAID's helicity amplitudes (Luis' helicity expressions - CM frame)\item 1: fit of Lalakulich (PRD 74, 014009 (2006))\item 2: MAID's helicity amplitudes (Lalakulich's helicity expressions - LAB frame)\end{itemize}\end{minipage}\\*
\midrule
aDelta & \begin{minipage}[t]{2cm}real\end{minipage} & \begin{minipage}[t]{2cm}0\end{minipage} & \begin{minipage}[t]{12cm}fit parameter for C\_5\^{}A (Adler)\end{minipage}\\*
\midrule
bDelta & \begin{minipage}[t]{2cm}real\end{minipage} & \begin{minipage}[t]{2cm}0\end{minipage} & \begin{minipage}[t]{12cm}fit parameter for C\_5\^{}A (Adler)\end{minipage}\\*
\midrule
cDelta & \begin{minipage}[t]{2cm}real\end{minipage} & \begin{minipage}[t]{2cm}3.\end{minipage} & \begin{minipage}[t]{12cm}fit parameter for C\_5\^{}A (Paschos)\end{minipage}\\*
\midrule
C5A0corr & \begin{minipage}[t]{2cm}real\end{minipage} & \begin{minipage}[t]{2cm}0.85\end{minipage} & \begin{minipage}[t]{12cm}fit parameter for C\_5\^{}A (Adler), adjusts strength of C5A0 for the Delta\end{minipage}\\*
\midrule
DeltaAxFF & \begin{minipage}[t]{2cm}integer\end{minipage} & \begin{minipage}[t]{2cm}1\end{minipage} & \begin{minipage}[t]{12cm}choose between different axial form factors for the Delta:\begin{itemize}\leftmargin0em\itemindent0pt\item 1: Adler\item 2: Paschos\item 3: dipol\end{itemize}\end{minipage}\\*
\midrule
HNV\_axialFF & \begin{minipage}[t]{2cm}logical\end{minipage} & \begin{minipage}[t]{2cm}.false.\end{minipage} & \begin{minipage}[t]{12cm}choose axial form factors for the Delta:\begin{itemize}\leftmargin0em\itemindent0pt\item .true. is Hernandez-Nieves-Valverde fit with C5A=0.867,MA=0.985 (PRD 76)\item .false. is as it was used by Lalakulich et al in PRD 74\end{itemize}\end{minipage}\\*
\midrule
nenner\_C5A\_Lalakulich & \begin{minipage}[t]{2cm}real\end{minipage} & \begin{minipage}[t]{2cm}3.0\end{minipage} & \begin{minipage}[t]{12cm}Factor which appears in the Lalakulich parameterization of the axial C\_5\^{}A form factor:\begin{itemize}\leftmargin0em\itemindent0pt\item 3.0 was fitted to BNL and used in Lalakulich PRD71 and PRD 74\item 0.5 is given by a fit of ANL\end{itemize}\end{minipage}\\*
\midrule
refit\_barnu\_axialFF & \begin{minipage}[t]{2cm}logical\end{minipage} & \begin{minipage}[t]{2cm}.false.\end{minipage} & \begin{minipage}[t]{12cm}if .true., axial form factors are refitted to explain the low value of antineutrino cross section (exp. data  Bolognese PLB81,393 (1979))\end{minipage}\\*
\midrule
W\_cutOff\_lambda & \begin{minipage}[t]{2cm}real\end{minipage} & \begin{minipage}[t]{2cm}1.071\end{minipage} & \begin{minipage}[t]{12cm}Value for lambda in the W-dependent cut-off function.\end{minipage}\\*
\midrule
W\_cutOff\_switch & \begin{minipage}[t]{2cm}logical\end{minipage} & \begin{minipage}[t]{2cm}.false.\end{minipage} & \begin{minipage}[t]{12cm}Switch to include a W-dependent cut-off function for the vector form factor of the Delta\end{minipage}\\*
\midrule
vector\_FF\_switch & \begin{minipage}[t]{2cm}logical\end{minipage} & \begin{minipage}[t]{2cm}.true.\end{minipage} & \begin{minipage}[t]{12cm}Switch to turn off the vector form factors\end{minipage}\\*
\midrule
axial\_FF\_switch & \begin{minipage}[t]{2cm}logical\end{minipage} & \begin{minipage}[t]{2cm}.true.\end{minipage} & \begin{minipage}[t]{12cm}Switch to turn off the axial form factors\end{minipage}\\*
\midrule
W\_cutOff\_switchAll & \begin{minipage}[t]{2cm}logical\end{minipage} & \begin{minipage}[t]{2cm}.false.\end{minipage} & \begin{minipage}[t]{12cm}Switch to include a W-dependent cut-off function for the vector and the axial form factor of all resonances\\NOTES\\ we assume the same dependence as for the Delta vector form factor\end{minipage}\\*
\midrule
DeltaCouplrelErr & \begin{minipage}[t]{2cm}real\end{minipage} & \begin{minipage}[t]{2cm}0.\end{minipage} & \begin{minipage}[t]{12cm}error in percent for C\_5\^{}A(0) for the Delta\end{minipage}\\*
\midrule
MA & \begin{minipage}[t]{2cm}real\end{minipage} & \begin{minipage}[t]{2cm}1.05\end{minipage} & \begin{minipage}[t]{12cm}Delta resonance axial mass parameter. Wilkinson et al have shown in Phys.Rev.D 90 (2014) that the ANL pion production data are more reliable than the BNL ones.\end{minipage}\\*
\bottomrule
\end{longtable}
{ }



% insertion     code/collisions/insertion.f90

\begin{longtable}{llll}
\toprule
\textbf{\large{insertion}} & \multicolumn{3}{l}{\footnotesize{code/collisions/insertion.f90}}\\*
\midrule
\endfirsthead
\midrule
\endhead
minimumEnergy & \begin{minipage}[t]{2cm}real\end{minipage} & \begin{minipage}[t]{2cm}0.005\end{minipage} & \begin{minipage}[t]{12cm}Minimal kinetic energy in GeV for produced perturbative nucleons. If their energy is below this threshold, then they are not propagated, i.e. they are not inserted in the particle vector.\end{minipage}\\*
\midrule
propagateNoPhoton & \begin{minipage}[t]{2cm}logical\end{minipage} & \begin{minipage}[t]{2cm}.true.\end{minipage} & \begin{minipage}[t]{12cm}If .true. then we eliminate all photons, such that they are not propagated and do not show up in the particle vector. If .false. then photons are explicitly propagated.\end{minipage}\\*
\bottomrule
\end{longtable}
{ }



% Lepton2p2h     code/init/neutrino/neutrinoParms.f90

\begin{longtable}{llll}
\toprule
\textbf{\large{Lepton2p2h}} & \multicolumn{3}{l}{\footnotesize{code/init/neutrino/neutrinoParms.f90}}\\*
\midrule
\endfirsthead
\midrule
\endhead
ME\_Version & \begin{minipage}[t]{2cm}integer\end{minipage} & \begin{minipage}[t]{2cm}6\end{minipage} & \begin{minipage}[t]{12cm}indicate the type of matrix element parametrisation\\NOTES\\ possible values:\begin{itemize}\leftmargin0em\itemindent0pt\item 1: const ME\_Norm\_XX  ! const for CC  fitted to MiniBooNE is 1.8e-6\item 2: constant transverse and decreasing with Enu\item 3: "Dipole transverse" transverse, fall with Q2 as 4-th power\item 4: MEC from E. Christy (8/2015), with parametrization for longitudinal\item 5: MEC from Bosted arXiV:1203.2262, with parametrization for longitudinal\item 6: MEC additional parametrization, with parametrization for longitudinal   not yet implemented\end{itemize} remarks:\begin{itemize}\leftmargin0em\itemindent0pt\item case 1 is model-I in Lalakulich,Gallmeister,Mosel PRC86(2012)014614\item case 2 is model-II from Lalakulich,Gallmeister,Mosel PRC86(2012)014614\item case 3 gives a good description of MiniBooNE data with MA \~{} 1.5 GeV\end{itemize}\end{minipage}\\*
\midrule
ME\_Norm\_QE & \begin{minipage}[t]{2cm}real, dimension(1:3)\end{minipage} & \begin{minipage}[t]{2cm}(/1.0, 1.0, 1.0/)\end{minipage} & \begin{minipage}[t]{12cm}Overall strength of 2p2h matrix element with 2N out\\ for (EM,CC,NC)\\NOTES\\ The value == 1 gives the coded strength\end{minipage}\\*
\midrule
ME\_Norm\_Delta & \begin{minipage}[t]{2cm}real, dimension(1:3)\end{minipage} & \begin{minipage}[t]{2cm}(/1.0, 1.0, 1.0/)\end{minipage} & \begin{minipage}[t]{12cm}Overall strength of 2p2h matrix element with NDelta out\\ for (EM,CC,NC)\\NOTES\\ The value == 1 is a dummy value\end{minipage}\\*
\midrule
ME\_Mass\_QE & \begin{minipage}[t]{2cm}real, dimension(1:3)\end{minipage} & \begin{minipage}[t]{2cm}(/1.0, 1.0, 1.0/)\end{minipage} & \begin{minipage}[t]{12cm}Cutoff-mass in some parametrizations of 2p2h matrix element for NN out\\ for (EM,CC,NC)\\NOTES\\ The value == 1 is a dummy value\end{minipage}\\*
\midrule
ME\_Mass\_Delta & \begin{minipage}[t]{2cm}real, dimension(1:3)\end{minipage} & \begin{minipage}[t]{2cm}(/1.0, 1.0, 1.0/)\end{minipage} & \begin{minipage}[t]{12cm}Cutoff-mass in some parametrizations of matrix element for NDelta out\\ for (EM,CC,NC)\\NOTES\\ The value == 1 is a dummy value\end{minipage}\\*
\midrule
ME\_Transversity & \begin{minipage}[t]{2cm}real, dimension(1:3)\end{minipage} & \begin{minipage}[t]{2cm}(/1.0, 1.0, 1.0/)\end{minipage} & \begin{minipage}[t]{12cm}Parametrisation of structure functions\\ for (EM,CC,NC)\\NOTES\\ The value = 1 chooses structure function W2 so that 2p2h is pure transverse\end{minipage}\\*
\midrule
ME\_LONG & \begin{minipage}[t]{2cm}real, dimension(1:3)\end{minipage} & \begin{minipage}[t]{2cm}(/0.0, 0.0, 0.0/)\end{minipage} & \begin{minipage}[t]{12cm}Parametrization of structure functions\\ for (EM,CC,NC)\\NOTES\\ The value = 0 turns any additional longitudinal contribution to structure funct. W2 off\end{minipage}\\*
\midrule
ME\_W3 & \begin{minipage}[t]{2cm}real, dimension(1:3)\end{minipage} & \begin{minipage}[t]{2cm}(/0.0, 1.0, 1.0/)\end{minipage} & \begin{minipage}[t]{12cm}Overall strength factor for structure function W3\\ only for (CC,NC)\\NOTES\\ overall strength parameter for structure function W3\end{minipage}\\*
\midrule
inmedW & \begin{minipage}[t]{2cm}integer\end{minipage} & \begin{minipage}[t]{2cm}1\end{minipage} & \begin{minipage}[t]{12cm}Controls which inv mass W is used in parametrization of 2p2h W1\\NOTES\begin{itemize}\leftmargin0em\itemindent0pt\item 1: W = static inv. mass in 2p2h parametrization of W1\item 2: W = inv mass for Fermi moving nucleons in potential\item 3: W = inv mass for Fermi moving nucleons without potential\end{itemize}\end{minipage}\\*
\midrule
Adep & \begin{minipage}[t]{2cm}integer\end{minipage} & \begin{minipage}[t]{2cm}2\end{minipage} & \begin{minipage}[t]{12cm}Switch for A-dependence of 2p2h structure function\\NOTES\begin{itemize}\leftmargin0em\itemindent0pt\item 1: A-dependence for zero-range force (Mosel, Gallmeister, 2016)\item 2: linear A-dependence, normalized to C12\end{itemize}\end{minipage}\\*
\bottomrule
\end{longtable}
{ }



% lepton_bin     code/init/neutrino/initNeutrino.f90

\begin{longtable}{llll}
\toprule
\textbf{\large{lepton\_bin}} & \multicolumn{3}{l}{\footnotesize{code/init/neutrino/initNeutrino.f90}}\\*
\midrule
\endfirsthead
\midrule
\endhead
cost\_min & \begin{minipage}[t]{2cm}real\end{minipage} & \begin{minipage}[t]{2cm}-1.0\end{minipage} & \begin{minipage}[t]{12cm}if detailed\_diff\_output is TRUE: Minimal cos(theta) of outgoing leptons, used in 2D dsigma/dEdcos(theta) This cut affects *only* the outgoing lepton\end{minipage}\\*
\midrule
cost\_max & \begin{minipage}[t]{2cm}real\end{minipage} & \begin{minipage}[t]{2cm}+1.0\end{minipage} & \begin{minipage}[t]{12cm}if detailed\_diff\_output is TRUE: Maximal cos(theta) of outgoing leptons, used in 2D dsigma/dEdcos(theta) This cut affects *only* the outgoing lepton\end{minipage}\\*
\midrule
delta\_cost & \begin{minipage}[t]{2cm}real\end{minipage} & \begin{minipage}[t]{2cm}0.1\end{minipage} & \begin{minipage}[t]{12cm}if detailed\_diff\_output is TRUE: stepsize of cos(theta) of outgoing leptons, used in 2D dsigma/dEdcos(theta)\end{minipage}\\*
\midrule
Elept\_min & \begin{minipage}[t]{2cm}real\end{minipage} & \begin{minipage}[t]{2cm}0.0\end{minipage} & \begin{minipage}[t]{12cm}if detailed\_diff\_output is TRUE: minimal energy of outgoing leptons, used in 2D dsigma/dEdcos(theta)\end{minipage}\\*
\midrule
Elept\_max & \begin{minipage}[t]{2cm}real\end{minipage} & \begin{minipage}[t]{2cm}2.0\end{minipage} & \begin{minipage}[t]{12cm}if detailed\_diff\_output or printAbsorption are TRUE: maximal energy of outgoing leptons, used in 2D dsigma/dEdcos(theta)\end{minipage}\\*
\midrule
delta\_Elept & \begin{minipage}[t]{2cm}real\end{minipage} & \begin{minipage}[t]{2cm}0.01\end{minipage} & \begin{minipage}[t]{12cm}if detailed\_diff\_output or printAbsorption are TRUE: stepsize of energy of outgoing leptons, used in 2D dsigma/dEdcos(theta)\end{minipage}\\*
\midrule
pL\_min & \begin{minipage}[t]{2cm}real\end{minipage} & \begin{minipage}[t]{2cm}0.0\end{minipage} & \begin{minipage}[t]{12cm}if detailed\_diff or printAbsorption are TRUE: minimal long. momentum  of outgoing leptons, used in 2D dsigma/dpLdpT\end{minipage}\\*
\midrule
pL\_max & \begin{minipage}[t]{2cm}real\end{minipage} & \begin{minipage}[t]{2cm}20.0\end{minipage} & \begin{minipage}[t]{12cm}if detailed\_diff\_output or printAbsorption are TRUE: maximal long. momentum  of outgoing leptons, used in 2D dsigma/dpLdpT\end{minipage}\\*
\midrule
delta\_pL & \begin{minipage}[t]{2cm}real\end{minipage} & \begin{minipage}[t]{2cm}0.25\end{minipage} & \begin{minipage}[t]{12cm}if detailed\_diff\_output or printAbsorption are TRUE: stepsize of long. momentum of outgoing leptons, used in 2D dsigma/dpLdpT\end{minipage}\\*
\midrule
pT\_min & \begin{minipage}[t]{2cm}real\end{minipage} & \begin{minipage}[t]{2cm}0.0\end{minipage} & \begin{minipage}[t]{12cm}if detailed\_diff or printAbsorption are TRUE: minimal transv. momentum  of outgoing leptons, used in 2D dsigma/dpLdpT\end{minipage}\\*
\midrule
pT\_max & \begin{minipage}[t]{2cm}real\end{minipage} & \begin{minipage}[t]{2cm}2.5\end{minipage} & \begin{minipage}[t]{12cm}if detailed\_diff\_output or printAbsorption are TRUE: maximal transv. momentum  of outgoing leptons, used in 2D dsigma/dpLdpT\end{minipage}\\*
\midrule
delta\_pT & \begin{minipage}[t]{2cm}real\end{minipage} & \begin{minipage}[t]{2cm}0.1\end{minipage} & \begin{minipage}[t]{12cm}if detailed\_diff\_output or printAbsorption are TRUE: binwidth of transv. momentum of outgoing leptons, used in 2D dsigma/dpLdpT\end{minipage}\\*
\midrule
Q2\_Max & \begin{minipage}[t]{2cm}real\end{minipage} & \begin{minipage}[t]{2cm}100.\end{minipage} & \begin{minipage}[t]{12cm}maximal value of Q2 in Q2-distribution\end{minipage}\\*
\bottomrule
\end{longtable}
{ }



% low_photo_induced     code/init/lowPhoton/initLowPhoton.f90

\begin{longtable}{llll}
\toprule
\textbf{\large{low\_photo\_induced}} & \multicolumn{3}{l}{\footnotesize{code/init/lowPhoton/initLowPhoton.f90}}\\*
\midrule
\endfirsthead
\midrule
\endhead
energy\_gamma & \begin{minipage}[t]{2cm}real\end{minipage} & \begin{minipage}[t]{2cm}0.\end{minipage} & \begin{minipage}[t]{12cm}Energy of incoming photon in nucleus rest frame (in GeV).\end{minipage}\\*
\midrule
delta\_energy & \begin{minipage}[t]{2cm}real\end{minipage} & \begin{minipage}[t]{2cm}0.\end{minipage} & \begin{minipage}[t]{12cm}Increase of energy for energy scans.\end{minipage}\\*
\midrule
energy\_weighting & \begin{minipage}[t]{2cm}integer\end{minipage} & \begin{minipage}[t]{2cm}0\end{minipage} & \begin{minipage}[t]{12cm}Determines the relative weight of different photon energies in energy scans Possible values:\begin{itemize}\leftmargin0em\itemindent0pt\item 0 = flat distribution (all energies are weighted equal)\item 1 = exponential distr. (energies are weighted \~{} 1/E)\end{itemize}\end{minipage}\\*
\bottomrule
\end{longtable}
{ }



% LowElectron     code/init/lowElectron/initLowElectron.f90

\begin{longtable}{llll}
\toprule
\textbf{\large{LowElectron}} & \multicolumn{3}{l}{\footnotesize{code/init/lowElectron/initLowElectron.f90}}\\*
\midrule
\endfirsthead
\midrule
\endhead
runType & \begin{minipage}[t]{2cm}integer\end{minipage} & \begin{minipage}[t]{2cm}1\end{minipage} & \begin{minipage}[t]{12cm}\begin{itemize}\leftmargin0em\itemindent0pt\item 1: we make runs at some fixed angle defined by initLowElectron/theta\_lf.\item 2: we make runs at some fixed QSquared defined by initLowElectron/QSquared\end{itemize}\end{minipage}\\*
\midrule
inputType & \begin{minipage}[t]{2cm}integer\end{minipage} & \begin{minipage}[t]{2cm}1\end{minipage} & \begin{minipage}[t]{12cm}Decides which set of variables is used to determine the final electron energy energy\_lf and the step size delta\_energy\_lf:\begin{itemize}\leftmargin0em\itemindent0pt\item 1: we use directly energy\_lf and delta\_energy\_lf as input\item 2: we use W\_min and W\_max as input. For this we assume the   nucleon to be at rest to calculate energy\_lf out of W.\item 3: we use energy\_lf\_min and energy\_lf\_max as input.\end{itemize}\end{minipage}\\*
\midrule
theta\_lf & \begin{minipage}[t]{2cm}real\end{minipage} & \begin{minipage}[t]{2cm}10.\end{minipage} & \begin{minipage}[t]{12cm}Theta scattering angle of outgoing electron with respect to the incoming one.\\ Only relevant of runType=1.\end{minipage}\\*
\midrule
phi\_lf & \begin{minipage}[t]{2cm}real\end{minipage} & \begin{minipage}[t]{2cm}-10.\end{minipage} & \begin{minipage}[t]{12cm}Phi scattering angle of outgoing electron with respect to the incoming one. If less than 0, then we do a Monte-Carlo-Integration over phi!\end{minipage}\\*
\midrule
energy\_li & \begin{minipage}[t]{2cm}real\end{minipage} & \begin{minipage}[t]{2cm}1.2\end{minipage} & \begin{minipage}[t]{12cm}Energy of incoming electron in GeV.\end{minipage}\\*
\midrule
energy\_lf & \begin{minipage}[t]{2cm}real\end{minipage} & \begin{minipage}[t]{2cm}0.8\end{minipage} & \begin{minipage}[t]{12cm}Energy of final state electron in GeV.\\ Only used if inputType=1\end{minipage}\\*
\midrule
energy\_lf\_min & \begin{minipage}[t]{2cm}real\end{minipage} & \begin{minipage}[t]{2cm}0.1\end{minipage} & \begin{minipage}[t]{12cm}Minimal energy\_lf\\ Only used if inputType=3\end{minipage}\\*
\midrule
energy\_lf\_max & \begin{minipage}[t]{2cm}real\end{minipage} & \begin{minipage}[t]{2cm}0.1\end{minipage} & \begin{minipage}[t]{12cm}Maximal energy\_lf\\ Only used if inputType=3\end{minipage}\\*
\midrule
delta\_energy\_lf & \begin{minipage}[t]{2cm}real\end{minipage} & \begin{minipage}[t]{2cm}0.8\end{minipage} & \begin{minipage}[t]{12cm}delta(Energy) of final state electron in GeV for energy scans.\\ Only used if inputType=1\end{minipage}\\*
\midrule
W\_min & \begin{minipage}[t]{2cm}real\end{minipage} & \begin{minipage}[t]{2cm}0.9\end{minipage} & \begin{minipage}[t]{12cm}Minimal W at the hadronic vertex assuming a resting nucleon\\ Only used if inputType=2\end{minipage}\\*
\midrule
W\_max & \begin{minipage}[t]{2cm}real\end{minipage} & \begin{minipage}[t]{2cm}1.9\end{minipage} & \begin{minipage}[t]{12cm}Maximal W at the hadronic vertex assuming a resting nucleon\\ Only used if inputType=2\end{minipage}\\*
\midrule
QSquared & \begin{minipage}[t]{2cm}real\end{minipage} & \begin{minipage}[t]{2cm}0.5\end{minipage} & \begin{minipage}[t]{12cm}QSquared of virtual photon.\\ Only relevant of runType=2.\end{minipage}\\*
\midrule
Do\_QE & \begin{minipage}[t]{2cm}logical\end{minipage} & \begin{minipage}[t]{2cm}.true.\end{minipage} & \begin{minipage}[t]{12cm}Switch for including or excluding Quasi-Elastic (QE) processes\end{minipage}\\*
\midrule
Do\_1Pi & \begin{minipage}[t]{2cm}logical\end{minipage} & \begin{minipage}[t]{2cm}.true.\end{minipage} & \begin{minipage}[t]{12cm}Switch for including or excluding direct Single pion production processes. If resonances are included (Do\_Res=.true.), then only the background part is included.\end{minipage}\\*
\midrule
Do\_Res & \begin{minipage}[t]{2cm}logical\end{minipage} & \begin{minipage}[t]{2cm}.true.\end{minipage} & \begin{minipage}[t]{12cm}Switch for including or excluding resonance production processes\end{minipage}\\*
\midrule
Do\_2Pi & \begin{minipage}[t]{2cm}logical\end{minipage} & \begin{minipage}[t]{2cm}.true.\end{minipage} & \begin{minipage}[t]{12cm}Switch for including or excluding direct Double pion production processes. If resonances are included (Do\_Res=.true.), then only the background part is included.\end{minipage}\\*
\midrule
Do\_DIS & \begin{minipage}[t]{2cm}logical\end{minipage} & \begin{minipage}[t]{2cm}.true.\end{minipage} & \begin{minipage}[t]{12cm}Switch for including or excluding deeply inelastic scattering (DIS) events\\ Only relevant for W $>$ 1.4-1.5 GeV.\end{minipage}\\*
\midrule
Do\_2p2hQE & \begin{minipage}[t]{2cm}logical\end{minipage} & \begin{minipage}[t]{2cm}.false.\end{minipage} & \begin{minipage}[t]{12cm}Switch for including or excluding event according gamma* N1 N2 $\rightarrow$ N1' N2'\end{minipage}\\*
\midrule
Do\_2p2hDelta & \begin{minipage}[t]{2cm}logical\end{minipage} & \begin{minipage}[t]{2cm}.false.\end{minipage} & \begin{minipage}[t]{12cm}Switch for including or excluding event according gamma* N1 N2 $\rightarrow$ N' Delta\end{minipage}\\*
\midrule
minMass\_QE & \begin{minipage}[t]{2cm}real\end{minipage} & \begin{minipage}[t]{2cm}0.3\end{minipage} & \begin{minipage}[t]{12cm}Minimal mass of a nucleon in QE event. Prevents super-luminous nucleons when embedded in a Skyrme potential.\end{minipage}\\*
\midrule
minEnergy\_1pi & \begin{minipage}[t]{2cm}real\end{minipage} & \begin{minipage}[t]{2cm}0.16\end{minipage} & \begin{minipage}[t]{12cm}Minimal q\_0 such that pion production processes are considered.\end{minipage}\\*
\midrule
onlyDelta & \begin{minipage}[t]{2cm}logical\end{minipage} & \begin{minipage}[t]{2cm}.false.\end{minipage} & \begin{minipage}[t]{12cm}Switch for including only delta resonance\end{minipage}\\*
\midrule
nuclearTarget\_corr & \begin{minipage}[t]{2cm}logical\end{minipage} & \begin{minipage}[t]{2cm}.true.\end{minipage} & \begin{minipage}[t]{12cm}\begin{itemize}\leftmargin0em\itemindent0pt\item If the input is a nuclear targer, then the target nucleus is at rest   and we calculate the cross section for   nuclear target: use flux with respect to the nucleus.\item Use .false. only for debugging.\end{itemize}\end{minipage}\\*
\bottomrule
\end{longtable}
{ }



% lowElePhoto_Analysis     code/analysis/lowElectronAnalysis.f90

\begin{longtable}{llll}
\toprule
\textbf{\large{lowElePhoto\_Analysis}} & \multicolumn{3}{l}{\footnotesize{code/analysis/lowElectronAnalysis.f90}}\\*
\midrule
\endfirsthead
\midrule
\endhead
dOmega\_switch & \begin{minipage}[t]{2cm}logical\end{minipage} & \begin{minipage}[t]{2cm}.false.\end{minipage} & \begin{minipage}[t]{12cm}If .true. then also dSigma/dOmega is produced, if false not..\end{minipage}\\*
\midrule
dE\_switch & \begin{minipage}[t]{2cm}logical\end{minipage} & \begin{minipage}[t]{2cm}.false.\end{minipage} & \begin{minipage}[t]{12cm}If .true. then also dSigma/dE is produced, if false not..\end{minipage}\\*
\bottomrule
\end{longtable}
{ }



% lowPhotonAnalysis     code/analysis/LowPhotonAnalysis.f90

\begin{longtable}{llll}
\toprule
\textbf{\large{lowPhotonAnalysis}} & \multicolumn{3}{l}{\footnotesize{code/analysis/LowPhotonAnalysis.f90}}\\*
\midrule
\endfirsthead
\midrule
\endhead
outputEvents & \begin{minipage}[t]{2cm}logical\end{minipage} & \begin{minipage}[t]{2cm}.false.\end{minipage} & \begin{minipage}[t]{12cm}If .true. then all events are printed to file.\end{minipage}\\*
\midrule
outputEvents\_onlyFree & \begin{minipage}[t]{2cm}logical\end{minipage} & \begin{minipage}[t]{2cm}.false.\end{minipage} & \begin{minipage}[t]{12cm}If outputEvents=.true. then only particles which may leave the nucleus, i.e. may become "free", are printed to file.\end{minipage}\\*
\midrule
KruscheOutput & \begin{minipage}[t]{2cm}logical\end{minipage} & \begin{minipage}[t]{2cm}.false.\end{minipage} & \begin{minipage}[t]{12cm}If .true. then we perform an analysis as in EPJA22 347-351 (2004)\end{minipage}\\*
\midrule
KruscheAnalyse\_cut & \begin{minipage}[t]{2cm}real\end{minipage} & \begin{minipage}[t]{2cm}0.\end{minipage} & \begin{minipage}[t]{12cm}Value of the cut for the deltaE cut in EPJA22 347-351 (2004).\end{minipage}\\*
\midrule
FissumOutput & \begin{minipage}[t]{2cm}logical\end{minipage} & \begin{minipage}[t]{2cm}.false.\end{minipage} & \begin{minipage}[t]{12cm}If .true. then we perform an analysis as in PRC 53,\#3 pages 1278 ff. (1996) Produces dsigma/dOmega/dT\_pi for pi\^{}+\end{minipage}\\*
\midrule
photonAnalyse & \begin{minipage}[t]{2cm}logical\end{minipage} & \begin{minipage}[t]{2cm}.false.\end{minipage} & \begin{minipage}[t]{12cm}Special analysis for final state photons\end{minipage}\\*
\midrule
TwoPiOutput & \begin{minipage}[t]{2cm}logical\end{minipage} & \begin{minipage}[t]{2cm}.false.\end{minipage} & \begin{minipage}[t]{12cm}If .true. then we perform an analysis for 2pi production, including statistics for the mass of the pi-pi pair.\end{minipage}\\*
\midrule
pi0gamma\_analysis & \begin{minipage}[t]{2cm}logical\end{minipage} & \begin{minipage}[t]{2cm}.false.\end{minipage} & \begin{minipage}[t]{12cm}Do analysis of pi0 gamma pairs (dsigma/dm), to reconstruct invariant mass spectrum of omega mesons.\end{minipage}\\*
\midrule
pi0gamma\_momcut & \begin{minipage}[t]{2cm}real\end{minipage} & \begin{minipage}[t]{2cm}0.5\end{minipage} & \begin{minipage}[t]{12cm}Cut on the absolute omega three momentum in GeV, being applied to the pi0 gamma spectrum.\end{minipage}\\*
\midrule
pi0gamma\_masscut & \begin{minipage}[t]{2cm}real, dimension(1:2)\end{minipage} & \begin{minipage}[t]{2cm}(/0.,2./)\end{minipage} & \begin{minipage}[t]{12cm}Cuts on the pi0-gamma invariant mass in GeV, being applied to all pi0-gamma spectra (except the mass spectrum). First component is lower limit, second component is upper limit.\end{minipage}\\*
\midrule
pi0gamma\_mombin & \begin{minipage}[t]{2cm}real\end{minipage} & \begin{minipage}[t]{2cm}0.050\end{minipage} & \begin{minipage}[t]{12cm}Bin size for the pi0 gamma momenentum spectrum in GeV.\end{minipage}\\*
\midrule
pi0gamma\_massres\_sigma & \begin{minipage}[t]{2cm}real\end{minipage} & \begin{minipage}[t]{2cm}0.025\end{minipage} & \begin{minipage}[t]{12cm}Sigma parameter for the exp. resolution smearing (width of the Gauss or Novosibirsk function in GeV). See also pi0gamma\_massres\_tau.\end{minipage}\\*
\midrule
pi0gamma\_massres\_tau & \begin{minipage}[t]{2cm}real\end{minipage} & \begin{minipage}[t]{2cm}-0.090\end{minipage} & \begin{minipage}[t]{12cm}Skewness parameter tau of the Novosibirsk function (for exp. resolution smearing). See also pi0gamma\_massres\_sigma.\end{minipage}\\*
\midrule
Ekin\_pi0\_cut & \begin{minipage}[t]{2cm}real\end{minipage} & \begin{minipage}[t]{2cm}0.\end{minipage} & \begin{minipage}[t]{12cm}Cut on the kinetic energy of neutral pions in the pi0gamma\_analysis. Only pions with kinetic energies larger than this cutoff are used for the analysis.\end{minipage}\\*
\midrule
eta\_analysis & \begin{minipage}[t]{2cm}logical\end{minipage} & \begin{minipage}[t]{2cm}.false.\end{minipage} & \begin{minipage}[t]{12cm}Do analysis of eta mesons\end{minipage}\\*
\midrule
MissingMass\_analysis & \begin{minipage}[t]{2cm}logical\end{minipage} & \begin{minipage}[t]{2cm}.false.\end{minipage} & \begin{minipage}[t]{12cm}Do analysis of missing mass distributions for 1 nucleon and 2 nucleon final states\end{minipage}\\*
\midrule
DoOutChannels & \begin{minipage}[t]{2cm}logical\end{minipage} & \begin{minipage}[t]{2cm}.false.\end{minipage} & \begin{minipage}[t]{12cm}switch on/off: reporting of all final state channels\end{minipage}\\*
\bottomrule
\end{longtable}
{ }



% MassAssInfo     code/typeDefinitions/MassAssInfoDefinition.f90

\begin{longtable}{llll}
\toprule
\textbf{\large{MassAssInfo}} & \multicolumn{3}{l}{\footnotesize{code/typeDefinitions/MassAssInfoDefinition.f90}}\\*
\midrule
\endfirsthead
\midrule
\endhead
UseMassAssInfo & \begin{minipage}[t]{2cm}logical\end{minipage} & \begin{minipage}[t]{2cm}.true.\end{minipage} & \begin{minipage}[t]{12cm}This switch indicates, whether we want to use the whole MassAssInfo machinery or stick to the old prescription of mass assignment.\\ You may set this switch via the jobcard. Anyhow, if your selection of switches for baryon and medium switches leads to cases which are not yet implemented, this flag is set to false automatically.\end{minipage}\\*
\bottomrule
\end{longtable}
{ }



% master_1body     code/collisions/oneBodyReactions/master_1Body.f90

\begin{longtable}{llll}
\toprule
\textbf{\large{master\_1body}} & \multicolumn{3}{l}{\footnotesize{code/collisions/oneBodyReactions/master\_1Body.f90}}\\*
\midrule
\endfirsthead
\midrule
\endhead
correctEnergy & \begin{minipage}[t]{2cm}logical\end{minipage} & \begin{minipage}[t]{2cm}.true.\end{minipage} & \begin{minipage}[t]{12cm}Scale final state momenta to fulfill energy and momentum conservation. If .false. energy conservation is violated\end{minipage}\\*
\midrule
StableInFormation & \begin{minipage}[t]{2cm}logical\end{minipage} & \begin{minipage}[t]{2cm}.true.\end{minipage} & \begin{minipage}[t]{12cm}Particles during its formation time are considered to be stable or not.\end{minipage}\\*
\midrule
omegaDecayMediumInfo & \begin{minipage}[t]{2cm}logical\end{minipage} & \begin{minipage}[t]{2cm}.false.\end{minipage} & \begin{minipage}[t]{12cm}Write out information about all decaying omega mesons to a file called "omegaMediumInfo.dat" (decay point, momentum, density, etc).\end{minipage}\\*
\midrule
omegaDecay\_restriction & \begin{minipage}[t]{2cm}integer\end{minipage} & \begin{minipage}[t]{2cm}0\end{minipage} & \begin{minipage}[t]{12cm}This switch, like omegaDecayMediumInfo, helps to analyze omega $\rightarrow$ pi0 gamma decays. It will only have an effect for omegaDecayMediumInfo = .true.\\ Possible values:\begin{itemize}\leftmargin0em\itemindent0pt\item 0 = none (default)\item 1 = vacuum ( rho $<$ 0.1 rho0)\item 2 = medium ( rho $>$ 0.1 rho0)\end{itemize} With the default value (0), all omega decays are carried out as usual. For the value 1, the decay products are only kept, if the decay happens in the vacuum (i.e. at rho $<$ 0.1 * rho0). For the value 2, the decay products are only kept, if the decay happens in the medium (i.e. at rho $>$ 0.1 * rho0). If the density does not meet these conditions, the decay products are simply removed and will not be put in the particle vector (and thus they will not appear in the analysis).\end{minipage}\\*
\bottomrule
\end{longtable}
{ }



% master_2body     code/collisions/twoBodyReactions/master_2Body.f90

\begin{longtable}{llll}
\toprule
\textbf{\large{master\_2body}} & \multicolumn{3}{l}{\footnotesize{code/collisions/twoBodyReactions/master\_2Body.f90}}\\*
\midrule
\endfirsthead
\midrule
\endhead
correctEnergy & \begin{minipage}[t]{2cm}logical\end{minipage} & \begin{minipage}[t]{2cm}.true.\end{minipage} & \begin{minipage}[t]{12cm}Scale final state momenta to fulfill energy and momentum conservation. If .false. energy conservation is violated\end{minipage}\\*
\midrule
baryonBaryonScattering & \begin{minipage}[t]{2cm}logical\end{minipage} & \begin{minipage}[t]{2cm}.true.\end{minipage} & \begin{minipage}[t]{12cm}Switch to turn off baryon-baryon-Scattering\end{minipage}\\*
\midrule
baryonMesonScattering & \begin{minipage}[t]{2cm}logical\end{minipage} & \begin{minipage}[t]{2cm}.true.\end{minipage} & \begin{minipage}[t]{12cm}Switch to turn off baryon-Meson-Scattering\end{minipage}\\*
\midrule
mesonMesonScattering & \begin{minipage}[t]{2cm}logical\end{minipage} & \begin{minipage}[t]{2cm}.true.\end{minipage} & \begin{minipage}[t]{12cm}Switch to turn off meson-Meson-Scattering\end{minipage}\\*
\midrule
usePythia & \begin{minipage}[t]{2cm}integer\end{minipage} & \begin{minipage}[t]{2cm}1\end{minipage} & \begin{minipage}[t]{12cm}This flag decides whether to use Fritiof or Pythia for high-energy collisions:\begin{itemize}\leftmargin0em\itemindent0pt\item 0: use Fritiof\item 1: use Pythia\end{itemize}NOTES\begin{itemize}\leftmargin0em\itemindent0pt\item This flag is not used in the baryon-antibaryon channel\end{itemize}\end{minipage}\\*
\midrule
usePythia\_BaB & \begin{minipage}[t]{2cm}integer\end{minipage} & \begin{minipage}[t]{2cm}0\end{minipage} & \begin{minipage}[t]{12cm}This flag decides whether to use Fritiof or Pythia for high-energy baryon-antibaryon collisions:\begin{itemize}\leftmargin0em\itemindent0pt\item 0: use Fritiof\item 1: use Pythia\end{itemize}\end{minipage}\\*
\midrule
useHiEnergy & \begin{minipage}[t]{2cm}logical\end{minipage} & \begin{minipage}[t]{2cm}.true.\end{minipage} & \begin{minipage}[t]{12cm}Switch to turn HiEnergy on/off. Formerly known as "useFritiof".\\NOTES\\ Please be very sure what you are doing when setting this parameter to .false.!\end{minipage}\\*
\midrule
HiEnergyThresholdBarMes & \begin{minipage}[t]{2cm}real\end{minipage} & \begin{minipage}[t]{2cm}2.2\end{minipage} & \begin{minipage}[t]{12cm}Sqrt(s) threshold for HiEnergy in Baryon-Meson Reactions\end{minipage}\\*
\midrule
HiEnergyThresholdBarMesDelta & \begin{minipage}[t]{2cm}real\end{minipage} & \begin{minipage}[t]{2cm}0.2\end{minipage} & \begin{minipage}[t]{12cm}width for the Sqrt(s) threshold for HiEnergy in Baryon-Meson Reactions\end{minipage}\\*
\midrule
HiEnergyThresholdBarBar & \begin{minipage}[t]{2cm}real\end{minipage} & \begin{minipage}[t]{2cm}3.4\end{minipage} & \begin{minipage}[t]{12cm}Sqrt(s) threshold for HiEnergy in Baryon-Baryon Reactions\end{minipage}\\*
\midrule
HiEnergyThresholdBarBarDelta & \begin{minipage}[t]{2cm}real\end{minipage} & \begin{minipage}[t]{2cm}0.1\end{minipage} & \begin{minipage}[t]{12cm}width for the Sqrt(s) threshold for HiEnergy in Baryon-Baryon Reactions\end{minipage}\\*
\midrule
HiEnergyThresholdBarAntibar & \begin{minipage}[t]{2cm}real\end{minipage} & \begin{minipage}[t]{2cm}2.38\end{minipage} & \begin{minipage}[t]{12cm}Sqrt(s) threshold for HiEnergy in Baryon-Antibaryon Reactions\end{minipage}\\*
\midrule
HiEnergyThresholdBarAntibarDelta & \begin{minipage}[t]{2cm}real\end{minipage} & \begin{minipage}[t]{2cm}0.0\end{minipage} & \begin{minipage}[t]{12cm}width for the Sqrt(s) threshold for HiEnergy in Baryon-Antibaryon Reactions\end{minipage}\\*
\midrule
useManni & \begin{minipage}[t]{2cm}logical\end{minipage} & \begin{minipage}[t]{2cm}.true.\end{minipage} & \begin{minipage}[t]{12cm}Flag, whether to use meson-baryon annhilation as proposed by Markus Wagner (Diploma, Giessen 2004), but with some enhanced treatment\end{minipage}\\*
\midrule
ElastAngDist & \begin{minipage}[t]{2cm}integer\end{minipage} & \begin{minipage}[t]{2cm}3\end{minipage} & \begin{minipage}[t]{12cm}Choice of angular distribution in (high-energy) elastic collisions (cf. DoColl\_Elast):\begin{itemize}\leftmargin0em\itemindent0pt\item 1 = isotropic\item 2 = J. Cugnon et al., NPA 352, 505 (1981)\item 3 = Pythia (default)\end{itemize}\end{minipage}\\*
\midrule
flagElastBB & \begin{minipage}[t]{2cm}logical\end{minipage} & \begin{minipage}[t]{2cm}.false.\end{minipage} & \begin{minipage}[t]{12cm}If .true., use a constant elastic baryon-baryon cross section of 40 mb and no inelastic baryon-baryon scattering.\end{minipage}\\*
\midrule
coarse & \begin{minipage}[t]{2cm}real, dimension(1:3)\end{minipage} & \begin{minipage}[t]{2cm}(/3.,4.,4./)\end{minipage} & \begin{minipage}[t]{12cm}coarse maximal impact parameter (in fm)\end{minipage}\\*
\midrule
bmax\_nucleonNucleon & \begin{minipage}[t]{2cm}real\end{minipage} & \begin{minipage}[t]{2cm}2.52\end{minipage} & \begin{minipage}[t]{12cm}Real maximal impact parameter for nucleon-nucleon-scattering. Maximal crossection is\\<pre>   bMax**2 * pi * 10 mb/fm**2 = (2.52**2*pi*10) mb  = 199.5 mb\end{minipage}\\*
\midrule
bmax\_nucleonResonance & \begin{minipage}[t]{2cm}real\end{minipage} & \begin{minipage}[t]{2cm}1.60\end{minipage} & \begin{minipage}[t]{12cm}Real maximal impact parameter for nucleon-resonance scattering. Maximal crossection is\\<pre>   bMax**2 * pi * 10 mb/fm**2 = (1.60**2*pi*10) mb  = 80.4 mb\end{minipage}\\*
\midrule
bmax\_hyperonNucleon & \begin{minipage}[t]{2cm}real\end{minipage} & \begin{minipage}[t]{2cm}2.52\end{minipage} & \begin{minipage}[t]{12cm}Real maximal impact parameter for hyperon-nucleon-scattering. Maximal crossection is\\<pre>   bMax**2 * pi * 10 mb/fm**2 = (2.52**2*pi*10) mb  = 199.5 mb\end{minipage}\\*
\midrule
bmax\_baryonPion & \begin{minipage}[t]{2cm}real\end{minipage} & \begin{minipage}[t]{2cm}2.52\end{minipage} & \begin{minipage}[t]{12cm}real maximal impact parameter for baryon pion scattering\end{minipage}\\*
\midrule
bmax\_baryonMeson & \begin{minipage}[t]{2cm}real\end{minipage} & \begin{minipage}[t]{2cm}2.52\end{minipage} & \begin{minipage}[t]{12cm}real maximal impact parameter for baryon-Meson-scattering\end{minipage}\\*
\midrule
bmax\_mesonMeson & \begin{minipage}[t]{2cm}real\end{minipage} & \begin{minipage}[t]{2cm}2.\end{minipage} & \begin{minipage}[t]{12cm}real maximal impact parameter for meson-meson-scattering\end{minipage}\\*
\midrule
correctEnergy\_message & \begin{minipage}[t]{2cm}logical\end{minipage} & \begin{minipage}[t]{2cm}.true.\end{minipage} & \begin{minipage}[t]{12cm}Switch off error message for energy correction failures.\end{minipage}\\*
\midrule
OverideSigma\_PiN & \begin{minipage}[t]{2cm}real\end{minipage} & \begin{minipage}[t]{2cm}-99.9\end{minipage} & \begin{minipage}[t]{12cm}Parameter to replace the calculated cross section for pi+N collision by a fixed value (in mb). Only in use if $>$= 0.\\ The elastic cross section is assumed to be 1/10 of the given value.\end{minipage}\\*
\midrule
OverideSigma\_RhoN & \begin{minipage}[t]{2cm}real\end{minipage} & \begin{minipage}[t]{2cm}-99.9\end{minipage} & \begin{minipage}[t]{12cm}Parameter to replace the calculated cross section for rho+N collision by a fixed value (in mb). Only in use if $>$= 0.\\ The elastic cross section is assumed to be 1/10 of the given value.\end{minipage}\\*
\midrule
OverideSigma\_PiPi & \begin{minipage}[t]{2cm}real\end{minipage} & \begin{minipage}[t]{2cm}-99.9\end{minipage} & \begin{minipage}[t]{12cm}Parameter to replace the calculated cross section for pi+pi collision by a fixed value (in mb). Only in use if $>$= 0.\\ We set sigma\_elast = sigma\_tot\end{minipage}\\*
\midrule
Overide\_PiPi\_ResIsElast & \begin{minipage}[t]{2cm}logical\end{minipage} & \begin{minipage}[t]{2cm}.false.\end{minipage} & \begin{minipage}[t]{12cm}Flag to replace the calculated cross section for pi+pi collision; The calculated resonant cross section will be transformed into the elastic cross section. Thus no resonances will be propagated explicitely, but they show up in the cross section\\ We set sigma\_elast = sigma\_Res, sigma\_Res = 0, sigma\_tot = sigma\_elast\\ please note: background processes as pi pi $\leftrightarrow$ K K\~{} are *not* affected by this switch. You have to disable those additionally by hand, see mesMes\_do2to2\end{minipage}\\*
\midrule
omega\_K\_factor & \begin{minipage}[t]{2cm}real\end{minipage} & \begin{minipage}[t]{2cm}2.\end{minipage} & \begin{minipage}[t]{12cm}Modification factor for the inelastic omega-nucleon cross section. Necessary to describe transpacrency ratio data measured by CBELSA/TAPS, see: http://arxiv.org/abs/1210.3074\end{minipage}\\*
\midrule
mesMes\_do2to2 & \begin{minipage}[t]{2cm}logical\end{minipage} & \begin{minipage}[t]{2cm}.true.\end{minipage} & \begin{minipage}[t]{12cm}flag whether to do m m' $\leftrightarrow$ K K\~{}, K K*\~{} etc.\end{minipage}\\*
\midrule
mesMes\_useWidth & \begin{minipage}[t]{2cm}logical\end{minipage} & \begin{minipage}[t]{2cm}.false.\end{minipage} & \begin{minipage}[t]{12cm}flag whether to use the width in m m' $\leftrightarrow$ K K\~{}, K K*\~{} etc. This is needed to enforce detailed balance. Otherwise only pole masses are used.\end{minipage}\\*
\midrule
doScaleResidue & \begin{minipage}[t]{2cm}logical\end{minipage} & \begin{minipage}[t]{2cm}.true.\end{minipage} & \begin{minipage}[t]{12cm}scale the cross section of real-pert collisions by a factor N'/N or Z'/Z for a scattering on a neutron or proton, where N' and Z' are calculated via the residuum.\end{minipage}\\*
\bottomrule
\end{longtable}
{ }



% master_3body     code/collisions/threeBodyReactions/master_3Body.f90

\begin{longtable}{llll}
\toprule
\textbf{\large{master\_3body}} & \multicolumn{3}{l}{\footnotesize{code/collisions/threeBodyReactions/master\_3Body.f90}}\\*
\midrule
\endfirsthead
\midrule
\endhead
correctEnergy & \begin{minipage}[t]{2cm}logical\end{minipage} & \begin{minipage}[t]{2cm}.true.\end{minipage} & \begin{minipage}[t]{12cm}Scale final state momenta to fulfill energy and momentum conservation. If .false., energy conservation is violated.\end{minipage}\\*
\midrule
radiusNukSearch & \begin{minipage}[t]{2cm}real\end{minipage} & \begin{minipage}[t]{2cm}2.9\end{minipage} & \begin{minipage}[t]{12cm}Radius for the search of nucleons, i.e. the radius in which nucleons shall be searched for at rho\_0.\end{minipage}\\*
\midrule
deltaThreeBody & \begin{minipage}[t]{2cm}\end{minipage} & \begin{minipage}[t]{2cm}\end{minipage} & \begin{minipage}[t]{12cm}\end{minipage}\\*
\midrule
pionThreeBody & \begin{minipage}[t]{2cm}logical\end{minipage} & \begin{minipage}[t]{2cm}.true.\end{minipage} & \begin{minipage}[t]{12cm}Switch for the NNpion $\rightarrow$ NN processes (false=OFF).\end{minipage}\\*
\midrule
positionNNpi & \begin{minipage}[t]{2cm}logical\end{minipage} & \begin{minipage}[t]{2cm}.false.\end{minipage} & \begin{minipage}[t]{12cm}This switch determines where the final state particles in NNpi$\rightarrow$NN are positioned:\begin{itemize}\leftmargin0em\itemindent0pt\item true: pion position\item false: center of NNPi (default)\end{itemize}\end{minipage}\\*
\bottomrule
\end{longtable}
{ }



% MatrixElementQE     code/init/lepton/matrixElementQE.f90

\begin{longtable}{llll}
\toprule
\textbf{\large{MatrixElementQE}} & \multicolumn{3}{l}{\footnotesize{code/init/lepton/matrixElementQE.f90}}\\*
\midrule
\endfirsthead
\midrule
\endhead
useQEextraterm & \begin{minipage}[t]{2cm}logical\end{minipage} & \begin{minipage}[t]{2cm}.true.\end{minipage} & \begin{minipage}[t]{12cm}switch on/off an extra term appearing in the current due to different masses of in- and outgoing nucleons\end{minipage}\\*
\midrule
useCorrelations & \begin{minipage}[t]{2cm}logical\end{minipage} & \begin{minipage}[t]{2cm}.false.\end{minipage} & \begin{minipage}[t]{12cm}switch on/off RPA correlations according to Nieves, Amaro, Valverde, PRC70, 055503 (2004)\end{minipage}\\*
\midrule
nievesCorr\_para & \begin{minipage}[t]{2cm}integer\end{minipage} & \begin{minipage}[t]{2cm}2\end{minipage} & \begin{minipage}[t]{12cm}if RPA correlations are switched on, this parameter decides which set of varibles to use:\begin{itemize}\leftmargin0em\itemindent0pt\item 1: modified Nieves et al., PRC70, 055503 (2004)\item 2: original Nieves et al., PRC70, 055503 (2004)\item 3: Tselyaev, Speth et al., PRC75, 014315 (2007)\end{itemize}\end{minipage}\\*
\midrule
gp & \begin{minipage}[t]{2cm}real\end{minipage} & \begin{minipage}[t]{2cm}0.63\end{minipage} & \begin{minipage}[t]{12cm}vary gp if RPA correlations are switched on\end{minipage}\\*
\midrule
withScalarInt & \begin{minipage}[t]{2cm}logical\end{minipage} & \begin{minipage}[t]{2cm}.true.\end{minipage} & \begin{minipage}[t]{12cm}switch on/off scalar interactions\end{minipage}\\*
\bottomrule
\end{longtable}
{ }



% MediumModule     code/density/medium.f90

\begin{longtable}{llll}
\toprule
\textbf{\large{MediumModule}} & \multicolumn{3}{l}{\footnotesize{code/density/medium.f90}}\\*
\midrule
\endfirsthead
\midrule
\endhead
mediumCutOff & \begin{minipage}[t]{2cm}real\end{minipage} & \begin{minipage}[t]{2cm}1.E-8\end{minipage} & \begin{minipage}[t]{12cm}If the density is lower than this value, then we treat the medium like vacuum.\end{minipage}\\*
\bottomrule
\end{longtable}
{ }



% mesonPotential     code/potential/mesonPotential.f90

\begin{longtable}{llll}
\toprule
\textbf{\large{mesonPotential}} & \multicolumn{3}{l}{\footnotesize{code/potential/mesonPotential.f90}}\\*
\midrule
\endfirsthead
\midrule
\endhead
pionPot\_Switch & \begin{minipage}[t]{2cm}integer\end{minipage} & \begin{minipage}[t]{2cm}0\end{minipage} & \begin{minipage}[t]{12cm}Switch for pion potential:\begin{itemize}\leftmargin0em\itemindent0pt\item 0 = no potential\item 1 = Oset potential (NPA 554), which is valid up to 50 MeV kinetic energy\item 2 = Kapusta suggestion for pion potential (rather unusual)\item 3 = Delta-Hole potential, which is valid up to 130 MeV kinetic energy\item 4 = Smooth spline transition between switch 1 and 3.\end{itemize}NOTES\\ Can be set in namelist mesonPotential.\end{minipage}\\*
\midrule
noPerturbativePotential & \begin{minipage}[t]{2cm}logical\end{minipage} & \begin{minipage}[t]{2cm}.false.\end{minipage} & \begin{minipage}[t]{12cm}Switch for potential of perturbative particles. If .true. then perturbative mesons feel no potential.\\NOTES\\ Can be set in namelist mesonPotential.\end{minipage}\\*
\midrule
vectorMesonPot & \begin{minipage}[t]{2cm}integer\end{minipage} & \begin{minipage}[t]{2cm}0\end{minipage} & \begin{minipage}[t]{12cm}Switch for medium-modification of vector mesons:\begin{itemize}\leftmargin0em\itemindent0pt\item 0 = no modification\item 1 = Brown-Rho-Scaling\item 2 = Brown-Rho-Scaling with momentum dependence   according to Kondtradyuk (see page 162 in Effenberger's thesis).   Currently not available!\end{itemize}NOTES\\ Can be set in namelist mesonPotential.\end{minipage}\\*
\midrule
brownRho & \begin{minipage}[t]{2cm}real\end{minipage} & \begin{minipage}[t]{2cm}0.16\end{minipage} & \begin{minipage}[t]{12cm}Brown-Rho scaling parameter alpha.\end{minipage}\\*
\bottomrule
\end{longtable}
{ }



% MesonWidthVacuum     code/width/mesonWidthVacuum.f90

\begin{longtable}{llll}
\toprule
\textbf{\large{MesonWidthVacuum}} & \multicolumn{3}{l}{\footnotesize{code/width/mesonWidthVacuum.f90}}\\*
\midrule
\endfirsthead
\midrule
\endhead
omega\_width & \begin{minipage}[t]{2cm}integer\end{minipage} & \begin{minipage}[t]{2cm}1\end{minipage} & \begin{minipage}[t]{12cm}Select a parametrization for the omega vacuum width:\begin{itemize}\leftmargin0em\itemindent0pt\item 1 = GiBUU default (a la Manley)\item 2 = Muehlich\end{itemize}\end{minipage}\\*
\midrule
srts\_srt\_switch & \begin{minipage}[t]{2cm}logical\end{minipage} & \begin{minipage}[t]{2cm}.false.\end{minipage} & \begin{minipage}[t]{12cm}Modifies the width according to S. Leupold's definition of the width, one especially has to exchange s against sqrt(s) in the denominator of Formula 2.76 of Effenbergers Phd\\NOTES\\ The default value is .false. and the power of the mass resp. sqrt(s) is 1. If the flag is .true., the power is 2.\end{minipage}\\*
\bottomrule
\end{longtable}
{ }



% ModifyParticles     code/database/particleProperties.f90

\begin{longtable}{llll}
\toprule
\textbf{\large{ModifyParticles}} & \multicolumn{3}{l}{\footnotesize{code/database/particleProperties.f90}}\\*
\midrule
\endfirsthead
\midrule
\endhead
mass & \begin{minipage}[t]{2cm}real, dimension(1:pion+nMes-1)\end{minipage} & \begin{minipage}[t]{2cm}-1.0\end{minipage} & \begin{minipage}[t]{12cm}Input array for modifications on the particle mass\\NOTES\\ This array is intended to "input" values for the mass of the particles, which are different from the default. Therefore only entries, which are positive after reading the file are stored in the internal database.\end{minipage}\\*
\midrule
width & \begin{minipage}[t]{2cm}real, dimension(1:pion+nMes-1)\end{minipage} & \begin{minipage}[t]{2cm}-1.0\end{minipage} & \begin{minipage}[t]{12cm}Input array for modifications on the particle width\\NOTES\\ This array is intended to "input" values for the width of the particles, which are different from the default. Therefore only entries, which are positive after reading the file are stored in the internal database.\end{minipage}\\*
\midrule
stabilityFlag & \begin{minipage}[t]{2cm}integer, dimension(1:pion+nMes-1)\end{minipage} & \begin{minipage}[t]{2cm}-1\end{minipage} & \begin{minipage}[t]{12cm}Input array for modifications on the particle stability\\NOTES\\ This array is intended to "input" values for the stability of the particles, which are different from the default. Therefore only entries, which are $>$-1 after reading the file are stored in the internal database.\\ The index of the array is the particle ID. The value encodes on a bitwise level, how the particle may decay (cf. also master\_1Body):\begin{itemize}\leftmargin0em\itemindent0pt\item 1: particle may decay during run, if Gamma $>$ gammaCutOff\item 2: particle may decay at the end of the run, if Gamma $>$ 0.\item 4: particle may decay at the end via Jetset, if there the parameters allow for a decay.\end{itemize} The default values are one of the following:\begin{itemize}\leftmargin0em\itemindent0pt\item 0: particle may not decay at all (i.e. it is stable)\item 3: particle may decay both during run and at the end (combination of 1 and 2)\end{itemize}\end{minipage}\\*
\midrule
propagated & \begin{minipage}[t]{2cm}integer, dimension(1:pion+nMes-1)\end{minipage} & \begin{minipage}[t]{2cm}-1\end{minipage} & \begin{minipage}[t]{12cm}Input array for modifications on the flag propagated\\NOTES\\ This array is intended to "input" values for the flag of the particles, which are different from the default. Therefore only entries, which are zero or positive after reading the file are stored in the internal database. Here 0 is understood as .false., while all positive values stand for .true.\end{minipage}\\*
\bottomrule
\end{longtable}
{ }



% neutrino_induced     code/init/neutrino/initNeutrino.f90

\begin{longtable}{llll}
\toprule
\textbf{\large{neutrino\_induced}} & \multicolumn{3}{l}{\footnotesize{code/init/neutrino/initNeutrino.f90}}\\*
\midrule
\endfirsthead
\midrule
\endhead
process\_ID & \begin{minipage}[t]{2cm}integer\end{minipage} & \begin{minipage}[t]{2cm}2\end{minipage} & \begin{minipage}[t]{12cm}Determine the process (cf. module leptonicID):\begin{itemize}\leftmargin0em\itemindent0pt\item 1 = EM\item 2 = CC\item 3 = NC\item -1 = antiEM\item -2 = antiCC\item -3 = antiNC\end{itemize}\end{minipage}\\*
\midrule
flavor\_ID & \begin{minipage}[t]{2cm}integer\end{minipage} & \begin{minipage}[t]{2cm}2\end{minipage} & \begin{minipage}[t]{12cm}Determine the lepton flavor:\begin{itemize}\leftmargin0em\itemindent0pt\item 1 = electron\item 2 = muon\item 3 = tau\end{itemize}\end{minipage}\\*
\midrule
nuXsectionMode & \begin{minipage}[t]{2cm}integer\end{minipage} & \begin{minipage}[t]{2cm}0\end{minipage} & \begin{minipage}[t]{12cm}To choose which kind of Xsection is calculated. All values set in module neutrino\_IDTable.f90\\ possible values:\begin{itemize}\leftmargin0em\itemindent0pt\item 0 = integratedSigma: required input: enu\item 1 = dSigmadCosThetadElepton: required input: enu, costheta, elepton\item 2 = dSigmadQ2dElepton: required input: enu, Q2, elepton\item 4 = dSigmadCosTheta: required input: enu, costheta\item 5 = dSigmadElepton: required input: enu, elepton\item 6 = dSigmaMC: required input: enu\item 7 = dSigmaMC\_dW: required input: enu, W\item 3 = dSigmaMC\_dQ2: required input: enu, Q2\end{itemize} calculation for specific experiments taking into account the flux (choose your favorite experiment with flag nuExp):\begin{itemize}\leftmargin0em\itemindent0pt\item 10 = EXP\_dSigmadEnu\item 11 = EXP\_dSigmadCosThetadElepton\item 12 = EXP\_dSigmadQ2dElepton\item 14 = EXP\_dSigmadCosTheta\item 15 = EXP\_dSigmadElepton\item 16 = EXP\_dSigmaMC\item 17 = EXP\_dSigmaMC\_dW\item 13 = EXP\_dSigmaMC\_dQ2\end{itemize}\end{minipage}\\*
\midrule
nuExp & \begin{minipage}[t]{2cm}integer\end{minipage} & \begin{minipage}[t]{2cm}0\end{minipage} & \begin{minipage}[t]{12cm}\begin{itemize}\leftmargin0em\itemindent0pt\item 0 = no specific experiment\item 1 = MiniBooNE neutrino flux (in neutrino mode = positive polarity)\item 2 = ANL\item 3 = K2K\item 4 = BNL\item 5 = MiniBooNE antienutrino flux (in antineutrino mode = negative polarity)\item 6 = MINOS muon-neutrino  in neutrino mode\item 7 = MINOS muon-antineutrino  in neutrino mode\item 8 = NOVA neutrino (medium energy NuMI, 14 mrad off-axis), FD\item 9 = T2K neutrino off-axix 2.5 degrees ( at ND280 detector )\item 10 = uniform distribution from Eflux\_min to Eflux\_max       (see namelist nl\_neutrino\_energyFlux in the module expNeutrinoFluxes)\item 11 = MINOS muon-neutrino  in antineutrino mode\item 12 = MINOS muon-antineutrino  in antineutrino mode\item 13 = MINERvA muon neutrino, old flux\item 14 = MINERvA muon antineutrino, old flux\item 15 = LBNF/DUNE in neutrino mode\item 16 = LBNF/DUNE in antineutrino mode\item 17 = LBNO neutrino in neutrino mode\item 18 = NOMAD\item 19 = BNB nue          BNB= Booster Neutrino Beam\item 20 = BNB nuebar\item 21 = BNB numu\item 22 = BNB numubar\item 23 = NOvA ND\item 24 = T2K on axis\item 25 = MINERvA, 2016 flux\item 99 = user provided input file\end{itemize}\end{minipage}\\*
\midrule
includeQE & \begin{minipage}[t]{2cm}logical\end{minipage} & \begin{minipage}[t]{2cm}.true.\end{minipage} & \begin{minipage}[t]{12cm}include QE scattering\end{minipage}\\*
\midrule
includeDELTA & \begin{minipage}[t]{2cm}logical\end{minipage} & \begin{minipage}[t]{2cm}.true.\end{minipage} & \begin{minipage}[t]{12cm}include Delta excitation\end{minipage}\\*
\midrule
includeRES & \begin{minipage}[t]{2cm}logical\end{minipage} & \begin{minipage}[t]{2cm}.true.\end{minipage} & \begin{minipage}[t]{12cm}include excitation of higher resonances\end{minipage}\\*
\midrule
include1pi & \begin{minipage}[t]{2cm}logical\end{minipage} & \begin{minipage}[t]{2cm}.false.\end{minipage} & \begin{minipage}[t]{12cm}include one-pion cross section see neutrinoXsection.f90 for details: there one might choose between different models and also whether it is taken as background or as total cross section\end{minipage}\\*
\midrule
include2pi & \begin{minipage}[t]{2cm}logical\end{minipage} & \begin{minipage}[t]{2cm}.false.\end{minipage} & \begin{minipage}[t]{12cm}include 2 pion background channel\end{minipage}\\*
\midrule
includeDIS & \begin{minipage}[t]{2cm}logical\end{minipage} & \begin{minipage}[t]{2cm}.false.\end{minipage} & \begin{minipage}[t]{12cm}include DIS contribution\end{minipage}\\*
\midrule
include2p2hQE & \begin{minipage}[t]{2cm}logical\end{minipage} & \begin{minipage}[t]{2cm}.false.\end{minipage} & \begin{minipage}[t]{12cm}include 2p2h QE contribution\end{minipage}\\*
\midrule
include2p2hDelta & \begin{minipage}[t]{2cm}logical\end{minipage} & \begin{minipage}[t]{2cm}.false.\end{minipage} & \begin{minipage}[t]{12cm}include 2p2h Delta contribution\end{minipage}\\*
\midrule
sigmacut & \begin{minipage}[t]{2cm}real\end{minipage} & \begin{minipage}[t]{2cm}10e-4\end{minipage} & \begin{minipage}[t]{12cm}events with a cross section smaller than this value are skipped.\end{minipage}\\*
\midrule
realRun & \begin{minipage}[t]{2cm}logical\end{minipage} & \begin{minipage}[t]{2cm}.false.\end{minipage} & \begin{minipage}[t]{12cm}Do not initialize the final state particles as perturbative particles but as real ones.\end{minipage}\\*
\midrule
printAbsorptionXS & \begin{minipage}[t]{2cm}logical\end{minipage} & \begin{minipage}[t]{2cm}.false.\end{minipage} & \begin{minipage}[t]{12cm}flag to produce output about inclusive (absorption) cross sections\end{minipage}\\*
\midrule
printInclHist & \begin{minipage}[t]{2cm}logical\end{minipage} & \begin{minipage}[t]{2cm}.true.\end{minipage} & \begin{minipage}[t]{12cm}flag to produce additional output about inclusive cross sections\\ only checked, if printAbsorptionXS = T\end{minipage}\\*
\midrule
FileNameFlux & \begin{minipage}[t]{2cm}character(1000)\end{minipage} & \begin{minipage}[t]{2cm}''\end{minipage} & \begin{minipage}[t]{12cm}The absolute filename of the file containing flux info, if user supplied\\ possible values:\begin{itemize}\leftmargin0em\itemindent0pt\item if given, but does not contain '/':   default is '[path\_To\_Input]/[FileNameFlux]'\item otherwise: filename is absolute, including path ('\~{}' is okay)\end{itemize} NOTE if you want to use the file 'XXX.dat' in the actual directory, give it as './XXX.dat'\end{minipage}\\*
\midrule
storeNucleon & \begin{minipage}[t]{2cm}integer\end{minipage} & \begin{minipage}[t]{2cm}2\end{minipage} & \begin{minipage}[t]{12cm}indicate which kind of struck nucleon to save:\begin{itemize}\leftmargin0em\itemindent0pt\item 1: free Nucleon (i.e. potential removed)\item 2: bound nucleon\end{itemize}NOTES\\ real check of energy and momentum conservation only possible with '2'\end{minipage}\\*
\midrule
equalWeights\_Mode & \begin{minipage}[t]{2cm}integer\end{minipage} & \begin{minipage}[t]{2cm}0\end{minipage} & \begin{minipage}[t]{12cm}possible values are:\begin{itemize}\leftmargin0em\itemindent0pt\item 0: default perweight mode is used (default)\item 1: default perweight mode is used, but max is printed\item 2: MC rejection method is used.\end{itemize} In the default mode, the perweights of the final particles are given by cross section/(A * numEnsembles)\\ If equalWeightsMode==2, then the perweights are given by equalWeights\_Max/(A * numEnsembles)\\ Please check in the output the line "numberOfSuccess = ..." for the number of events actually generated.\end{minipage}\\*
\midrule
equalWeights\_Max & \begin{minipage}[t]{2cm}real\end{minipage} & \begin{minipage}[t]{2cm}-1e99\end{minipage} & \begin{minipage}[t]{12cm}The maximum value the MC-rejection method is done against.\end{minipage}\\*
\bottomrule
\end{longtable}
{ }



% neutrino_MAIDlikeBG     code/init/neutrino/singlePionProductionMAIDlike.f90

\begin{longtable}{llll}
\toprule
\textbf{\large{neutrino\_MAIDlikeBG}} & \multicolumn{3}{l}{\footnotesize{code/init/neutrino/singlePionProductionMAIDlike.f90}}\\*
\midrule
\endfirsthead
\midrule
\endhead
b\_proton\_pinull & \begin{minipage}[t]{2cm}real\end{minipage} & \begin{minipage}[t]{2cm}3.\end{minipage} & \begin{minipage}[t]{12cm}strength of 1pi BG for CC, multiplies EM BG 3. is  tuned to ANL, 6. is tuned to BNL\end{minipage}\\*
\midrule
b\_neutron\_piplus & \begin{minipage}[t]{2cm}real\end{minipage} & \begin{minipage}[t]{2cm}1.5\end{minipage} & \begin{minipage}[t]{12cm}strength of 1pi BG for CC, multiplies EM BG 1.5 is tuned to ANL, 3. is tuned to BNL\end{minipage}\\*
\bottomrule
\end{longtable}
{ }



% neutrino_matrixelement     code/init/neutrino/NeutrinoMatrixElement.f90

\begin{longtable}{llll}
\toprule
\textbf{\large{neutrino\_matrixelement}} & \multicolumn{3}{l}{\footnotesize{code/init/neutrino/NeutrinoMatrixElement.f90}}\\*
\midrule
\endfirsthead
\midrule
\endhead
which\_resonanceModel & \begin{minipage}[t]{2cm}integer\end{minipage} & \begin{minipage}[t]{2cm}0\end{minipage} & \begin{minipage}[t]{12cm}to change between different realizations of the matrix elements:\begin{itemize}\leftmargin0em\itemindent0pt\item 0 = with Fortran calculated matrix elements containing all resonances (default)\item 1 = with Mathematica calculated matrix elements (only Delta)\item 2 = Rein and Sehgals matrix elements\end{itemize}\end{minipage}\\*
\bottomrule
\end{longtable}
{ }



% NeutrinoAnalysis     code/analysis/neutrinoAnalysis.f90

\begin{longtable}{llll}
\toprule
\textbf{\large{NeutrinoAnalysis}} & \multicolumn{3}{l}{\footnotesize{code/analysis/neutrinoAnalysis.f90}}\\*
\midrule
\endfirsthead
\midrule
\endhead
detailed\_diff\_output & \begin{minipage}[t]{2cm}logical\end{minipage} & \begin{minipage}[t]{2cm}.false.\end{minipage} & \begin{minipage}[t]{12cm}If .true. then also the detailed output of differential cross sections is produced\end{minipage}\\*
\midrule
include\_W\_dist & \begin{minipage}[t]{2cm}logical\end{minipage} & \begin{minipage}[t]{2cm}.false.\end{minipage} & \begin{minipage}[t]{12cm}If .true. then the invariant mass distributions for events with 1 pion and 1 nucleon in the final state are produced\end{minipage}\\*
\midrule
kineticEnergyDetectionThreshold\_lepton & \begin{minipage}[t]{2cm}real\end{minipage} & \begin{minipage}[t]{2cm}0.0\end{minipage} & \begin{minipage}[t]{12cm}kineticEnergyDetectionThreshold only lepton kinetic energies above this threshold can be detected This cut affects *all* events, not just the outgoing lepton!\end{minipage}\\*
\midrule
AngleUpperDetectionThresholdDegrees\_lepton & \begin{minipage}[t]{2cm}real\end{minipage} & \begin{minipage}[t]{2cm}180.0\end{minipage} & \begin{minipage}[t]{12cm}lepton angles up to this value can be detected This cut affects *all* events, not just the outgoing lepton!\end{minipage}\\*
\midrule
kineticEnergyDetectionThreshold\_nucleon & \begin{minipage}[t]{2cm}real\end{minipage} & \begin{minipage}[t]{2cm}0.0\end{minipage} & \begin{minipage}[t]{12cm}kineticEnergyDetectionThreshold lower detection threshold for nucleon kinetic energies\end{minipage}\\*
\midrule
AngleUpperDetectionThresholdDegrees\_nucleon & \begin{minipage}[t]{2cm}real\end{minipage} & \begin{minipage}[t]{2cm}180.0\end{minipage} & \begin{minipage}[t]{12cm}nucleon angles up to this value can be detected\end{minipage}\\*
\midrule
kineticEnergyDetectionThreshold\_chargedpion & \begin{minipage}[t]{2cm}real\end{minipage} & \begin{minipage}[t]{2cm}0.0\end{minipage} & \begin{minipage}[t]{12cm}kineticEnergyDetectionThreshold\end{minipage}\\*
\midrule
AngleUpperDetectionThresholdDegrees\_chargedpion & \begin{minipage}[t]{2cm}real\end{minipage} & \begin{minipage}[t]{2cm}180.0\end{minipage} & \begin{minipage}[t]{12cm}charged pion angles up to this value can be detected\end{minipage}\\*
\midrule
kineticEnergyDetectionThreshold\_neutralpion & \begin{minipage}[t]{2cm}real\end{minipage} & \begin{minipage}[t]{2cm}0.0\end{minipage} & \begin{minipage}[t]{12cm}kineticEnergyDetectionThreshold\end{minipage}\\*
\midrule
AngleUpperDetectionThresholdDegrees\_neutralpion & \begin{minipage}[t]{2cm}real\end{minipage} & \begin{minipage}[t]{2cm}180.0\end{minipage} & \begin{minipage}[t]{12cm}neutral pion angles angles up to this value can be detected\end{minipage}\\*
\midrule
applyCuts & \begin{minipage}[t]{2cm}integer\end{minipage} & \begin{minipage}[t]{2cm}0\end{minipage} & \begin{minipage}[t]{12cm}This parameter encodes 'binary', which cuts should be applied:\begin{itemize}\leftmargin0em\itemindent0pt\item 1: lepton\_acceptance\item 2: isBound\item 4: isBelowThreshold\end{itemize} Instead of having three indipendent flags (with values=0 or 1) as e.g. labelled 'doLepton', 'doIsBound', 'doBelowThr', applyCuts combines them formally into one number as\\<pre>   applyCuts = 1*doLepton + 2*doIsBound + 4*doBelowThr So by setting any number between 0 and 7, one can individually switch on and off each of these cuts.\\ 'lepton\_acceptance' uses the input parameters:\begin{itemize}\leftmargin0em\itemindent0pt\item kineticEnergyDetectionThreshold\_lepton (for all kind of outgoing leptons)\item AngleUpperDetectionThresholdDegrees\_lepton\end{itemize} 'isBound' tests, whether kinetic energy plus potential is $<$0\\ 'isBelowThreshold' uses the input parameters:\begin{itemize}\leftmargin0em\itemindent0pt\item kineticEnergyDetectionThreshold\_lepton (only for muons)\item AngleUpperDetectionThresholdDegrees\_lepton (only for muons)\item kineticEnergyDetectionThreshold\_nucleon (only for nucleons)\item AngleUpperDetectionThresholdDegrees\_nucleon (only for nucleons)\item kineticEnergyDetectionThreshold\_chargedpion (only for charged pion)\item AngleUpperDetectionThresholdDegrees\_chargedpion (only for charged pion)\item kineticEnergyDetectionThreshold\_neutralpion (only for neutral pions)\item AngleUpperDetectionThresholdDegrees\_neutralpion (only for neutral pions)\end{itemize} Some examples:\begin{itemize}\leftmargin0em\itemindent0pt\item To generate full inclusive output, set the value applyCuts=0\item To generate output where bound nucleons are dropped, set applyCuts=2\item To generate output with specific experimental cuts for the outgoing   hadrons, set applyCuts=4 or applyCuts=6 and set the corresponding   threshold parameters accordingly.\item If in the experiment also cuts on the outgoing lepton are used, set   applyCuts=7 and set the corresponding threshold parameters accordingly.\end{itemize}NOTES\\ These cuts affects the output into the file "FinalEvents.dat".\\ The cut 'lepton\_acceptance' (de-)selects the full event, while the other two cuts only decide whether a specific particle is accepted or not.\\ The kinetic energy of a bound nucleon is $<$ 0. Therefore using the default value kineticEnergyDetectionThreshold\_nucleon=0.0 also tests, whether the particle is bound or not. Set the parameter to a large negative value to become ineffective.\end{minipage}\\*
\midrule
Fissum\_analysis & \begin{minipage}[t]{2cm}logical\end{minipage} & \begin{minipage}[t]{2cm}.false.\end{minipage} & \begin{minipage}[t]{12cm}do analysis with cuts as needed for Fig 25 in Fissum et al, PRC 70, 034606 (2004)\end{minipage}\\*
\midrule
ZeroPion\_analysis & \begin{minipage}[t]{2cm}logical\end{minipage} & \begin{minipage}[t]{2cm}.false.\end{minipage} & \begin{minipage}[t]{12cm}produce output of xsec for various final states with 0 pions and 2 pions see file see sigma\_0pions.dat  for the list of the final states\\ see files neutrino\_0pions.dat,  neutrino\_0pions\_QE.dat, neutrino\_0pions\_Delta.dat, ... for output\end{minipage}\\*
\midrule
calorimetric\_analysis & \begin{minipage}[t]{2cm}logical\end{minipage} & \begin{minipage}[t]{2cm}.false.\end{minipage} & \begin{minipage}[t]{12cm}do calorimetric energy-transfer and neutrino-energy reconstruction (for each QE, Delta, ...)  as in the MINOS experiment\end{minipage}\\*
\midrule
radialScale & \begin{minipage}[t]{2cm}real\end{minipage} & \begin{minipage}[t]{2cm}1.5\end{minipage} & \begin{minipage}[t]{12cm}If radial position of nucleon $<$ radialScale*target radius, then the nucleon is assumed to be bound\end{minipage}\\*
\midrule
reconstruct\_neutrino\_energy & \begin{minipage}[t]{2cm}logical\end{minipage} & \begin{minipage}[t]{2cm}.false.\end{minipage} & \begin{minipage}[t]{12cm}reconstruct neutrino energy for final state in "specificEvent\_analysis"\\NOTES\\ .true. must be combined with specificEvent\_analysis=.true. and at least one specific event .true.\end{minipage}\\*
\midrule
outputEvents & \begin{minipage}[t]{2cm}logical\end{minipage} & \begin{minipage}[t]{2cm}.false.\end{minipage} & \begin{minipage}[t]{12cm}If .true. then all events are written to the file 'FinalEvents.dat'.\end{minipage}\\*
\midrule
specificEvent\_analysis & \begin{minipage}[t]{2cm}logical\end{minipage} & \begin{minipage}[t]{2cm}.false.\end{minipage} & \begin{minipage}[t]{12cm}do analysis for specific final states\end{minipage}\\*
\midrule
Xsection\_analysis & \begin{minipage}[t]{2cm}logical\end{minipage} & \begin{minipage}[t]{2cm}.false.\end{minipage} & \begin{minipage}[t]{12cm}If .true. then files "...\_total\_Xsection\_..."  and "...\_dSigmadEkin\_..." are printed.\end{minipage}\\*
\midrule
doPipe & \begin{minipage}[t]{2cm}logical\end{minipage} & \begin{minipage}[t]{2cm}.false.\end{minipage} & \begin{minipage}[t]{12cm}If .true. then events are not written to the file 'FinalEvents.dat', but insted written into a named pipe (fifo) with the name fileNamePipe.\end{minipage}\\*
\midrule
fileNamePipe & \begin{minipage}[t]{2cm}character(len=1000)\end{minipage} & \begin{minipage}[t]{2cm}""\end{minipage} & \begin{minipage}[t]{12cm}name of the pipe to be used\end{minipage}\\*
\midrule
DoOutChannels & \begin{minipage}[t]{2cm}logical\end{minipage} & \begin{minipage}[t]{2cm}.false.\end{minipage} & \begin{minipage}[t]{12cm}switch on/off: reporting of all final state channels\end{minipage}\\*
\bottomrule
\end{longtable}
{ }



% nl_calorimetric_analysis     code/analysis/neutrinoAnalysis.f90

\begin{longtable}{llll}
\toprule
\textbf{\large{nl\_calorimetric\_analysis}} & \multicolumn{3}{l}{\footnotesize{code/analysis/neutrinoAnalysis.f90}}\\*
\midrule
\endfirsthead
\midrule
\endhead
numin & \begin{minipage}[t]{2cm}real\end{minipage} & \begin{minipage}[t]{2cm}0.\end{minipage} & \begin{minipage}[t]{12cm}for calorimetric analysis: values for transferred energy; only work if calorimetric\_analysis is .true. set the min, max and bins for nu distributions\end{minipage}\\*
\midrule
numax & \begin{minipage}[t]{2cm}real\end{minipage} & \begin{minipage}[t]{2cm}10.0\end{minipage} & \begin{minipage}[t]{12cm}for calorimetric analysis: values for transferred energy; only work if calorimetric\_analysis is .true. set the min, max and bins for nu distributions\end{minipage}\\*
\midrule
nubin & \begin{minipage}[t]{2cm}real\end{minipage} & \begin{minipage}[t]{2cm}0.1\end{minipage} & \begin{minipage}[t]{12cm}for calorimetric analysis: values for transferred energy; only work if calorimetric\_analysis is .true. set the min, max and bins for nu distributions\end{minipage}\\*
\midrule
Enumin & \begin{minipage}[t]{2cm}real\end{minipage} & \begin{minipage}[t]{2cm}0.\end{minipage} & \begin{minipage}[t]{12cm}for calorimetric analysis: values for neutrino energy; only work if calorimetric\_analysis is .true. set the min, max and bins for nu distributions\end{minipage}\\*
\midrule
Enumax & \begin{minipage}[t]{2cm}real\end{minipage} & \begin{minipage}[t]{2cm}10.0\end{minipage} & \begin{minipage}[t]{12cm}for calorimetric analysis: values for neutrino energy; only work if calorimetric\_analysis is .true. set the min, max and bins for nu distributions\end{minipage}\\*
\midrule
Enubin & \begin{minipage}[t]{2cm}real\end{minipage} & \begin{minipage}[t]{2cm}0.1\end{minipage} & \begin{minipage}[t]{12cm}for calorimetric analysis: values for neutrino energy; only work if calorimetric\_analysis is .true. set the min, max and bins for nu distributions\end{minipage}\\*
\bottomrule
\end{longtable}
{ }



% nl_dSigmadcostheta     code/init/neutrino/neutrinoSigma.f90

\begin{longtable}{llll}
\toprule
\textbf{\large{nl\_dSigmadcostheta}} & \multicolumn{3}{l}{\footnotesize{code/init/neutrino/neutrinoSigma.f90}}\\*
\midrule
\endfirsthead
\midrule
\endhead
enu & \begin{minipage}[t]{2cm}real\end{minipage} & \begin{minipage}[t]{2cm}-10.\end{minipage} & \begin{minipage}[t]{12cm}neutrino energy, read in by namelist\end{minipage}\\*
\midrule
costheta & \begin{minipage}[t]{2cm}real\end{minipage} & \begin{minipage}[t]{2cm}-10.\end{minipage} & \begin{minipage}[t]{12cm}cosine of the angle between the neutrino (z-direction) and the outgoing lepton\end{minipage}\\*
\midrule
delta\_costheta & \begin{minipage}[t]{2cm}real\end{minipage} & \begin{minipage}[t]{2cm}-10.\end{minipage} & \begin{minipage}[t]{12cm}value by which costheta is increased\end{minipage}\\*
\midrule
integralPrecision & \begin{minipage}[t]{2cm}integer\end{minipage} & \begin{minipage}[t]{2cm}3\end{minipage} & \begin{minipage}[t]{12cm}precision for the Gauss integration (reduce it for nuXsectionMode.eq.0 (sigma) to e.g. 2)\end{minipage}\\*
\midrule
integralPrecisionQE & \begin{minipage}[t]{2cm}integer\end{minipage} & \begin{minipage}[t]{2cm}500\end{minipage} & \begin{minipage}[t]{12cm}precision for the Gauss integration over the QE peak (reduce it for nuXsectionMode.eq.0 (sigma) to e.g. 300)\end{minipage}\\*
\bottomrule
\end{longtable}
{ }



% nl_dSigmadCosThetadElepton     code/init/neutrino/neutrinoSigma.f90

\begin{longtable}{llll}
\toprule
\textbf{\large{nl\_dSigmadCosThetadElepton}} & \multicolumn{3}{l}{\footnotesize{code/init/neutrino/neutrinoSigma.f90}}\\*
\midrule
\endfirsthead
\midrule
\endhead
enu & \begin{minipage}[t]{2cm}real\end{minipage} & \begin{minipage}[t]{2cm}-10.\end{minipage} & \begin{minipage}[t]{12cm}neutrino energy, read in by namelist\end{minipage}\\*
\midrule
costheta & \begin{minipage}[t]{2cm}real\end{minipage} & \begin{minipage}[t]{2cm}-10.\end{minipage} & \begin{minipage}[t]{12cm}cosine of the angle between the neutrino (z-direction) and the outgoing lepton\end{minipage}\\*
\midrule
elepton & \begin{minipage}[t]{2cm}real\end{minipage} & \begin{minipage}[t]{2cm}-10.\end{minipage} & \begin{minipage}[t]{12cm}energy of the outgoing lepton\end{minipage}\\*
\midrule
delta\_elepton & \begin{minipage}[t]{2cm}real\end{minipage} & \begin{minipage}[t]{2cm}-10.\end{minipage} & \begin{minipage}[t]{12cm}value by which elepton is increased\end{minipage}\\*
\bottomrule
\end{longtable}
{ }



% nl_dSigmadElepton     code/init/neutrino/neutrinoSigma.f90

\begin{longtable}{llll}
\toprule
\textbf{\large{nl\_dSigmadElepton}} & \multicolumn{3}{l}{\footnotesize{code/init/neutrino/neutrinoSigma.f90}}\\*
\midrule
\endfirsthead
\midrule
\endhead
enu & \begin{minipage}[t]{2cm}real\end{minipage} & \begin{minipage}[t]{2cm}-10.\end{minipage} & \begin{minipage}[t]{12cm}neutrino energy, read in by namelist\end{minipage}\\*
\midrule
elepton & \begin{minipage}[t]{2cm}real\end{minipage} & \begin{minipage}[t]{2cm}-10.\end{minipage} & \begin{minipage}[t]{12cm}energy of the outgoing lepton\end{minipage}\\*
\midrule
delta\_elepton & \begin{minipage}[t]{2cm}real\end{minipage} & \begin{minipage}[t]{2cm}-10.\end{minipage} & \begin{minipage}[t]{12cm}value by which elepton is increased\end{minipage}\\*
\midrule
integralPrecision & \begin{minipage}[t]{2cm}integer\end{minipage} & \begin{minipage}[t]{2cm}3\end{minipage} & \begin{minipage}[t]{12cm}precision for the Gauss integration (reduce it for nuXsectionMode.eq.0 (sigma) to e.g. 2)\end{minipage}\\*
\midrule
integralPrecisionQE & \begin{minipage}[t]{2cm}integer\end{minipage} & \begin{minipage}[t]{2cm}500\end{minipage} & \begin{minipage}[t]{12cm}precision for the Gauss integration over the QE peak (reduce it for nuXsectionMode.eq.0 (sigma) to e.g. 300)\end{minipage}\\*
\bottomrule
\end{longtable}
{ }



% nl_dSigmadQ2     code/init/neutrino/neutrinoSigma.f90

\begin{longtable}{llll}
\toprule
\textbf{\large{nl\_dSigmadQ2}} & \multicolumn{3}{l}{\footnotesize{code/init/neutrino/neutrinoSigma.f90}}\\*
\midrule
\endfirsthead
\midrule
\endhead
enu & \begin{minipage}[t]{2cm}real\end{minipage} & \begin{minipage}[t]{2cm}-10.\end{minipage} & \begin{minipage}[t]{12cm}neutrino energy, read in by namelist\end{minipage}\\*
\midrule
Q2 & \begin{minipage}[t]{2cm}real\end{minipage} & \begin{minipage}[t]{2cm}-10.\end{minipage} & \begin{minipage}[t]{12cm}momentum transfer squared\end{minipage}\\*
\midrule
delta\_Q2 & \begin{minipage}[t]{2cm}real\end{minipage} & \begin{minipage}[t]{2cm}-10.\end{minipage} & \begin{minipage}[t]{12cm}value by which Q2 is increased\end{minipage}\\*
\bottomrule
\end{longtable}
{ }



% nl_dSigmadQ2dElepton     code/init/neutrino/neutrinoSigma.f90

\begin{longtable}{llll}
\toprule
\textbf{\large{nl\_dSigmadQ2dElepton}} & \multicolumn{3}{l}{\footnotesize{code/init/neutrino/neutrinoSigma.f90}}\\*
\midrule
\endfirsthead
\midrule
\endhead
enu & \begin{minipage}[t]{2cm}real\end{minipage} & \begin{minipage}[t]{2cm}-10.\end{minipage} & \begin{minipage}[t]{12cm}neutrino energy, read in by namelist\end{minipage}\\*
\midrule
Q2 & \begin{minipage}[t]{2cm}real\end{minipage} & \begin{minipage}[t]{2cm}-10.\end{minipage} & \begin{minipage}[t]{12cm}momentum transfer squared\end{minipage}\\*
\midrule
elepton & \begin{minipage}[t]{2cm}real\end{minipage} & \begin{minipage}[t]{2cm}-10.\end{minipage} & \begin{minipage}[t]{12cm}energy of the outgoing lepton\end{minipage}\\*
\midrule
delta\_elepton & \begin{minipage}[t]{2cm}real\end{minipage} & \begin{minipage}[t]{2cm}-10.\end{minipage} & \begin{minipage}[t]{12cm}value by which elepton is increased\end{minipage}\\*
\bottomrule
\end{longtable}
{ }



% nl_dSigmadW     code/init/neutrino/neutrinoSigma.f90

\begin{longtable}{llll}
\toprule
\textbf{\large{nl\_dSigmadW}} & \multicolumn{3}{l}{\footnotesize{code/init/neutrino/neutrinoSigma.f90}}\\*
\midrule
\endfirsthead
\midrule
\endhead
enu & \begin{minipage}[t]{2cm}real\end{minipage} & \begin{minipage}[t]{2cm}-10.\end{minipage} & \begin{minipage}[t]{12cm}neutrino energy, read in by namelist\end{minipage}\\*
\midrule
W & \begin{minipage}[t]{2cm}real\end{minipage} & \begin{minipage}[t]{2cm}-10.\end{minipage} & \begin{minipage}[t]{12cm}invariant mass defined as (p+q)\^{}2\end{minipage}\\*
\midrule
delta\_W & \begin{minipage}[t]{2cm}real\end{minipage} & \begin{minipage}[t]{2cm}-10.\end{minipage} & \begin{minipage}[t]{12cm}value by which W is increased\end{minipage}\\*
\bottomrule
\end{longtable}
{ }



% nl_fluxcuts     code/init/neutrino/esample.f90

\begin{longtable}{llll}
\toprule
\textbf{\large{nl\_fluxcuts}} & \multicolumn{3}{l}{\footnotesize{code/init/neutrino/esample.f90}}\\*
\midrule
\endfirsthead
\midrule
\endhead
Enu\_lower\_cut & \begin{minipage}[t]{2cm}real\end{minipage} & \begin{minipage}[t]{2cm}0.\end{minipage} & \begin{minipage}[t]{12cm}cut events with neutrino energy below Enu\_lower\_cut; for ANL experiment, for example, Enu\_lower\_cut=0.5 for some analysis of ppi+\end{minipage}\\*
\midrule
Enu\_upper\_cut & \begin{minipage}[t]{2cm}real\end{minipage} & \begin{minipage}[t]{2cm}200.\end{minipage} & \begin{minipage}[t]{12cm}cut events with neutrino energy above Enu\_upper\_cut; for ANL experiment, for example, Enu\_upper\_cut=1.5 for ppi0 and npi+ final state, but 5.98 for ppi+\end{minipage}\\*
\midrule
energylimit\_for\_Qsrec & \begin{minipage}[t]{2cm}logical\end{minipage} & \begin{minipage}[t]{2cm}.false.\end{minipage} & \begin{minipage}[t]{12cm}switch for using the energylimits Enu\_upper\_cut and Enu\_lower\_cut for the Q\^{}2 reconstruction; values: .true. or .false.  (default: .false.)\end{minipage}\\*
\bottomrule
\end{longtable}
{ }



% nl_integratedSigma     code/init/neutrino/neutrinoSigma.f90

\begin{longtable}{llll}
\toprule
\textbf{\large{nl\_integratedSigma}} & \multicolumn{3}{l}{\footnotesize{code/init/neutrino/neutrinoSigma.f90}}\\*
\midrule
\endfirsthead
\midrule
\endhead
enu & \begin{minipage}[t]{2cm}real\end{minipage} & \begin{minipage}[t]{2cm}-10.\end{minipage} & \begin{minipage}[t]{12cm}neutrino energy, read in by namelist\end{minipage}\\*
\midrule
delta\_enu & \begin{minipage}[t]{2cm}real\end{minipage} & \begin{minipage}[t]{2cm}-10.\end{minipage} & \begin{minipage}[t]{12cm}value by which the neutrino energy is increased\end{minipage}\\*
\midrule
integralPrecision & \begin{minipage}[t]{2cm}integer\end{minipage} & \begin{minipage}[t]{2cm}3\end{minipage} & \begin{minipage}[t]{12cm}precision for the Gauss integration (reduce it for nuXsectionMode.eq.0 (sigma) to e.g. 2)\end{minipage}\\*
\midrule
integralPrecisionQE & \begin{minipage}[t]{2cm}integer\end{minipage} & \begin{minipage}[t]{2cm}500\end{minipage} & \begin{minipage}[t]{12cm}precision for the Gauss integration over the QE peak (reduce it for nuXsectionMode.eq.0 (sigma) to e.g. 300)\end{minipage}\\*
\bottomrule
\end{longtable}
{ }



% nl_Neutrino2piBack     code/init/neutrino/neutrinoParms.f90

\begin{longtable}{llll}
\toprule
\textbf{\large{nl\_Neutrino2piBack}} & \multicolumn{3}{l}{\footnotesize{code/init/neutrino/neutrinoParms.f90}}\\*
\midrule
\endfirsthead
\midrule
\endhead
Wtrans & \begin{minipage}[t]{2cm}real\end{minipage} & \begin{minipage}[t]{2cm}2.7\end{minipage} & \begin{minipage}[t]{12cm}W for transition from Bosted-Christy Parametrization to PYTHIA DIS\end{minipage}\\*
\midrule
NormpiBG & \begin{minipage}[t]{2cm}real\end{minipage} & \begin{minipage}[t]{2cm}1.0\end{minipage} & \begin{minipage}[t]{12cm}overall normalization factor for pi BG and Bloom-Gilman X-section, only relevant for neutrinos\end{minipage}\\*
\midrule
normRES & \begin{minipage}[t]{2cm}\end{minipage} & \begin{minipage}[t]{2cm}\end{minipage} & \begin{minipage}[t]{12cm}\end{minipage}\\*
\midrule
normBC & \begin{minipage}[t]{2cm}real\end{minipage} & \begin{minipage}[t]{2cm}1.0\end{minipage} & \begin{minipage}[t]{12cm}overall normalization factor for neutrino-induced Christy-Bosted contributions between 2 GeV and DIS onset only relevant for neutrinos\end{minipage}\\*
\bottomrule
\end{longtable}
{ }



% nl_neutrino_energyFlux     code/init/neutrino/expNeutrinofluxes.f90

\begin{longtable}{llll}
\toprule
\textbf{\large{nl\_neutrino\_energyFlux}} & \multicolumn{3}{l}{\footnotesize{code/init/neutrino/expNeutrinofluxes.f90}}\\*
\midrule
\endfirsthead
\midrule
\endhead
Eb & \begin{minipage}[t]{2cm}real\end{minipage} & \begin{minipage}[t]{2cm}0.034\end{minipage} & \begin{minipage}[t]{12cm}contant binding energy used for energy and Q2 reconstruction based on QE scattering kinematics\end{minipage}\\*
\midrule
Eflux\_min & \begin{minipage}[t]{2cm}real\end{minipage} & \begin{minipage}[t]{2cm}0.2\end{minipage} & \begin{minipage}[t]{12cm}minimum energy for uniform flux distribution\\ minimum and maximum energies for the uniform neutrino flux (nuExp=10 in the namelist neutrino\_induced) can be changed in the namelist nl\_neutrino\_energyFlux\end{minipage}\\*
\midrule
Eflux\_max & \begin{minipage}[t]{2cm}real\end{minipage} & \begin{minipage}[t]{2cm}2.5\end{minipage} & \begin{minipage}[t]{12cm}maximum energy for uniform flux distribution\\ minimum and maximum energies for the uniform neutrino flux (nuExp=10 in the namelist neutrino\_induced) can be changed in the namelist nl\_neutrino\_energyFlux\end{minipage}\\*
\bottomrule
\end{longtable}
{ }



% nl_neutrinoxsection     code/init/neutrino/neutrinoXsection.f90

\begin{longtable}{llll}
\toprule
\textbf{\large{nl\_neutrinoxsection}} & \multicolumn{3}{l}{\footnotesize{code/init/neutrino/neutrinoXsection.f90}}\\*
\midrule
\endfirsthead
\midrule
\endhead
singlePiModel & \begin{minipage}[t]{2cm}integer\end{minipage} & \begin{minipage}[t]{2cm}1\end{minipage} & \begin{minipage}[t]{12cm}to change between different models for the pion nucleon cross section:\begin{itemize}\leftmargin0em\itemindent0pt\item 0 = pi N according to Nieves et al (hep-ph/0701149)\item 1 = MAID-like model\item 2 = Bosted-Christy\end{itemize}\end{minipage}\\*
\midrule
invariantMassCut & \begin{minipage}[t]{2cm}real\end{minipage} & \begin{minipage}[t]{2cm}100.\end{minipage} & \begin{minipage}[t]{12cm}cut events with invariant Mass above this value (in GeV); cut pion production from Delta and DIS on Wrec = Sqrt[M\^{}2 + 2*M*nu - Q\^{}2]\end{minipage}\\*
\midrule
invariantMassCut\_BG & \begin{minipage}[t]{2cm}real\end{minipage} & \begin{minipage}[t]{2cm}100.\end{minipage} & \begin{minipage}[t]{12cm}cut MAID-like background events with invariantMass\_BG above this value (in GeV); cut 1pi BG on Wrec = Sqrt[M\^{}2 + 2*M*nu - Q\^{}2]\end{minipage}\\*
\midrule
DIScutW & \begin{minipage}[t]{2cm}real\end{minipage} & \begin{minipage}[t]{2cm}3.0\end{minipage} & \begin{minipage}[t]{12cm}W-cut for sigmoid onset of DIS\end{minipage}\\*
\midrule
DIScutwidth & \begin{minipage}[t]{2cm}real\end{minipage} & \begin{minipage}[t]{2cm}0.2\end{minipage} & \begin{minipage}[t]{12cm}width for sigmoid onset of DIS\end{minipage}\\*
\midrule
REScutW & \begin{minipage}[t]{2cm}real\end{minipage} & \begin{minipage}[t]{2cm}2.0\end{minipage} & \begin{minipage}[t]{12cm}W-cut for end of 1pi,2pi BGs\end{minipage}\\*
\midrule
DISformfakEM & \begin{minipage}[t]{2cm}integer\end{minipage} & \begin{minipage}[t]{2cm}2\end{minipage} & \begin{minipage}[t]{12cm}Introduce an additional form factor for the DIS cross section, when processed via a photon:\begin{itemize}\leftmargin0em\itemindent0pt\item 0: no form factor\item 1: Q\^{}2/(mcutDIS\^{}2+Q\^{}2)\item 2: Q\^{}4/(mcutDIS\^{}2+Q\^{}2)\^{}2\end{itemize} In case of electron induced events, we need choose 2 in order to be compatible with Pythia's electron machinery.\end{minipage}\\*
\midrule
DISformfakNCCC & \begin{minipage}[t]{2cm}integer\end{minipage} & \begin{minipage}[t]{2cm}1\end{minipage} & \begin{minipage}[t]{12cm}Introduce an additional form factor for the DIS cross section, when processed via W or Z boson:\begin{itemize}\leftmargin0em\itemindent0pt\item 0: no form factor\item 1: Q\^{}2/(mcutDIS\^{}2+Q\^{}2)\item 2: Q\^{}4/(mcutDIS\^{}2+Q\^{}2)\^{}2\end{itemize} In case of electron induced events, we need choose 2 in order to be compatible with Pythia's electron machinery.\end{minipage}\\*
\midrule
mcutDIS & \begin{minipage}[t]{2cm}real\end{minipage} & \begin{minipage}[t]{2cm}0.6\end{minipage} & \begin{minipage}[t]{12cm}parameter to control Q\^{}2 dependence of DIS\end{minipage}\\*
\midrule
DISrespectHad & \begin{minipage}[t]{2cm}logical\end{minipage} & \begin{minipage}[t]{2cm}.true.\end{minipage} & \begin{minipage}[t]{12cm}Flag to indicate, whether hadronization failures should be respected and affect the overall DIS cross section\\ Pythia is run to generate the DIS cross section. But not every of the generated events may lead to a correct hadronic final state.\end{minipage}\\*
\midrule
DISdoMSTP23 & \begin{minipage}[t]{2cm}logical\end{minipage} & \begin{minipage}[t]{2cm}.true.\end{minipage} & \begin{minipage}[t]{12cm}Flag to indicate, whether in Pythia for neutrino-DIS the value MSTP(23)=1 should be used or not\end{minipage}\\*
\midrule
new2piBG & \begin{minipage}[t]{2cm}logical\end{minipage} & \begin{minipage}[t]{2cm}.true.\end{minipage} & \begin{minipage}[t]{12cm}Flag to turn on the new treatment of 2pi BG for electrons and neutrinos\end{minipage}\\*
\midrule
indBG & \begin{minipage}[t]{2cm}integer\end{minipage} & \begin{minipage}[t]{2cm}3\end{minipage} & \begin{minipage}[t]{12cm}Index to choose Bloom-Gilman like BG parametrization 1 : original Bloom Gilman 2 : Niculescu fit 3 : nonresonant BG fit from Christy-Bosted\end{minipage}\\*
\bottomrule
\end{longtable}
{ }



% nl_neweN     code/init/neutrino/neutrinoParms.f90

\begin{longtable}{llll}
\toprule
\textbf{\large{nl\_neweN}} & \multicolumn{3}{l}{\footnotesize{code/init/neutrino/neutrinoParms.f90}}\\*
\midrule
\endfirsthead
\midrule
\endhead
ME\_Version & \begin{minipage}[t]{2cm}integer\end{minipage} & \begin{minipage}[t]{2cm}6\end{minipage} & \begin{minipage}[t]{12cm}indicate the type of matrix element parametrisation\\NOTES\\ possible values:\begin{itemize}\leftmargin0em\itemindent0pt\item 1: const ME\_Norm\_XX  ! const for CC  fitted to MiniBooNE is 1.8e-6\item 2: constant transverse and decreasing with Enu\item 3: "Dipole transverse" transverse, fall with Q2 as 4-th power\item 4: MEC from E. Christy (8/2015), with parametrization for longitudinal\item 5: MEC from Bosted arXiV:1203.2262, with parametrization for longitudinal\item 6: MEC additional parametrization, with parametrization for longitudinal   not yet implemented\end{itemize} remarks:\begin{itemize}\leftmargin0em\itemindent0pt\item case 1 is model-I in Lalakulich,Gallmeister,Mosel PRC86(2012)014614\item case 2 is model-II from Lalakulich,Gallmeister,Mosel PRC86(2012)014614\item case 3 gives a good description of MiniBooNE data with MA \~{} 1.5 GeV\end{itemize}\end{minipage}\\*
\midrule
ME\_Norm\_QE & \begin{minipage}[t]{2cm}real, dimension(1:3)\end{minipage} & \begin{minipage}[t]{2cm}(/1.0, 1.0, 1.0/)\end{minipage} & \begin{minipage}[t]{12cm}Overall strength of 2p2h matrix element with 2N out\\ for (EM,CC,NC)\\NOTES\\ The value == 1 gives the coded strength\end{minipage}\\*
\midrule
ME\_Norm\_Delta & \begin{minipage}[t]{2cm}real, dimension(1:3)\end{minipage} & \begin{minipage}[t]{2cm}(/1.0, 1.0, 1.0/)\end{minipage} & \begin{minipage}[t]{12cm}Overall strength of 2p2h matrix element with NDelta out\\ for (EM,CC,NC)\\NOTES\\ The value == 1 is a dummy value\end{minipage}\\*
\midrule
ME\_Mass\_QE & \begin{minipage}[t]{2cm}real, dimension(1:3)\end{minipage} & \begin{minipage}[t]{2cm}(/1.0, 1.0, 1.0/)\end{minipage} & \begin{minipage}[t]{12cm}Cutoff-mass in some parametrizations of 2p2h matrix element for NN out\\ for (EM,CC,NC)\\NOTES\\ The value == 1 is a dummy value\end{minipage}\\*
\midrule
ME\_Mass\_Delta & \begin{minipage}[t]{2cm}real, dimension(1:3)\end{minipage} & \begin{minipage}[t]{2cm}(/1.0, 1.0, 1.0/)\end{minipage} & \begin{minipage}[t]{12cm}Cutoff-mass in some parametrizations of matrix element for NDelta out\\ for (EM,CC,NC)\\NOTES\\ The value == 1 is a dummy value\end{minipage}\\*
\midrule
ME\_Transversity & \begin{minipage}[t]{2cm}real, dimension(1:3)\end{minipage} & \begin{minipage}[t]{2cm}(/1.0, 1.0, 1.0/)\end{minipage} & \begin{minipage}[t]{12cm}Parametrisation of structure functions\\ for (EM,CC,NC)\\NOTES\\ The value = 1 chooses structure function W2 so that 2p2h is pure transverse\end{minipage}\\*
\midrule
ME\_LONG & \begin{minipage}[t]{2cm}real, dimension(1:3)\end{minipage} & \begin{minipage}[t]{2cm}(/0.0, 0.0, 0.0/)\end{minipage} & \begin{minipage}[t]{12cm}Parametrization of structure functions\\ for (EM,CC,NC)\\NOTES\\ The value = 0 turns any additional longitudinal contribution to structure funct. W2 off\end{minipage}\\*
\midrule
ME\_W3 & \begin{minipage}[t]{2cm}real, dimension(1:3)\end{minipage} & \begin{minipage}[t]{2cm}(/0.0, 1.0, 1.0/)\end{minipage} & \begin{minipage}[t]{12cm}Overall strength factor for structure function W3\\ only for (CC,NC)\\NOTES\\ overall strength parameter for structure function W3\end{minipage}\\*
\midrule
inmedW & \begin{minipage}[t]{2cm}integer\end{minipage} & \begin{minipage}[t]{2cm}1\end{minipage} & \begin{minipage}[t]{12cm}Controls which inv mass W is used in parametrization of 2p2h W1\\NOTES\begin{itemize}\leftmargin0em\itemindent0pt\item 1: W = static inv. mass in 2p2h parametrization of W1\item 2: W = inv mass for Fermi moving nucleons in potential\item 3: W = inv mass for Fermi moving nucleons without potential\end{itemize}\end{minipage}\\*
\midrule
Adep & \begin{minipage}[t]{2cm}integer\end{minipage} & \begin{minipage}[t]{2cm}2\end{minipage} & \begin{minipage}[t]{12cm}Switch for A-dependence of 2p2h structure function\\NOTES\begin{itemize}\leftmargin0em\itemindent0pt\item 1: A-dependence for zero-range force (Mosel, Gallmeister, 2016)\item 2: linear A-dependence, normalized to C12\end{itemize}\end{minipage}\\*
\bottomrule
\end{longtable}
{ }



% nl_NievesHadronTensor     code/init/neutrino/NievesHadronTensor.f90

\begin{longtable}{llll}
\toprule
\textbf{\large{nl\_NievesHadronTensor}} & \multicolumn{3}{l}{\footnotesize{code/init/neutrino/NievesHadronTensor.f90}}\\*
\midrule
\endfirsthead
\midrule
\endhead
DeltaPole & \begin{minipage}[t]{2cm}logical\end{minipage} & \begin{minipage}[t]{2cm}.true.\end{minipage} & \begin{minipage}[t]{12cm}\end{minipage}\\*
\midrule
crossedDelta & \begin{minipage}[t]{2cm}logical\end{minipage} & \begin{minipage}[t]{2cm}.true.\end{minipage} & \begin{minipage}[t]{12cm}\end{minipage}\\*
\midrule
nucleonPole & \begin{minipage}[t]{2cm}logical\end{minipage} & \begin{minipage}[t]{2cm}.true.\end{minipage} & \begin{minipage}[t]{12cm}\end{minipage}\\*
\midrule
crossedNucleonPole & \begin{minipage}[t]{2cm}logical\end{minipage} & \begin{minipage}[t]{2cm}.true.\end{minipage} & \begin{minipage}[t]{12cm}\end{minipage}\\*
\midrule
contactTerm & \begin{minipage}[t]{2cm}logical\end{minipage} & \begin{minipage}[t]{2cm}.true.\end{minipage} & \begin{minipage}[t]{12cm}\end{minipage}\\*
\midrule
pionPole & \begin{minipage}[t]{2cm}logical\end{minipage} & \begin{minipage}[t]{2cm}.true.\end{minipage} & \begin{minipage}[t]{12cm}\end{minipage}\\*
\midrule
pionInFlight & \begin{minipage}[t]{2cm}logical\end{minipage} & \begin{minipage}[t]{2cm}.true.\end{minipage} & \begin{minipage}[t]{12cm}\end{minipage}\\*
\bottomrule
\end{longtable}
{ }



% nl_SigmaMC     code/init/neutrino/neutrinoSigma.f90

\begin{longtable}{llll}
\toprule
\textbf{\large{nl\_SigmaMC}} & \multicolumn{3}{l}{\footnotesize{code/init/neutrino/neutrinoSigma.f90}}\\*
\midrule
\endfirsthead
\midrule
\endhead
enu & \begin{minipage}[t]{2cm}real\end{minipage} & \begin{minipage}[t]{2cm}-10.\end{minipage} & \begin{minipage}[t]{12cm}neutrino energy, read in by namelist\end{minipage}\\*
\midrule
MC\_xmax & \begin{minipage}[t]{2cm}real\end{minipage} & \begin{minipage}[t]{2cm}2.0\end{minipage} & \begin{minipage}[t]{12cm}\end{minipage}\\*
\bottomrule
\end{longtable}
{ }



% nl_singlePionProductionNHVlike     code/init/neutrino/singlePionProductionNHVlike.f90

\begin{longtable}{llll}
\toprule
\textbf{\large{nl\_singlePionProductionNHVlike}} & \multicolumn{3}{l}{\footnotesize{code/init/neutrino/singlePionProductionNHVlike.f90}}\\*
\midrule
\endfirsthead
\midrule
\endhead
integrate\_over & \begin{minipage}[t]{2cm}integer\end{minipage} & \begin{minipage}[t]{2cm}2\end{minipage} & \begin{minipage}[t]{12cm}possible values:\begin{itemize}\leftmargin0em\itemindent0pt\item 1 = costhetaPi\item 2 = Epi\item 3 = over cosThetaPi\_star\_qz in CM frame\end{itemize} which 3-pl differential cross sectio to use for integration:\begin{itemize}\leftmargin0em\itemindent0pt\item 1= dsigma/dcostheta/dElepton/dcosThetaPion was originally used and   works for nuclei.   disadvantage: for some cosThetaPion there are two solutions for Epi,   this  leads to fluctuations on the cross section\item 2= dsigma/dcostheta/dElepton/dEPion  has an advantage, that for a given pion energy there is   only one solution for  the angle between the resonance and pion momenta.   so the integration is simpler and results should be smoother\end{itemize}NOTES\begin{itemize}\leftmargin0em\itemindent0pt\item for 1 : the only option checked for nucleus\item for 2 : code works better and faster,   gives significantly smoother results below Delta peak.   disadvantage: now for the free nucleon only, TO DO : nuclei\end{itemize}\end{minipage}\\*
\bottomrule
\end{longtable}
{ }



% nl_specificEvent     code/analysis/neutrinoAnalysis.f90

\begin{longtable}{llll}
\toprule
\textbf{\large{nl\_specificEvent}} & \multicolumn{3}{l}{\footnotesize{code/analysis/neutrinoAnalysis.f90}}\\*
\midrule
\endfirsthead
\midrule
\endhead
no\_pi & \begin{minipage}[t]{2cm}logical\end{minipage} & \begin{minipage}[t]{2cm}.false.\end{minipage} & \begin{minipage}[t]{12cm}do analysis for specific final states: specificEvent=1, no\_pi (for example, for QE-like MiniBooNE)\end{minipage}\\*
\midrule
p\_Xn\_no\_pi & \begin{minipage}[t]{2cm}logical\end{minipage} & \begin{minipage}[t]{2cm}.false.\end{minipage} & \begin{minipage}[t]{12cm}do analysis for specific final states: specificEvent=2\end{minipage}\\*
\midrule
piplus & \begin{minipage}[t]{2cm}logical\end{minipage} & \begin{minipage}[t]{2cm}.false.\end{minipage} & \begin{minipage}[t]{12cm}do analysis for specific final states: specificEvent=3, 1 pi+ X nucleons mesons of other flavor\end{minipage}\\*
\midrule
piplus\_MULTI & \begin{minipage}[t]{2cm}logical\end{minipage} & \begin{minipage}[t]{2cm}.false.\end{minipage} & \begin{minipage}[t]{12cm}do analysis for specific final states: specificEvent=4 $>$=1 pi+  X other pions (incl pi+) X nucleons\end{minipage}\\*
\midrule
pi0 & \begin{minipage}[t]{2cm}logical\end{minipage} & \begin{minipage}[t]{2cm}.false.\end{minipage} & \begin{minipage}[t]{12cm}do analysis for specific final states: specificEvent=5, 1 pi0 X nucleons, plus mesons of other flavor\end{minipage}\\*
\midrule
pi0\_MULTI & \begin{minipage}[t]{2cm}logical\end{minipage} & \begin{minipage}[t]{2cm}.false.\end{minipage} & \begin{minipage}[t]{12cm}do analysis for specific final states: specificEvent=6, $>$=1 pi0  X other pions X nucleons, (pi0 K2K)\end{minipage}\\*
\midrule
piminus & \begin{minipage}[t]{2cm}logical\end{minipage} & \begin{minipage}[t]{2cm}.false.\end{minipage} & \begin{minipage}[t]{12cm}do analysis for specific final states: specificEvent=7 1 pi-  X other pions X nucleons\end{minipage}\\*
\midrule
piminus\_MULTI & \begin{minipage}[t]{2cm}logical\end{minipage} & \begin{minipage}[t]{2cm}.false.\end{minipage} & \begin{minipage}[t]{12cm}do analysis for specific final states: specificEvent=8 $>$=1 pi-  X other pions X nucleons\end{minipage}\\*
\midrule
pp\_no\_pi & \begin{minipage}[t]{2cm}logical\end{minipage} & \begin{minipage}[t]{2cm}.false.\end{minipage} & \begin{minipage}[t]{12cm}do analysis for specific final states:  specificEvent=9 2 protons, X neutrons, 0 pions\end{minipage}\\*
\midrule
pn\_no\_pi & \begin{minipage}[t]{2cm}logical\end{minipage} & \begin{minipage}[t]{2cm}.false.\end{minipage} & \begin{minipage}[t]{12cm}do analysis for specific final states: specificEvent=10 1 neutron, 1 proton, 0 pions\end{minipage}\\*
\midrule
nn\_no\_pi & \begin{minipage}[t]{2cm}logical\end{minipage} & \begin{minipage}[t]{2cm}.false.\end{minipage} & \begin{minipage}[t]{12cm}do analysis for specific final states: specificEvent=11 2 neutrons, X protons, 0 pions\end{minipage}\\*
\midrule
pp\_Xn\_no\_pi & \begin{minipage}[t]{2cm}logical\end{minipage} & \begin{minipage}[t]{2cm}.false.\end{minipage} & \begin{minipage}[t]{12cm}do analysis for specific final states: specificEvent=12 2 protons, X neutrons, 0 pions\end{minipage}\\*
\midrule
nn\_Xp\_no\_pi & \begin{minipage}[t]{2cm}logical\end{minipage} & \begin{minipage}[t]{2cm}.false.\end{minipage} & \begin{minipage}[t]{12cm}do analysis for specific final states: specificEvent=13 2 neutrons, X protons, 0 pions\end{minipage}\\*
\midrule
ppp\_Xn\_no\_pi & \begin{minipage}[t]{2cm}logical\end{minipage} & \begin{minipage}[t]{2cm}.false.\end{minipage} & \begin{minipage}[t]{12cm}do analysis for specific final states: specificEvent=14 3 protons, X neutrons, 0 pions\end{minipage}\\*
\midrule
pppp\_Xn\_no\_pi & \begin{minipage}[t]{2cm}logical\end{minipage} & \begin{minipage}[t]{2cm}.false.\end{minipage} & \begin{minipage}[t]{12cm}do analysis for specific final states: specificEvent=15 4 protons, X neutrons, 0 pions\end{minipage}\\*
\midrule
p\_no\_pi & \begin{minipage}[t]{2cm}logical\end{minipage} & \begin{minipage}[t]{2cm}.false.\end{minipage} & \begin{minipage}[t]{12cm}do analysis for specific final states: specificEvent=16 1 proton, 0 neutron, 0 pion\end{minipage}\\*
\midrule
n\_no\_pi & \begin{minipage}[t]{2cm}logical\end{minipage} & \begin{minipage}[t]{2cm}.false.\end{minipage} & \begin{minipage}[t]{12cm}do analysis for specific final states: specificEvent=17 1 neutron, 0 proton, 0 pion\end{minipage}\\*
\midrule
Xn\_no\_pi & \begin{minipage}[t]{2cm}logical\end{minipage} & \begin{minipage}[t]{2cm}.false.\end{minipage} & \begin{minipage}[t]{12cm}do analysis for specific final states: specificEvent=18, 0 proton, X neutrons, 0 pions\end{minipage}\\*
\midrule
excl\_pi0 & \begin{minipage}[t]{2cm}\end{minipage} & \begin{minipage}[t]{2cm}\end{minipage} & \begin{minipage}[t]{12cm}\end{minipage}\\*
\midrule
excl\_piplus & \begin{minipage}[t]{2cm}\end{minipage} & \begin{minipage}[t]{2cm}\end{minipage} & \begin{minipage}[t]{12cm}\end{minipage}\\*
\midrule
excl\_piminus & \begin{minipage}[t]{2cm}\end{minipage} & \begin{minipage}[t]{2cm}\end{minipage} & \begin{minipage}[t]{12cm}\end{minipage}\\*
\midrule
full\_incl & \begin{minipage}[t]{2cm}logical\end{minipage} & \begin{minipage}[t]{2cm}.true.\end{minipage} & \begin{minipage}[t]{12cm}do analysis for specific final states: specificEvent=22 fully inclusive event, all hadrons in final state\end{minipage}\\*
\midrule
binsizeQ2 & \begin{minipage}[t]{2cm}real\end{minipage} & \begin{minipage}[t]{2cm}0.01\end{minipage} & \begin{minipage}[t]{12cm}do analysis for specific final states: binning for reconstruction of Q2 and Enu\end{minipage}\\*
\midrule
binsizeEnu & \begin{minipage}[t]{2cm}real\end{minipage} & \begin{minipage}[t]{2cm}0.02\end{minipage} & \begin{minipage}[t]{12cm}do analysis for specific final states: binning for reconstruction of Q2 and Enu\end{minipage}\\*
\midrule
maxQ2 & \begin{minipage}[t]{2cm}real\end{minipage} & \begin{minipage}[t]{2cm}5.0\end{minipage} & \begin{minipage}[t]{12cm}do analysis for specific final states: max values for reconstruction of Q2 and Enu\end{minipage}\\*
\midrule
maxEnu & \begin{minipage}[t]{2cm}real\end{minipage} & \begin{minipage}[t]{2cm}5.0\end{minipage} & \begin{minipage}[t]{12cm}do analysis for specific final states: max values for reconstruction of Q2 and Enu\end{minipage}\\*
\midrule
excl\_hadron & \begin{minipage}[t]{2cm}logical\end{minipage} & \begin{minipage}[t]{2cm}.false.\end{minipage} & \begin{minipage}[t]{12cm}do analysis for specific final states: specificEvent=19,20,21 exclusive 1 pion, no other pions or other mesons of different flavor There could be still other mesons which are heavier than the D, Such events (very rare at DUNE energies) could be counted as exclusive single-meson cross section. This could be cured by extending the list of stable mesons\end{minipage}\\*
\midrule
QEp & \begin{minipage}[t]{2cm}logical\end{minipage} & \begin{minipage}[t]{2cm}.false.\end{minipage} & \begin{minipage}[t]{12cm}if .true, do analysis for specific analysis for QE-like event with 1 mu, 0 pi, X p\end{minipage}\\*
\bottomrule
\end{longtable}
{ }



% OffShellPotential     code/width/offShellPotential.f90

\begin{longtable}{llll}
\toprule
\textbf{\large{OffShellPotential}} & \multicolumn{3}{l}{\footnotesize{code/width/offShellPotential.f90}}\\*
\midrule
\endfirsthead
\midrule
\endhead
useOffShellPotentialBaryons & \begin{minipage}[t]{2cm}logical\end{minipage} & \begin{minipage}[t]{2cm}.false.\end{minipage} & \begin{minipage}[t]{12cm}Switch on or off whether the offshellness should be used for baryons.\\NOTES\begin{itemize}\leftmargin0em\itemindent0pt\item must be set to "TRUE" if mediumSwitch\_coll   (see module BaryonWidthMedium) is .true.\item if .true. then delta\_T (see module inputGeneral) must be $<$=0.05   AND delta\_P (see module propagation)  must be $<$=0.002;   AND delta\_E (see module propagation)  must be $<$=0.002;   slows down propagation by a factor of 10\end{itemize}\end{minipage}\\*
\midrule
useOffShellPotentialMesons & \begin{minipage}[t]{2cm}logical\end{minipage} & \begin{minipage}[t]{2cm}.false.\end{minipage} & \begin{minipage}[t]{12cm}Switch on or off whether the offshellness should be used for mesons.\end{minipage}\\*
\midrule
extrapolateBaryonWidth & \begin{minipage}[t]{2cm}logical\end{minipage} & \begin{minipage}[t]{2cm}.true.\end{minipage} & \begin{minipage}[t]{12cm}Whether to extrapolate the baryon width below minimal mass or not.\end{minipage}\\*
\midrule
max\_offshellparameter & \begin{minipage}[t]{2cm}real\end{minipage} & \begin{minipage}[t]{2cm}5.\end{minipage} & \begin{minipage}[t]{12cm}The maximal value for the offshell parameter. Note: empirical value! This only applies to baryons. For mesons we have no restrictions on the offshell parameter.\end{minipage}\\*
\midrule
relativistic & \begin{minipage}[t]{2cm}logical\end{minipage} & \begin{minipage}[t]{2cm}.false.\end{minipage} & \begin{minipage}[t]{12cm}\begin{itemize}\leftmargin0em\itemindent0pt\item false: Use non-rel. off-shell parameter x=Delta m/Gamma,   which obeys Stefan Leupold's non-rel. EOM.\item true: Use rel. off-shell parameter x=Delta m\^{}2/Gamma,   which obeys Cassing's rel. EOM.\end{itemize}\end{minipage}\\*
\midrule
SetOffShellEnergyFlag & \begin{minipage}[t]{2cm}logical\end{minipage} & \begin{minipage}[t]{2cm}.false.\end{minipage} & \begin{minipage}[t]{12cm}\begin{itemize}\leftmargin0em\itemindent0pt\item false: the energy of off-shell particle is constant during time evolution          (static nucleus)\item true: the energy of off-shell particle varies during time evolution          (dynamic case, e.g. heavy ion collision)\end{itemize}\end{minipage}\\*
\bottomrule
\end{longtable}
{ }



% paramEP     code/init/lowElectron/ParamEP.f90

\begin{longtable}{llll}
\toprule
\textbf{\large{paramEP}} & \multicolumn{3}{l}{\footnotesize{code/init/lowElectron/ParamEP.f90}}\\*
\midrule
\endfirsthead
\midrule
\endhead
useParam & \begin{minipage}[t]{2cm}integer\end{minipage} & \begin{minipage}[t]{2cm}2\end{minipage} & \begin{minipage}[t]{12cm}select, which parametrization to use when CalcParamEP is called:\begin{itemize}\leftmargin0em\itemindent0pt\item 1: Brasse\item 2: Bosted\end{itemize}\end{minipage}\\*
\bottomrule
\end{longtable}
{ }



% photonXS     code/collisions/twoBodyReactions/HiEnergy/photonXS.f90

\begin{longtable}{llll}
\toprule
\textbf{\large{photonXS}} & \multicolumn{3}{l}{\footnotesize{code/collisions/twoBodyReactions/HiEnergy/photonXS.f90}}\\*
\midrule
\endfirsthead
\midrule
\endhead
iParam & \begin{minipage}[t]{2cm}integer\end{minipage} & \begin{minipage}[t]{2cm}2\end{minipage} & \begin{minipage}[t]{12cm}Switch to select the kind of parametrization for gamma N $\rightarrow$ V N:\begin{itemize}\leftmargin0em\itemindent0pt\item 1: "old parametrization", fit to experimental data, cf. Effenberger PhD, p.53\item 2: Pythia, cf. Friberg/Sjöstrand hep-ph/0007314\item 3: Donnachie, Landshoff [citation needed]\end{itemize}\end{minipage}\\*
\midrule
omega\_saphir & \begin{minipage}[t]{2cm}logical\end{minipage} & \begin{minipage}[t]{2cm}.true.\end{minipage} & \begin{minipage}[t]{12cm}If .true. an improved fit (to SAPHIR data) will be used for gamma N $\rightarrow$ omega N. cf. "calcXS\_omega\_saphir"\end{minipage}\\*
\bottomrule
\end{longtable}
{ }



% pionAnalysis     code/analysis/pionXsection.f90

\begin{longtable}{llll}
\toprule
\textbf{\large{pionAnalysis}} & \multicolumn{3}{l}{\footnotesize{code/analysis/pionXsection.f90}}\\*
\midrule
\endfirsthead
\midrule
\endhead
CMFrame & \begin{minipage}[t]{2cm}logical\end{minipage} & \begin{minipage}[t]{2cm}.false.\end{minipage} & \begin{minipage}[t]{12cm}If .true. Xsection is evaluated in CM-Frame of the incoming pion and a resting nucleon, else in calculation frame.\end{minipage}\\*
\midrule
dsigma\_dOmegadE\_switch & \begin{minipage}[t]{2cm}logical\end{minipage} & \begin{minipage}[t]{2cm}.false.\end{minipage} & \begin{minipage}[t]{12cm}If .true. then dsigma/dOmega and dSigma/dOmega/dE are evaluated.\end{minipage}\\*
\midrule
twoPi\_switch & \begin{minipage}[t]{2cm}logical\end{minipage} & \begin{minipage}[t]{2cm}.false.\end{minipage} & \begin{minipage}[t]{12cm}If .true. then 2Pi output is evaluated.\end{minipage}\\*
\bottomrule
\end{longtable}
{ }



% pionNucleus     code/init/initPion.f90

\begin{longtable}{llll}
\toprule
\textbf{\large{pionNucleus}} & \multicolumn{3}{l}{\footnotesize{code/init/initPion.f90}}\\*
\midrule
\endfirsthead
\midrule
\endhead
UseCoulomb & \begin{minipage}[t]{2cm}logical\end{minipage} & \begin{minipage}[t]{2cm}.false.\end{minipage} & \begin{minipage}[t]{12cm}if .true. then a Coulomb propagation from CoulombDistance to distance is performed\end{minipage}\\*
\midrule
CoulombDistance & \begin{minipage}[t]{2cm}real\end{minipage} & \begin{minipage}[t]{2cm}200.\end{minipage} & \begin{minipage}[t]{12cm}distance from where the Coulomb propagation starts\end{minipage}\\*
\midrule
distance & \begin{minipage}[t]{2cm}real\end{minipage} & \begin{minipage}[t]{2cm}15.\end{minipage} & \begin{minipage}[t]{12cm}initialization distance\end{minipage}\\*
\midrule
impact\_parameter & \begin{minipage}[t]{2cm}real\end{minipage} & \begin{minipage}[t]{2cm}0.\end{minipage} & \begin{minipage}[t]{12cm}impact parameter. If less than 0, than an impact parameter integration is performed\end{minipage}\\*
\midrule
charge & \begin{minipage}[t]{2cm}integer\end{minipage} & \begin{minipage}[t]{2cm}0\end{minipage} & \begin{minipage}[t]{12cm}charge of pion\end{minipage}\\*
\midrule
numberPions & \begin{minipage}[t]{2cm}integer\end{minipage} & \begin{minipage}[t]{2cm}200\end{minipage} & \begin{minipage}[t]{12cm}number of initialized pions per ensemble\end{minipage}\\*
\midrule
ekin\_lab & \begin{minipage}[t]{2cm}real\end{minipage} & \begin{minipage}[t]{2cm}0.\end{minipage} & \begin{minipage}[t]{12cm}kinetic energies of pions in lab frame.\end{minipage}\\*
\midrule
delta\_ekin\_lab & \begin{minipage}[t]{2cm}real\end{minipage} & \begin{minipage}[t]{2cm}0.01\end{minipage} & \begin{minipage}[t]{12cm}step size for kinetic energies in energy scans\end{minipage}\\*
\bottomrule
\end{longtable}
{ }



% pn_medium     code/width/proton_neutron_width_medium.f90

\begin{longtable}{llll}
\toprule
\textbf{\large{pn\_medium}} & \multicolumn{3}{l}{\footnotesize{code/width/proton\_neutron\_width\_medium.f90}}\\*
\midrule
\endfirsthead
\midrule
\endhead
density\_dependent & \begin{minipage}[t]{2cm}logical\end{minipage} & \begin{minipage}[t]{2cm}.false.\end{minipage} & \begin{minipage}[t]{12cm}the density of the spectral function\end{minipage}\\*
\midrule
pn\_medium\_switch & \begin{minipage}[t]{2cm}logical\end{minipage} & \begin{minipage}[t]{2cm}.true.\end{minipage} & \begin{minipage}[t]{12cm}If .true. medium\_modifications will be used\end{minipage}\\*
\midrule
form\_factor & \begin{minipage}[t]{2cm}logical\end{minipage} & \begin{minipage}[t]{2cm}.true.\end{minipage} & \begin{minipage}[t]{12cm}If .true. the form factor for the width is used\end{minipage}\\*
\bottomrule
\end{longtable}
{ }



% projectile     code/density/nucleus.f90

\begin{longtable}{llll}
\toprule
\textbf{\large{projectile}} & \multicolumn{3}{l}{\footnotesize{code/density/nucleus.f90}}\\*
\midrule
\endfirsthead
\midrule
\endhead
A & \begin{minipage}[t]{2cm}integer\end{minipage} & \begin{minipage}[t]{2cm}0\end{minipage} & \begin{minipage}[t]{12cm}Mass A of target nucleus ( = number of nucleons). If zero, a default isotope is chosen for the given Z.\end{minipage}\\*
\midrule
Z & \begin{minipage}[t]{2cm}integer\end{minipage} & \begin{minipage}[t]{2cm}20\end{minipage} & \begin{minipage}[t]{12cm}Charge Z of target nucleus ( = number of protons).\end{minipage}\\*
\midrule
Projectile\_A & \begin{minipage}[t]{2cm}integer\end{minipage} & \begin{minipage}[t]{2cm}-99\end{minipage} & \begin{minipage}[t]{12cm}deprecated, use 'A' instead\end{minipage}\\*
\midrule
Projectile\_Z & \begin{minipage}[t]{2cm}integer\end{minipage} & \begin{minipage}[t]{2cm}-99\end{minipage} & \begin{minipage}[t]{12cm}deprecated, use 'Z' instead\end{minipage}\\*
\midrule
fermiMotion & \begin{minipage}[t]{2cm}logical\end{minipage} & \begin{minipage}[t]{2cm}.true.\end{minipage} & \begin{minipage}[t]{12cm}Determines whether particles in target nucleus feel Fermi motion.\end{minipage}\\*
\midrule
densitySwitch\_static & \begin{minipage}[t]{2cm}integer\end{minipage} & \begin{minipage}[t]{2cm}3\end{minipage} & \begin{minipage}[t]{12cm}This switch is important, because it decides, which static density is used to set up the testparticles in the nuclei before the first time-step.\\ Possible values:\begin{itemize}\leftmargin0em\itemindent0pt\item 0 : density=0.0\item 1 : Static density uses Woods-Saxon according to H. Lenske\item 2 : Static density according to NPA 554\item 3 : Static density according to Horst Lenske,   implements different radii for neutrons and protons\item 4 : Static density according oscillator shell model\item 5 : Density distribution is a sphere with density according to the       input value of "fermiMomentum\_input".\item 6 : Static Density based on LDA, implemented by Birger Steinmueller\item 7 : Static Density based on LDA + Welke potential\item 8 : Static Density prescription according Relativistic Thomas-Fermi       (Valid only in RMF-mode)\end{itemize} Possible nuclei for the different prescriptions:\begin{itemize}\leftmargin0em\itemindent0pt\item 1 : A $>$ 2 (only A $>$ 16 makes sense)\item 2 :\item 3 :   6$\rightarrow$C(12), 8$\rightarrow$O(16),O(18), 13$\rightarrow$Al(27), 20$\rightarrow$Ca(40),Ca(44), 79$\rightarrow$Au(197)   82$\rightarrow$Pb(208)\item 4: 2$\rightarrow$He(4), 4$\rightarrow$Be(9), 5$\rightarrow$B(11), 6$\rightarrow$C(12), 8$\rightarrow$O(16)\end{itemize}\end{minipage}\\*
\midrule
fermiMomentum\_input & \begin{minipage}[t]{2cm}real\end{minipage} & \begin{minipage}[t]{2cm}0.225\end{minipage} & \begin{minipage}[t]{12cm}Input value of the fermi momentum for densitySwitch\_static=5 (in GeV).\end{minipage}\\*
\midrule
anti & \begin{minipage}[t]{2cm}logical\end{minipage} & \begin{minipage}[t]{2cm}.false.\end{minipage} & \begin{minipage}[t]{12cm}Indicate, whether it is a anti-nucleus\end{minipage}\\*
\bottomrule
\end{longtable}
{ }



% Propagation     code/propagation/propagation.f90

\begin{longtable}{llll}
\toprule
\textbf{\large{Propagation}} & \multicolumn{3}{l}{\footnotesize{code/propagation/propagation.f90}}\\*
\midrule
\endfirsthead
\midrule
\endhead
delta\_P & \begin{minipage}[t]{2cm}real\end{minipage} & \begin{minipage}[t]{2cm}0.01\end{minipage} & \begin{minipage}[t]{12cm}Delta Momentum in derivatives\end{minipage}\\*
\midrule
delta\_E & \begin{minipage}[t]{2cm}real\end{minipage} & \begin{minipage}[t]{2cm}0.01\end{minipage} & \begin{minipage}[t]{12cm}Delta energy in derivatives\end{minipage}\\*
\midrule
UseCoulombDirectly & \begin{minipage}[t]{2cm}logical\end{minipage} & \begin{minipage}[t]{2cm}.true.\end{minipage} & \begin{minipage}[t]{12cm}Whether to use coulomb force directly in propagation or not. (If switched off while coulomb is switched on in module coulomb, the effect of the coulomb potential comes in via the gradient of the potentials. With this flag you can not switch on/off coulomb, you just select, how it is treated.)\end{minipage}\\*
\midrule
UseHadronic & \begin{minipage}[t]{2cm}logical\end{minipage} & \begin{minipage}[t]{2cm}.true.\end{minipage} & \begin{minipage}[t]{12cm}Whether to use hadronic potentials in propagation\end{minipage}\\*
\midrule
FreezeNonint & \begin{minipage}[t]{2cm}logical\end{minipage} & \begin{minipage}[t]{2cm}.false.\end{minipage} & \begin{minipage}[t]{12cm}If switched on, the real particles which did not interact will have zero velocities, i.e. will be "frozen". This is important for stability of the nuclear ground state in real particle simulations. Note that this flag influences only when freezeRealParticles=.false.\end{minipage}\\*
\midrule
RungeKuttaOrder & \begin{minipage}[t]{2cm}integer\end{minipage} & \begin{minipage}[t]{2cm}1\end{minipage} & \begin{minipage}[t]{12cm}Order of Runge-Kutta in derivatives:\begin{itemize}\leftmargin0em\itemindent0pt\item 1 = first order Runge-Kutta\item 2 = second order Runge-Kuttay\end{itemize}\end{minipage}\\*
\midrule
Mode & \begin{minipage}[t]{2cm}integer\end{minipage} & \begin{minipage}[t]{2cm}2\end{minipage} & \begin{minipage}[t]{12cm}define the type of propagation:\begin{itemize}\leftmargin0em\itemindent0pt\item 0: Cascade\item 1: Euler\item 2: PredictorCorrector\item 3: no propagation, random placement in box\end{itemize}\end{minipage}\\*
\midrule
dh\_dp0\_switch & \begin{minipage}[t]{2cm}logical\end{minipage} & \begin{minipage}[t]{2cm}.true.\end{minipage} & \begin{minipage}[t]{12cm}Switch which decides whether we use dh\_dp0.\end{minipage}\\*
\midrule
offShellInfoDetail & \begin{minipage}[t]{2cm}logical\end{minipage} & \begin{minipage}[t]{2cm}.false.\end{minipage} & \begin{minipage}[t]{12cm}print out detailed offShellInfo\end{minipage}\\*
\midrule
tachyonDebug & \begin{minipage}[t]{2cm}logical\end{minipage} & \begin{minipage}[t]{2cm}.false.\end{minipage} & \begin{minipage}[t]{12cm}...\end{minipage}\\*
\bottomrule
\end{longtable}
{ }



% propagation_RMF_input     code/propagation/propagation_RMF.f90

\begin{longtable}{llll}
\toprule
\textbf{\large{propagation\_RMF\_input}} & \multicolumn{3}{l}{\footnotesize{code/propagation/propagation\_RMF.f90}}\\*
\midrule
\endfirsthead
\midrule
\endhead
predictorCorrector & \begin{minipage}[t]{2cm}logical\end{minipage} & \begin{minipage}[t]{2cm}.true.\end{minipage} & \begin{minipage}[t]{12cm}Switch for predictor-corrector method in the propagation. If .false. then simple Euler method is used (i.e. only predictor step is done)\end{minipage}\\*
\midrule
deleteTachyons & \begin{minipage}[t]{2cm}logical\end{minipage} & \begin{minipage}[t]{2cm}.false.\end{minipage} & \begin{minipage}[t]{12cm}Switch for treatment of particles with velocity $>$ 1.\\ Possible values:\begin{itemize}\leftmargin0em\itemindent0pt\item if .true., these particles are deleted by setting ID=0\item if .false., these particles are propagated, but with modified velocity\end{itemize}\end{minipage}\\*
\bottomrule
\end{longtable}
{ }



% pythia     code/collisions/twoBodyReactions/HiEnergy/DoCollTools.f90

\begin{longtable}{llll}
\toprule
\textbf{\large{pythia}} & \multicolumn{3}{l}{\footnotesize{code/collisions/twoBodyReactions/HiEnergy/DoCollTools.f90}}\\*
\midrule
\endfirsthead
\midrule
\endhead
MSEL & \begin{minipage}[t]{2cm}integer\end{minipage} & \begin{minipage}[t]{2cm}\end{minipage} & \begin{minipage}[t]{12cm}Pythia variable\end{minipage}\\*
\midrule
MSTU & \begin{minipage}[t]{2cm}integer, dimension(200)\end{minipage} & \begin{minipage}[t]{2cm}\end{minipage} & \begin{minipage}[t]{12cm}Pythia array\end{minipage}\\*
\midrule
MSTJ & \begin{minipage}[t]{2cm}integer, dimension(200)\end{minipage} & \begin{minipage}[t]{2cm}\end{minipage} & \begin{minipage}[t]{12cm}Pythia array\end{minipage}\\*
\midrule
MSTP & \begin{minipage}[t]{2cm}integer, dimension(200)\end{minipage} & \begin{minipage}[t]{2cm}\end{minipage} & \begin{minipage}[t]{12cm}Pythia array\end{minipage}\\*
\midrule
MSTI & \begin{minipage}[t]{2cm}integer, dimension(200)\end{minipage} & \begin{minipage}[t]{2cm}\end{minipage} & \begin{minipage}[t]{12cm}Pythia array\end{minipage}\\*
\midrule
PARU & \begin{minipage}[t]{2cm}real, dimension(200)\end{minipage} & \begin{minipage}[t]{2cm}\end{minipage} & \begin{minipage}[t]{12cm}Pythia array\end{minipage}\\*
\midrule
PARJ & \begin{minipage}[t]{2cm}real, dimension(200)\end{minipage} & \begin{minipage}[t]{2cm}\end{minipage} & \begin{minipage}[t]{12cm}Pythia array\end{minipage}\\*
\midrule
PARP & \begin{minipage}[t]{2cm}real, dimension(200)\end{minipage} & \begin{minipage}[t]{2cm}\end{minipage} & \begin{minipage}[t]{12cm}Pythia array\end{minipage}\\*
\midrule
PARI & \begin{minipage}[t]{2cm}real, dimension(200)\end{minipage} & \begin{minipage}[t]{2cm}\end{minipage} & \begin{minipage}[t]{12cm}Pythia array\end{minipage}\\*
\midrule
CKIN & \begin{minipage}[t]{2cm}real, dimension(200)\end{minipage} & \begin{minipage}[t]{2cm}\end{minipage} & \begin{minipage}[t]{12cm}Pythia array\end{minipage}\\*
\midrule
PMAS & \begin{minipage}[t]{2cm}real, dimension(500, 4)\end{minipage} & \begin{minipage}[t]{2cm}\end{minipage} & \begin{minipage}[t]{12cm}Pythia array\end{minipage}\\*
\midrule
MDCY & \begin{minipage}[t]{2cm}integer, dimension(500, 3)\end{minipage} & \begin{minipage}[t]{2cm}\end{minipage} & \begin{minipage}[t]{12cm}Pythia array\end{minipage}\\*
\bottomrule
\end{longtable}
{ }



% residue_Input     code/analysis/sourceAnalysis/residue.f90

\begin{longtable}{llll}
\toprule
\textbf{\large{residue\_Input}} & \multicolumn{3}{l}{\footnotesize{code/analysis/sourceAnalysis/residue.f90}}\\*
\midrule
\endfirsthead
\midrule
\endhead
DetermineResidue & \begin{minipage}[t]{2cm}logical\end{minipage} & \begin{minipage}[t]{2cm}.true.\end{minipage} & \begin{minipage}[t]{12cm}If .true., then the determination of target residue properties for every event will be done.\\ Their output in file 'TargetResidue.dat' at the end of time evolution is called elsewhere. If nothing is stored, no output is generated.\end{minipage}\\*
\midrule
mode & \begin{minipage}[t]{2cm}integer\end{minipage} & \begin{minipage}[t]{2cm}1\end{minipage} & \begin{minipage}[t]{12cm}select the mode, how the residue energy is determined (field res\%mom(0)):\begin{itemize}\leftmargin0em\itemindent0pt\item 1: the sum of hole excitation energies\item 2: the sum of energies of the removed particles (with minus sign)\end{itemize}\end{minipage}\\*
\midrule
switchOutput & \begin{minipage}[t]{2cm}integer\end{minipage} & \begin{minipage}[t]{2cm}0\end{minipage} & \begin{minipage}[t]{12cm}select the output * 1: write out TargetResidue.dat * 2: write out TargetResidue.Plot.dat * 3: write out both files\end{minipage}\\*
\bottomrule
\end{longtable}
{ }



% ResonanceCrossSections     code/collisions/twoBodyReactions/baryonMeson/resonanceCrossSections.f90

\begin{longtable}{llll}
\toprule
\textbf{\large{ResonanceCrossSections}} & \multicolumn{3}{l}{\footnotesize{code/collisions/twoBodyReactions/baryonMeson/resonanceCrossSections.f90}}\\*
\midrule
\endfirsthead
\midrule
\endhead
fullPropagator & \begin{minipage}[t]{2cm}logical\end{minipage} & \begin{minipage}[t]{2cm}.false.\end{minipage} & \begin{minipage}[t]{12cm}Includes also the real parts in the resonance propagator. In former works (i.e. in the old Efffenberger code) this has been neglected. It should be set to .true. only if mediumSwitch\_coll=.true. in the namelist width\_Baryon.\end{minipage}\\*
\bottomrule
\end{longtable}
{ }



% RMF_input     code/rmf/RMF.f90

\begin{longtable}{llll}
\toprule
\textbf{\large{RMF\_input}} & \multicolumn{3}{l}{\footnotesize{code/rmf/RMF.f90}}\\*
\midrule
\endfirsthead
\midrule
\endhead
RMF\_flag & \begin{minipage}[t]{2cm}logical\end{minipage} & \begin{minipage}[t]{2cm}.false.\end{minipage} & \begin{minipage}[t]{12cm}If .true. then use relativistic mean fields.\end{minipage}\\*
\midrule
N\_set & \begin{minipage}[t]{2cm}integer\end{minipage} & \begin{minipage}[t]{2cm}1\end{minipage} & \begin{minipage}[t]{12cm}Select parameter set to use:\begin{itemize}\leftmargin0em\itemindent0pt\item  1 --- NL1 [Lalazissis] (K=211.29 MeV, m*/m=0.57)\item  2 --- NL3 [Lalazissis] (K=271.76 MeV, m*/m=0.60)\item  3 --- NL2 [Lang]       (K=210 MeV,    m*/m=0.83)\item  4 --- NLZ2 [Bender]    (K=172 MeV,    m*/m=0.583)\item  5 --- NL3* [Lalazissis, priv. comm.] (K=258.28 MeV, m*/m=0.594)\item  6 --- Same as N\_set=3, but including the rho meson.\item  7 --- NL1 [Lee]        (K=212 MeV,    m*/m=0.57)\item  8 --- NL2 [Lee]        (K=399 MeV,    m*/m=0.67)\item  9 --- Set I [ Liu]     (K=240 MeV,    m*/m=0.75)\item 10 --- NL1 [Lang]       (K=380 MeV,    m*/m=0.83)\item 11 --- NL3 [Lang]       (K=380 MeV,    m*/m=0.70)\item 31 --- Parity doublet model Set P3 [Zschiesche] (K=374 MeV)\item 32 --- Parity doublet model Set P2 [Zschiesche] (K=374 MeV)\item 33 --- Parity doublet model Set 1 [Shin] (K=240 MeV)\item 34 --- Parity doublet model Set 2 [Shin] (K=215 MeV)\end{itemize} References:\begin{itemize}\leftmargin0em\itemindent0pt\item Bender et al., PRC 60, 34304 (1999)\item Lalazissis et al., PRC 55, 540 (1997),\item Lang et al., NPA 541, 507 (1992)\item Lee et al., PRL 57, 2916 (1986)\item Liu et al., PRC 65, 045201 (2002)\item Shin et al., arXiv:1805.03402\item Zschiesche et al., PRC 75, 055202 (2007)\end{itemize}\end{minipage}\\*
\midrule
grad\_flag & \begin{minipage}[t]{2cm}logical\end{minipage} & \begin{minipage}[t]{2cm}.false.\end{minipage} & \begin{minipage}[t]{12cm}If .true. then include space derivatives of the fields.\end{minipage}\\*
\midrule
lorentz\_flag & \begin{minipage}[t]{2cm}logical\end{minipage} & \begin{minipage}[t]{2cm}.true.\end{minipage} & \begin{minipage}[t]{12cm}If .false. then the space components of the omega and rho fields are put to zero.\end{minipage}\\*
\midrule
Tens\_flag & \begin{minipage}[t]{2cm}logical\end{minipage} & \begin{minipage}[t]{2cm}.false.\end{minipage} & \begin{minipage}[t]{12cm}If .true. then compute the energy-momentum tensor and four-momentum density field (not used in propagation).\end{minipage}\\*
\midrule
flagCorThr & \begin{minipage}[t]{2cm}logical\end{minipage} & \begin{minipage}[t]{2cm}.false.\end{minipage} & \begin{minipage}[t]{12cm}If .true. then the srtfree of colliding particles is corrected to ensure in-medium thresholds of BB $\rightarrow$ BB and MB $\rightarrow$ B\end{minipage}\\*
\midrule
kaonpot\_flag & \begin{minipage}[t]{2cm}logical\end{minipage} & \begin{minipage}[t]{2cm}.false.\end{minipage} & \begin{minipage}[t]{12cm}This switch turns on the Kaon potential in RMF mode.\end{minipage}\\*
\midrule
fact\_pbar & \begin{minipage}[t]{2cm}real\end{minipage} & \begin{minipage}[t]{2cm}1.\end{minipage} & \begin{minipage}[t]{12cm}Modification factor for the antiproton coupling constants.\end{minipage}\\*
\midrule
fact\_Delta & \begin{minipage}[t]{2cm}real\end{minipage} & \begin{minipage}[t]{2cm}1.\end{minipage} & \begin{minipage}[t]{12cm}Modification factor for the Delta(1232) coupling constants.\end{minipage}\\*
\midrule
fact\_hyp & \begin{minipage}[t]{2cm}real\end{minipage} & \begin{minipage}[t]{2cm}1.\end{minipage} & \begin{minipage}[t]{12cm}Modification factor for the hyperon coupling constants.\end{minipage}\\*
\midrule
fact\_antihyp & \begin{minipage}[t]{2cm}real\end{minipage} & \begin{minipage}[t]{2cm}1.\end{minipage} & \begin{minipage}[t]{12cm}Modification factor for the antihyperon coupling constants.\end{minipage}\\*
\midrule
fact\_Xi & \begin{minipage}[t]{2cm}real\end{minipage} & \begin{minipage}[t]{2cm}1.\end{minipage} & \begin{minipage}[t]{12cm}Modification factor for the Xi and XiStar coupling constants.\end{minipage}\\*
\midrule
fact\_antiXi & \begin{minipage}[t]{2cm}real\end{minipage} & \begin{minipage}[t]{2cm}1.\end{minipage} & \begin{minipage}[t]{12cm}Modification factor for the antiXi and antiXiStar coupling constants.\end{minipage}\\*
\midrule
fact\_kaon & \begin{minipage}[t]{2cm}real\end{minipage} & \begin{minipage}[t]{2cm}0.\end{minipage} & \begin{minipage}[t]{12cm}Modification factor for the Kaon and antikaon coupling constants.\end{minipage}\\*
\midrule
flagVectMod & \begin{minipage}[t]{2cm}logical\end{minipage} & \begin{minipage}[t]{2cm}.true.\end{minipage} & \begin{minipage}[t]{12cm}This switch turns on the modification factors for vector couplings.\end{minipage}\\*
\bottomrule
\end{longtable}
{ }



% selfenergy_realPart     code/spectralFunctions/selfenergy_baryons.f90

\begin{longtable}{llll}
\toprule
\textbf{\large{selfenergy\_realPart}} & \multicolumn{3}{l}{\footnotesize{code/spectralFunctions/selfenergy\_baryons.f90}}\\*
\midrule
\endfirsthead
\midrule
\endhead
rel\_accuracy & \begin{minipage}[t]{2cm}real\end{minipage} & \begin{minipage}[t]{2cm}0.05\end{minipage} & \begin{minipage}[t]{12cm}Relative accuracy for resonance self energy\end{minipage}\\*
\midrule
intSolver & \begin{minipage}[t]{2cm}integer\end{minipage} & \begin{minipage}[t]{2cm}1\end{minipage} & \begin{minipage}[t]{12cm}Decide on the numerical package to be used for the Cauchy integral:\begin{itemize}\leftmargin0em\itemindent0pt\item 1=quadpack routine\item 2=cernlib routine\end{itemize}\end{minipage}\\*
\midrule
makeTable & \begin{minipage}[t]{2cm}logical\end{minipage} & \begin{minipage}[t]{2cm}.true.\end{minipage} & \begin{minipage}[t]{12cm}Switch on/off the usage of an input tabulation\end{minipage}\\*
\midrule
noDispersion & \begin{minipage}[t]{2cm}logical\end{minipage} & \begin{minipage}[t]{2cm}.false.\end{minipage} & \begin{minipage}[t]{12cm}Switch on/off the usage dispersion relations\end{minipage}\\*
\midrule
maxRes & \begin{minipage}[t]{2cm}integer\end{minipage} & \begin{minipage}[t]{2cm}100\end{minipage} & \begin{minipage}[t]{12cm}\end{minipage}\\*
\midrule
minRes & \begin{minipage}[t]{2cm}integer\end{minipage} & \begin{minipage}[t]{2cm}-100\end{minipage} & \begin{minipage}[t]{12cm}\end{minipage}\\*
\midrule
extrapolateAbsP & \begin{minipage}[t]{2cm}logical\end{minipage} & \begin{minipage}[t]{2cm}.false.\end{minipage} & \begin{minipage}[t]{12cm}if(true) then set absP to maxAbsP if absP is larger\end{minipage}\\*
\midrule
writeLocal & \begin{minipage}[t]{2cm}logical\end{minipage} & \begin{minipage}[t]{2cm}.false.\end{minipage} & \begin{minipage}[t]{12cm}\begin{itemize}\leftmargin0em\itemindent0pt\item Tables are outputted to local directory, not to buuinput\end{itemize}\end{minipage}\\*
\bottomrule
\end{longtable}
{ }



% selfEnergyMesons     code/spectralFunctions/selfenergy_mesons.f90

\begin{longtable}{llll}
\toprule
\textbf{\large{selfEnergyMesons}} & \multicolumn{3}{l}{\footnotesize{code/spectralFunctions/selfenergy\_mesons.f90}}\\*
\midrule
\endfirsthead
\midrule
\endhead
dispersion & \begin{minipage}[t]{2cm}logical\end{minipage} & \begin{minipage}[t]{2cm}.false.\end{minipage} & \begin{minipage}[t]{12cm}Use dispersive real parts of the self energy.\end{minipage}\\*
\bottomrule
\end{longtable}
{ }



% SMM_input     code/analysis/sourceAnalysis.f90

\begin{longtable}{llll}
\toprule
\textbf{\large{SMM\_input}} & \multicolumn{3}{l}{\footnotesize{code/analysis/sourceAnalysis.f90}}\\*
\midrule
\endfirsthead
\midrule
\endhead
SMM\_Flag & \begin{minipage}[t]{2cm}logical\end{minipage} & \begin{minipage}[t]{2cm}.false.\end{minipage} & \begin{minipage}[t]{12cm}if .true. then source analysis is switched on\end{minipage}\\*
\midrule
rho\_cutoff & \begin{minipage}[t]{2cm}real\end{minipage} & \begin{minipage}[t]{2cm}100.\end{minipage} & \begin{minipage}[t]{12cm}density cutoff (in units of the saturation density "rhoNull") which defines "emitting" particles\end{minipage}\\*
\midrule
spectator\_cutoff & \begin{minipage}[t]{2cm}real\end{minipage} & \begin{minipage}[t]{2cm}1.\end{minipage} & \begin{minipage}[t]{12cm}min. value of number of collisions which defines "spectator"-matter\end{minipage}\\*
\midrule
A\_cutoff & \begin{minipage}[t]{2cm}integer\end{minipage} & \begin{minipage}[t]{2cm}2\end{minipage} & \begin{minipage}[t]{12cm}min. value of the source mass number\end{minipage}\\*
\midrule
SelectionMethod & \begin{minipage}[t]{2cm}integer\end{minipage} & \begin{minipage}[t]{2cm}0\end{minipage} & \begin{minipage}[t]{12cm}defines the selection method of spectators and fireball. Can be used in high energy Hadron-Nucleus events.\end{minipage}\\*
\midrule
betaChoice & \begin{minipage}[t]{2cm}integer\end{minipage} & \begin{minipage}[t]{2cm}0\end{minipage} & \begin{minipage}[t]{12cm}Defines the way to calculate the source velocity in RMF mode. Has no influence in calculations with Skyrme potential.\end{minipage}\\*
\midrule
MaxTimePrinting & \begin{minipage}[t]{2cm}integer\end{minipage} & \begin{minipage}[t]{2cm}10\end{minipage} & \begin{minipage}[t]{12cm}Indicates how many times the results are printed into files.\\NOTES\\ Set MaxTimePrinting to a very big value, i.e. 1000, if you wish that the BUU-run developes until time=time\_max.\end{minipage}\\*
\midrule
DetailedHyperonOutput & \begin{minipage}[t]{2cm}logical\end{minipage} & \begin{minipage}[t]{2cm}.true.\end{minipage} & \begin{minipage}[t]{12cm}print more informations for Hyperons and pions.\end{minipage}\\*
\midrule
hyperSource & \begin{minipage}[t]{2cm}logical\end{minipage} & \begin{minipage}[t]{2cm}.false.\end{minipage} & \begin{minipage}[t]{12cm}If true, the Lambda and Sigma0 hyperons will be included into source\end{minipage}\\*
\bottomrule
\end{longtable}
{ }



% spectralFunction     code/spectralFunctions/spectralFunc.f90

\begin{longtable}{llll}
\toprule
\textbf{\large{spectralFunction}} & \multicolumn{3}{l}{\footnotesize{code/spectralFunctions/spectralFunc.f90}}\\*
\midrule
\endfirsthead
\midrule
\endhead
which\_nuclwidth & \begin{minipage}[t]{2cm}integer\end{minipage} & \begin{minipage}[t]{2cm}1\end{minipage} & \begin{minipage}[t]{12cm}This flag decides what is used for the nucleon width.\\ Possible values:\begin{itemize}\leftmargin0em\itemindent0pt\item 1 : use constant width given in const\_nuclwidth\item 2 : use width increasing linear with density;   Gamma=const*rho/rho0 with const given in nuclwidth\_dens\item 3 : use toy model (constant NN cross section)\item 4 : use realistic width (cf. diploma thesis of D. Kalok)\item 5 : use realistic width: width based on our collision term\end{itemize}NOTES\\ The correct normalisation has not been included here!!\end{minipage}\\*
\midrule
nuclwidth & \begin{minipage}[t]{2cm}real\end{minipage} & \begin{minipage}[t]{2cm}0.001\end{minipage} & \begin{minipage}[t]{12cm}if which\_nuclwidth=1, nuclwidth gives the width used in the Breit-Wigner for the nucleon\end{minipage}\\*
\midrule
nuclwidth\_dens & \begin{minipage}[t]{2cm}real\end{minipage} & \begin{minipage}[t]{2cm}0.006\end{minipage} & \begin{minipage}[t]{12cm}if which\_nuclwidth=2, nuclwidth\_dens gives the width used in density dependent width\\ 6 MeV are motivated in F. Froemel dissertation\end{minipage}\\*
\midrule
nuclwidth\_sig & \begin{minipage}[t]{2cm}real\end{minipage} & \begin{minipage}[t]{2cm}5.5\end{minipage} & \begin{minipage}[t]{12cm}if which\_nuclwidth=3, nuclwidth\_sig gives the NN cross section in fm\^{}2\end{minipage}\\*
\midrule
relativistic & \begin{minipage}[t]{2cm}logical\end{minipage} & \begin{minipage}[t]{2cm}.true.\end{minipage} & \begin{minipage}[t]{12cm}Use either relativistic or non-relativistic spectral functions.\end{minipage}\\*
\midrule
widthMass & \begin{minipage}[t]{2cm}integer\end{minipage} & \begin{minipage}[t]{2cm}1\end{minipage} & \begin{minipage}[t]{12cm}select which mass is used to calculate the width:\begin{itemize}\leftmargin0em\itemindent0pt\item 1: bare mass\item 2: invariant mass\item 3: invariant mass + mass shift of nucleon (for PDM)   (= 1 for RMF and Skyrme)\end{itemize}\end{minipage}\\*
\bottomrule
\end{longtable}
{ }



% spectralFunctionMesons     code/spectralFunctions/spectralFuncMesons.f90

\begin{longtable}{llll}
\toprule
\textbf{\large{spectralFunctionMesons}} & \multicolumn{3}{l}{\footnotesize{code/spectralFunctions/spectralFuncMesons.f90}}\\*
\midrule
\endfirsthead
\midrule
\endhead
relativistic & \begin{minipage}[t]{2cm}logical\end{minipage} & \begin{minipage}[t]{2cm}.true.\end{minipage} & \begin{minipage}[t]{12cm}\begin{itemize}\leftmargin0em\itemindent0pt\item Use either relativistic or non relativistic spectral functions.\end{itemize}\end{minipage}\\*
\bottomrule
\end{longtable}
{ }



% target     code/density/nucleus.f90

\begin{longtable}{llll}
\toprule
\textbf{\large{target}} & \multicolumn{3}{l}{\footnotesize{code/density/nucleus.f90}}\\*
\midrule
\endfirsthead
\midrule
\endhead
A & \begin{minipage}[t]{2cm}integer\end{minipage} & \begin{minipage}[t]{2cm}0\end{minipage} & \begin{minipage}[t]{12cm}Mass A of target nucleus ( = number of nucleons). If zero, a default isotope is chosen for the given Z.\end{minipage}\\*
\midrule
Z & \begin{minipage}[t]{2cm}integer\end{minipage} & \begin{minipage}[t]{2cm}20\end{minipage} & \begin{minipage}[t]{12cm}Charge Z of target nucleus ( = number of protons).\end{minipage}\\*
\midrule
Target\_A & \begin{minipage}[t]{2cm}integer\end{minipage} & \begin{minipage}[t]{2cm}-99\end{minipage} & \begin{minipage}[t]{12cm}deprecated, use 'A' instead\end{minipage}\\*
\midrule
Target\_Z & \begin{minipage}[t]{2cm}integer\end{minipage} & \begin{minipage}[t]{2cm}-99\end{minipage} & \begin{minipage}[t]{12cm}deprecated, use 'Z' instead\end{minipage}\\*
\midrule
fermiMotion & \begin{minipage}[t]{2cm}logical\end{minipage} & \begin{minipage}[t]{2cm}.true.\end{minipage} & \begin{minipage}[t]{12cm}Determines whether particles in target nucleus feel Fermi motion.\end{minipage}\\*
\midrule
densitySwitch\_static & \begin{minipage}[t]{2cm}integer\end{minipage} & \begin{minipage}[t]{2cm}3\end{minipage} & \begin{minipage}[t]{12cm}This switch is important, because it decides, which static density is used to set up the testparticles in the nuclei before the first time-step.\\ Possible values:\begin{itemize}\leftmargin0em\itemindent0pt\item 0 : density=0.0\item 1 : Static density uses Woods-Saxon according to H. Lenske\item 2 : Static density according to NPA 554\item 3 : Static density according to Horst Lenske,   implements different radii for neutrons and protons\item 4 : Static density according oscillator shell model\item 5 : Density distribution is a sphere with density according to the       input value of "fermiMomentum\_input".\item 6 : Static Density based on LDA, implemented by Birger Steinmueller\item 7 : Static Density based on LDA + Welke potential\item 8 : Static Density prescription according Relativistic Thomas-Fermi       (Valid only in RMF-mode)\end{itemize} Possible nuclei for the different prescriptions:\begin{itemize}\leftmargin0em\itemindent0pt\item 1 : A $>$ 2 (only A $>$ 16 makes sense)\item 2 :\item 3 :   6$\rightarrow$C(12), 8$\rightarrow$O(16),O(18), 13$\rightarrow$Al(27), 20$\rightarrow$Ca(40),Ca(44), 79$\rightarrow$Au(197)   82$\rightarrow$Pb(208)\item 4: 2$\rightarrow$He(4), 4$\rightarrow$Be(9), 5$\rightarrow$B(11), 6$\rightarrow$C(12), 8$\rightarrow$O(16)\end{itemize}\end{minipage}\\*
\midrule
fermiMomentum\_input & \begin{minipage}[t]{2cm}real\end{minipage} & \begin{minipage}[t]{2cm}0.225\end{minipage} & \begin{minipage}[t]{12cm}Input value of the fermi momentum for densitySwitch\_static=5 (in GeV).\end{minipage}\\*
\midrule
ReAdjustForConstBinding & \begin{minipage}[t]{2cm}logical\end{minipage} & \begin{minipage}[t]{2cm}.false.\end{minipage} & \begin{minipage}[t]{12cm}If this flag is set to true, we use the selected density distribution only for a preliminary step, where we calculate the baryonic potential as function of r (which depends on the density distribution). From the condition, that the binding energy has to be constant, we deduce the distribution of the fermi momentum and thus the 'new' density distribution.\\ The tabulated density distribution is replaced via the 'new' one and all behaviour is as usual.\end{minipage}\\*
\midrule
ConstBinding & \begin{minipage}[t]{2cm}real\end{minipage} & \begin{minipage}[t]{2cm}-0.008\end{minipage} & \begin{minipage}[t]{12cm}if 'ReAdjustForConstBinding' equals true, we a trying to readjust the fermi momentum and the density such, we quarantee this value for the binding energy.\end{minipage}\\*
\bottomrule
\end{longtable}
{ }



% TransportGivenParticle     code/init/initTransportGivenParticle.f90

\begin{longtable}{llll}
\toprule
\textbf{\large{TransportGivenParticle}} & \multicolumn{3}{l}{\footnotesize{code/init/initTransportGivenParticle.f90}}\\*
\midrule
\endfirsthead
\midrule
\endhead
particle\_ID & \begin{minipage}[t]{2cm}integer\end{minipage} & \begin{minipage}[t]{2cm}1\end{minipage} & \begin{minipage}[t]{12cm}Determines what kind of particle is initialized (see idTable)\end{minipage}\\*
\midrule
charge & \begin{minipage}[t]{2cm}integer\end{minipage} & \begin{minipage}[t]{2cm}1\end{minipage} & \begin{minipage}[t]{12cm}Determines what charge\end{minipage}\\*
\midrule
position & \begin{minipage}[t]{2cm}real, dimension(1:3)\end{minipage} & \begin{minipage}[t]{2cm}(/0.,0.,0./)\end{minipage} & \begin{minipage}[t]{12cm}Determines the position.\end{minipage}\\*
\midrule
threemomentum & \begin{minipage}[t]{2cm}real, dimension(1:3)\end{minipage} & \begin{minipage}[t]{2cm}(/0.,0.,1./)\end{minipage} & \begin{minipage}[t]{12cm}Determines the three-momentum.\end{minipage}\\*
\midrule
mass & \begin{minipage}[t]{2cm}real\end{minipage} & \begin{minipage}[t]{2cm}-1.\end{minipage} & \begin{minipage}[t]{12cm}Determines the mass (if negative, choose mass according to spectral function).\end{minipage}\\*
\midrule
maxmass & \begin{minipage}[t]{2cm}real\end{minipage} & \begin{minipage}[t]{2cm}1.5\end{minipage} & \begin{minipage}[t]{12cm}Determines the maximum mass (if mass is chosen according to spectral function).\end{minipage}\\*
\midrule
perweight & \begin{minipage}[t]{2cm}real\end{minipage} & \begin{minipage}[t]{2cm}1.\end{minipage} & \begin{minipage}[t]{12cm}Determines the weight.\end{minipage}\\*
\midrule
frequency & \begin{minipage}[t]{2cm}integer\end{minipage} & \begin{minipage}[t]{2cm}10\end{minipage} & \begin{minipage}[t]{12cm}after this amount of time steps a new output file is generated\end{minipage}\\*
\midrule
initRandomRadiativeDelta & \begin{minipage}[t]{2cm}logical\end{minipage} & \begin{minipage}[t]{2cm}.false.\end{minipage} & \begin{minipage}[t]{12cm}intented use: radiativeDelta decay. chooses position,threemomentum,mass of Delta randomly; charge is choosen either 0 or 1\end{minipage}\\*
\bottomrule
\end{longtable}
{ }



% W_distributions     code/analysis/neutrinoAnalysis.f90

\begin{longtable}{llll}
\toprule
\textbf{\large{W\_distributions}} & \multicolumn{3}{l}{\footnotesize{code/analysis/neutrinoAnalysis.f90}}\\*
\midrule
\endfirsthead
\midrule
\endhead
dW\_Npi & \begin{minipage}[t]{2cm}real\end{minipage} & \begin{minipage}[t]{2cm}0.02\end{minipage} & \begin{minipage}[t]{12cm}for dsigma/d(InvariantMass); only work if include\_W\_dist is .true. set the min, max and steps for various W-distributions\end{minipage}\\*
\midrule
Wmin\_Npi & \begin{minipage}[t]{2cm}real\end{minipage} & \begin{minipage}[t]{2cm}1.08\end{minipage} & \begin{minipage}[t]{12cm}for dsigma/d(InvariantMass); only work if include\_W\_dist is .true. set the min, max and steps for various W-distributions\end{minipage}\\*
\midrule
Wmax\_Npi & \begin{minipage}[t]{2cm}real\end{minipage} & \begin{minipage}[t]{2cm}1.6\end{minipage} & \begin{minipage}[t]{12cm}for dsigma/d(InvariantMass); only work if include\_W\_dist is .true. set the min, max and steps for various W-distributions\end{minipage}\\*
\midrule
dW\_mupi & \begin{minipage}[t]{2cm}real\end{minipage} & \begin{minipage}[t]{2cm}0.04\end{minipage} & \begin{minipage}[t]{12cm}only work if include\_W\_dist is .true. set the min, max and steps for various W-distributions\end{minipage}\\*
\midrule
Wmin\_mupi & \begin{minipage}[t]{2cm}real\end{minipage} & \begin{minipage}[t]{2cm}0.24\end{minipage} & \begin{minipage}[t]{12cm}only work if include\_W\_dist is .true. set the min, max and steps for various W-distributions\end{minipage}\\*
\midrule
Wmax\_mupi & \begin{minipage}[t]{2cm}real\end{minipage} & \begin{minipage}[t]{2cm}1.2\end{minipage} & \begin{minipage}[t]{12cm}only work if include\_W\_dist is .true. set the min, max and steps for various W-distributions\end{minipage}\\*
\midrule
dW\_muN & \begin{minipage}[t]{2cm}real\end{minipage} & \begin{minipage}[t]{2cm}0.04\end{minipage} & \begin{minipage}[t]{12cm}only work if include\_W\_dist is .true. set the min, max and steps for various W-distributions\end{minipage}\\*
\midrule
Wmin\_muN & \begin{minipage}[t]{2cm}real\end{minipage} & \begin{minipage}[t]{2cm}1.04\end{minipage} & \begin{minipage}[t]{12cm}only work if include\_W\_dist is .true. set the min, max and steps for various W-distributions\end{minipage}\\*
\midrule
Wmax\_muN & \begin{minipage}[t]{2cm}real\end{minipage} & \begin{minipage}[t]{2cm}2.12\end{minipage} & \begin{minipage}[t]{12cm}only work if include\_W\_dist is .true. set the min, max and steps for various W-distributions\end{minipage}\\*
\bottomrule
\end{longtable}
{ }



% width_Baryon     code/width/baryonWidthMedium.f90

\begin{longtable}{llll}
\toprule
\textbf{\large{width\_Baryon}} & \multicolumn{3}{l}{\footnotesize{code/width/baryonWidthMedium.f90}}\\*
\midrule
\endfirsthead
\midrule
\endhead
mediumSwitch & \begin{minipage}[t]{2cm}logical\end{minipage} & \begin{minipage}[t]{2cm}.false.\end{minipage} & \begin{minipage}[t]{12cm}Switch on and off the in-medium width of all baryons at once. If .false., the vacuum width are used.\end{minipage}\\*
\midrule
mediumSwitch\_Delta & \begin{minipage}[t]{2cm}logical\end{minipage} & \begin{minipage}[t]{2cm}.false.\end{minipage} & \begin{minipage}[t]{12cm}Only meaningful if mediumSwitch=.true.: Switch on and off the in-medium width of the Delta. (.false.=off)\\ Note that in that case the Delta is treated specially: what is used for the in-medium width is determined by the flag in deltaWidth. This switch is not consistent with mediumSwitch\_coll!\end{minipage}\\*
\midrule
mediumSwitch\_proton\_neutron & \begin{minipage}[t]{2cm}logical\end{minipage} & \begin{minipage}[t]{2cm}.false.\end{minipage} & \begin{minipage}[t]{12cm}Only meaningful if mediumSwitch=.true.: Switch on and off the in-medium width of the proton and the neutron. (.false.=off)\\ Note that in that case the nucleons are treated specially. This switch is not consistent with mediumSwitch\_coll!\end{minipage}\\*
\midrule
mediumSwitch\_coll & \begin{minipage}[t]{2cm}logical\end{minipage} & \begin{minipage}[t]{2cm}.false.\end{minipage} & \begin{minipage}[t]{12cm}Only meaningful if mediumSwitch=.true.: Use in-medium width according to collision term.\\NOTES\\ if set to TRUE, then also UseOffShellPotentialBaryons (see module offShellPotential) must be .true.\end{minipage}\\*
\midrule
verboseInit & \begin{minipage}[t]{2cm}logical\end{minipage} & \begin{minipage}[t]{2cm}.false.\end{minipage} & \begin{minipage}[t]{12cm}switch on/off informational messages during initialization\end{minipage}\\*
\midrule
verboseInitStop & \begin{minipage}[t]{2cm}logical\end{minipage} & \begin{minipage}[t]{2cm}.false.\end{minipage} & \begin{minipage}[t]{12cm}Stop after informational messages during initialization or not.\end{minipage}\\*
\bottomrule
\end{longtable}
{ }



% width_Meson     code/width/mesonWidthMedium.f90

\begin{longtable}{llll}
\toprule
\textbf{\large{width\_Meson}} & \multicolumn{3}{l}{\footnotesize{code/width/mesonWidthMedium.f90}}\\*
\midrule
\endfirsthead
\midrule
\endhead
mediumSwitch & \begin{minipage}[t]{2cm}integer\end{minipage} & \begin{minipage}[t]{2cm}0\end{minipage} & \begin{minipage}[t]{12cm}Treatment of In-Medium Widths for mesons:\begin{itemize}\leftmargin0em\itemindent0pt\item 0: Only vacuum widths are used.\item 1: The collisional width is assumed to be constant   (only density-dependent).\item 2: The full tabulated in-medium width is used, as calculated via the   collision term. Isospin asymmetry of nuclear matter included.   Zero temperature assumed. All mesons are in-medium broadened.\item 3: Same as 2 but for isospin symmetric nuclear matter at finite   temperature.   Only rho-meson is in-medium broadened. Other mesons not modified.\end{itemize}\end{minipage}\\*
\midrule
Gamma\_coll\_rho & \begin{minipage}[t]{2cm}real\end{minipage} & \begin{minipage}[t]{2cm}0.150\end{minipage} & \begin{minipage}[t]{12cm}Collisional width for the rho meson in GeV. Only used if mediumSwitch = 1.\end{minipage}\\*
\midrule
Gamma\_coll\_omega & \begin{minipage}[t]{2cm}real\end{minipage} & \begin{minipage}[t]{2cm}0.150\end{minipage} & \begin{minipage}[t]{12cm}Collisional width for the omega meson in GeV. Only used if mediumSwitch = 1.\end{minipage}\\*
\midrule
Gamma\_coll\_phi & \begin{minipage}[t]{2cm}real\end{minipage} & \begin{minipage}[t]{2cm}0.030\end{minipage} & \begin{minipage}[t]{12cm}Collisional width for the phi meson in GeV. Only used if mediumSwitch = 1.\end{minipage}\\*
\midrule
verboseInit & \begin{minipage}[t]{2cm}logical\end{minipage} & \begin{minipage}[t]{2cm}.false.\end{minipage} & \begin{minipage}[t]{12cm}switch on/off informational messages during initialization\end{minipage}\\*
\midrule
allowMix & \begin{minipage}[t]{2cm}logical\end{minipage} & \begin{minipage}[t]{2cm}.false.\end{minipage} & \begin{minipage}[t]{12cm}switch on/off linear interpolation between bins in density while returning the tabulated values for MassAssInfo.\end{minipage}\\*
\bottomrule
\end{longtable}
{ }



% XsectionRatios_input     code/collisions/phaseSpace/XsectionRatios.f90

\begin{longtable}{llll}
\toprule
\textbf{\large{XsectionRatios\_input}} & \multicolumn{3}{l}{\footnotesize{code/collisions/phaseSpace/XsectionRatios.f90}}\\*
\midrule
\endfirsthead
\midrule
\endhead
flagScreen & \begin{minipage}[t]{2cm}logical\end{minipage} & \begin{minipage}[t]{2cm}.false.\end{minipage} & \begin{minipage}[t]{12cm}\begin{itemize}\leftmargin0em\itemindent0pt\item If .true. -- in-medium screening is applied to the input cross section.\item If .false. -- no cross section modification.\end{itemize}\end{minipage}\\*
\midrule
ScreenMode & \begin{minipage}[t]{2cm}integer\end{minipage} & \begin{minipage}[t]{2cm}1\end{minipage} & \begin{minipage}[t]{12cm}possible values:\begin{itemize}\leftmargin0em\itemindent0pt\item 1: in-medium screening of NN total cross section according to   Li and Machleidt, PRC 48 (1993) 1702 and PRC 49 (1994) 566\item 2: in-medium screening of BB total cross section according to   P. Daniewlewicz, NPA 673, 375 (2000); Acta. Phys. Pol. B 33, 45 (2002)\end{itemize}NOTES\\ relevant when  flagScreen = .true.\end{minipage}\\*
\midrule
flagInMedium & \begin{minipage}[t]{2cm}logical\end{minipage} & \begin{minipage}[t]{2cm}.false.\end{minipage} & \begin{minipage}[t]{12cm}\begin{itemize}\leftmargin0em\itemindent0pt\item If .true. -- In-medium ratios are used to decide whether an event is   accepted or not.\item If .false. -- The event is always accepted\end{itemize}\end{minipage}\\*
\midrule
InMediumMode & \begin{minipage}[t]{2cm}integer\end{minipage} & \begin{minipage}[t]{2cm}2\end{minipage} & \begin{minipage}[t]{12cm}possible values:\begin{itemize}\leftmargin0em\itemindent0pt\item 1: all events of the type BB $\rightarrow$ BB (+ mesons) are subject to   in-medium reduction following Eqs.(194),(195) of GiBUU review paper   [currently works in RMF mode only]\item 2: NN $\rightarrow$ NN elastic scattering events are modified according to   Li and Machleidt   all other BB $\rightarrow$ BB (+ mesons  events are subject to   in-medium reduction according to Eq. (33) from   T. Song, C.M. Ko, PRC 91, 014901 (2015)   [works in all modes (Skyrme, RMF, cascade)]\end{itemize}NOTES\\ relevant when flagInMedium = .true.\end{minipage}\\*
\midrule
flagVacEL & \begin{minipage}[t]{2cm}logical\end{minipage} & \begin{minipage}[t]{2cm}.false.\end{minipage} & \begin{minipage}[t]{12cm}\begin{itemize}\leftmargin0em\itemindent0pt\item If .true. -- no in-medium modification for  NN $\rightarrow$ NN elastic\end{itemize}NOTES\\ relevant for flagInMedium = .true.\end{minipage}\\*
\midrule
flagVacHRES & \begin{minipage}[t]{2cm}logical\end{minipage} & \begin{minipage}[t]{2cm}.false.\end{minipage} & \begin{minipage}[t]{12cm}\begin{itemize}\leftmargin0em\itemindent0pt\item If .true. -- no in-medium modification for  BB $\leftrightarrow$ BB   with at least one participating resonance higher than P33(1232)   or with more than one P33(1232)\end{itemize}NOTES\\ relevant for flagInMedium = .true.\end{minipage}\\*
\midrule
flagVacMesProd & \begin{minipage}[t]{2cm}logical\end{minipage} & \begin{minipage}[t]{2cm}.false.\end{minipage} & \begin{minipage}[t]{12cm}\begin{itemize}\leftmargin0em\itemindent0pt\item If .true. -- no in-medium modification for  BB $\rightarrow$ BB + meson(s)\end{itemize}NOTES\\ relevant for flagInMedium = .true.\end{minipage}\\*
\midrule
alpha & \begin{minipage}[t]{2cm}real\end{minipage} & \begin{minipage}[t]{2cm}1.2\end{minipage} & \begin{minipage}[t]{12cm}Parameter which controls the density dependence of the NN $\leftrightarrow$ N Delta cross section via suppression factor of exp(-alpha*(rho/rho\_0)**beta)\\ for the density dependence from: Song/Ko, arXiv:1403.7363 (InMediumMode=2)\end{minipage}\\*
\midrule
beta & \begin{minipage}[t]{2cm}real\end{minipage} & \begin{minipage}[t]{2cm}1.\end{minipage} & \begin{minipage}[t]{12cm}Parameter which controls the density dependence of the NN $\leftrightarrow$ N Delta cross section via suppression factor of exp(-alpha*(rho/rho\_0)**beta)\\ for the density dependence of the type of arXiv:2107.13384 (InMediumMode=2)\end{minipage}\\*
\midrule
shift0 & \begin{minipage}[t]{2cm}real\end{minipage} & \begin{minipage}[t]{2cm}0.\end{minipage} & \begin{minipage}[t]{12cm}Mass shift m-m\^{}* (GeV) for using in elementary particle collision mode.\end{minipage}\\*
\bottomrule
\end{longtable}
{ }



% YScalingAnalysis     code/analysis/yScalingAnalysis.f90

\begin{longtable}{llll}
\toprule
\textbf{\large{YScalingAnalysis}} & \multicolumn{3}{l}{\footnotesize{code/analysis/yScalingAnalysis.f90}}\\*
\midrule
\endfirsthead
\midrule
\endhead
analyze & \begin{minipage}[t]{2cm}logical\end{minipage} & \begin{minipage}[t]{2cm}.false.\end{minipage} & \begin{minipage}[t]{12cm}Determines wether the y-scaling analysis is performed\end{minipage}\\*
\midrule
optionalOutput & \begin{minipage}[t]{2cm}logical\end{minipage} & \begin{minipage}[t]{2cm}.false.\end{minipage} & \begin{minipage}[t]{12cm}Determines wether in addition to the standard 'scaling\_analysis.dat' other histograms will be generated. E.g. * 'single\_nucleon.dat' - a table for comparing nucleon-knockout with fully inclusive\\<pre>   x sections * 'scaling\_info.dat' - general parameters of the analysis, to be used for quick\\<pre>   analyis * 'scaling\_delta.dat' - output to be used for analyis of scaling function in\\<pre>   resonance excitation region\end{minipage}\\*
\midrule
variable & \begin{minipage}[t]{2cm}integer\end{minipage} & \begin{minipage}[t]{2cm}1\end{minipage} & \begin{minipage}[t]{12cm}determines which kind of scaling variable will be used (cf. Donnelly, Sick 1999):\begin{itemize}\leftmargin0em\itemindent0pt\item 1) RFG full variable Psi\item 2) RFG approximation Psi\item 3) PWIA full Upsilon (y/kf)\item 9) evaluation will be done for all variables, output written to   seperate files\end{itemize}\end{minipage}\\*
\midrule
kFermi & \begin{minipage}[t]{2cm}real\end{minipage} & \begin{minipage}[t]{2cm}0.2251\end{minipage} & \begin{minipage}[t]{12cm}Nucleon Fermi momentum in nucleus. If none specified 0.2251 will be used, except if densitySwitch\_static is set to 5, then fermiMomentum\_input is used. The 0.225\_1\_ aims at preventing confusion whith delibaretely set differences between kFermi and fermiMomentum\_input\end{minipage}\\*
\midrule
E\_shift & \begin{minipage}[t]{2cm}real\end{minipage} & \begin{minipage}[t]{2cm}0.020\end{minipage} & \begin{minipage}[t]{12cm}Energy correction to account for binding effects, otherwise neglected in RFG model\end{minipage}\\*
\bottomrule
\end{longtable}
{ }



% Yukawa     code/potential/yukawa.f90

\begin{longtable}{llll}
\toprule
\textbf{\large{Yukawa}} & \multicolumn{3}{l}{\footnotesize{code/potential/yukawa.f90}}\\*
\midrule
\endfirsthead
\midrule
\endhead
yukawaFlag & \begin{minipage}[t]{2cm}logical\end{minipage} & \begin{minipage}[t]{2cm}.false.\end{minipage} & \begin{minipage}[t]{12cm}Switches Yukawa potential on/off\end{minipage}\\*
\midrule
smu & \begin{minipage}[t]{2cm}real\end{minipage} & \begin{minipage}[t]{2cm}2.175\end{minipage} & \begin{minipage}[t]{12cm}Yukawa mass in fm**(-1). (range of potential)\end{minipage}\\*
\bottomrule
\end{longtable}
{ }



\end{document}
